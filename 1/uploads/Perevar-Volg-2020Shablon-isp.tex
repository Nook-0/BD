\documentclass[a4paper,11pt,twoside]{article}
\usepackage{vvmph2}
\usepackage{rumathbr}


\begin{document}

\udk{57.02.001.57, 517.929, 519.1, 27.35.43, 510.6}
\bbk{22.176}
\art{Моделирование пространственного развития инвазий в дискретной среде}{Modelling of spatial spreading of invasions in the discrete homogeneous environment}
\support{Работа выполнена в рамках проекта РФФИ 17-07-00125 и бюджетной темы СПИИРАН AAAA-A16-116051250009-8.} %команда может отсутствовать
\fio[Perevaryukha Andrey]{Переварюха}{Андрей}{Юрьевич}
\sdeg{Кандидат технических наук}{Ph.D. in Technical Sciences}
\pos{Старший научный сотрудник}{Senior Research Fellow}
\emp{Санкт--Петербургского института информатики и автоматизации РАН}{St. Petersburg Institute for Informatics and Automation of the Russian Academy of Sciences}
\email{temp$\_elf@$mail$.$ru}
\addr{199178, Санкт-Петербург, 14-Линия В.О., д. 39}{199178, St. Petersburg, 14-Line V.O., 39}

\maketitle

\begin{abstract}
Рассматривается проблема моделирования процессов биологических инвазий в пространстве с применением нового алгоритма размножения и гибели составляющих популяции отдельных клеток. Включение в непрерывную модель запаздывания $x(t-\tau)$ очевидный способ разнообразить варианты поведения траектории, не расширяя структуру и не увеличивая размерность фазового пространства. Использование популяционных моделей с отклоняющимся аргументом $\dot x =rf(x-\tau)- \Psi(x^k(t-\nu))$ в некоторых случаях не следует реалиям. Явная форма запаздывания пригодна для включения в феноменологические модели быстро созревающих видов. Актуальна пространственная модель, где временные факторы смогут задаваться наглядно. Цель работы --- исследовать алгоритм преобразования состояния клеток в пространстве квадратной решетки и получить нестационарную динамику кластеров двух популяций с явной интерпретацией параметров временного запаздывания. Для демонстрации путей развития инвазии с комплексом реалистичных факторов временного последействия предложен алгоритм клеточного автомата. Наш алгоритм не является очередной модификацией <<Жизни>> или <<Аква-Тор>>, так как используется окрестность с восемью соседними точками и три цвета клеток в квадратной решетке. За явление запаздывания в алгоритме отвечают параметры ограничения скорости размножения особей, обновления среды и время миграции новых особей к доступным им ресурсам. Проведена вычислительная реализация трансформации заданного начального состояния клеток при инвазии согласно правилам преобразования. Показан сценарий цикличности двух основных величин в системе. Возникновение или разрушение циклов зависит от скорости обновления светло-серых клеток. Формы трансформации состояния клеток подтверждают, что формализуемое запаздывание в модели Николсона в гораздо большей степени относится к динамике взаимодействия вида-вселенца и поддерживающей условия его существования среды. Действие запаздывания $\tau$ не имеет смыла при моделировании отождествлять с характеристикой непосредственно биологического вида. При выработке ответной реакции со стороны среды на инвазию запаздывание $\nu$ иное по сути, чем при восстановлении ресурсов. Практическая значимость заключается в моделировании перемещения гребня инвазионной волны и итоговой синхронизации пиков колебаний у противоборствующих видов-хищников {\it Beroe ovata} и {\it Mnemiopsis leidyi} в Черном и Азовском морях --- экологической системы <<хищник $A\Longleftrightarrow$ хищник $B$>>. Колебательное поведение численности двух популяций отличается от сценариев, которые можно получить в непрерывных моделях.
\end{abstract}

\keywords{алгоритмические модели, запаздывание в моделях процессов, критические сценарии популяционной динамики, механизмы регуляции, популяционные фронты инвазий, синхронизация колебаний хищников, Beroe ovata, распространение инфекций.}{algorithmic models, delay in process models, critical scenarios of population dynamics, regulation mechanisms, population invasion fronts, synchronization of predators oscillations, Beroe ovata, infections spreading.}

\section*{Введение}

Не все биологические процессы можно анализировать непрерывными моделями, хотя автор в предыдущих работах разрабатывал модели противоборства популяций в форме дифференциальных уравнений, но указывал их ограничения. Взаимное пространственное расположение объектов влияет на динамику, как живых клеток в составе органов или гнезд в колонии птиц. Биологические инвазии яркий пример экстремальных процессов с начальной пространственной неоднородностью, которые трудно описать диффузионными уравнениями из-за неустойчивости пространственной агрегации особей. Клеточные автоматы имеют преимущество наглядности при рассмотрении процессов распространения возмущений в сложной среде, но в тоже время оперируют условными фиксированными правилами преобразования типов клеток \cite{Kalm}. Известны примеры распада начального скопления образующейся популяции.  Длительно подавлять инвазивный сорняк амброзию путем интродукции листоеда Zygogramma suturalis \cite{Koval} не получилось, после быстрого угасания единичной популяционной волны жук более не образовывал фронты распространения высокой плотности и сейчас зигограмма редко встречается. Статья посвящена модели распространения популяции, где алгоритмически реализованы проблемные аспекты для моделирования биологических систем с запаздыванием и нелокальным гетерогенным ареалом.  Обсудим алгоритм автомата, который детерминировано опишет взаимодействие конкурирующих особей популяции  при их распространении из центра игрового поля в среде с жизненными ресурсами, которые будут исчерпываться и возобновляться различными темпами. Модифицированный алгоритм помогает выявить связь между сразу несколькими временными факторами. Мы исследуем влияние временных характеристик онтогенеза и миграции активных темно-серых клеток на образование пространственных неоднородностей в момент распространения исходной плотной группы в доступной среде и дальнейшее асимптотическую декластеризацию при распределении клеток на замкнутом игровом поле, когда доступность ресурсов естественно понижается и среда теряет исходную однородность.  В отличие от модели <<Аква-Тор>> с рыбами и акулами наши клетки не заинтересованы напрямую уничтожать друг-друга, хотя их совместное сосуществование будет вызывать гибель. Совместно рассмотрим факторы пространственного расселения и задержки онтогенетического развития в автоматной модели.  В отличие от анализа игры <<Жизнь>> вычислительное исследование будет проводиться с изучением влияния изменения темпа возобновления ресурсов клеток-<<деревьев>>, а не в зависимости от начальной конфигурации клеток, которая полагается в данном варианте стандартной.  Дискретная модель практически актуальна для классификации первых фаз в ситуациях активного распространения инвазивных видов: с фронтами высокой плотности и дальнейшем распаде плотных кластеров из-за медленного восстановления жизненных ресурсов.

\section{Проблема связи длинны жизненного цикла и периода колебаний численности}

Многие популяции могут генерировать колебания численности (в том числе нерегулярные) с периодом, который никак не связан длинной стадий жизненного цикла вида. Инвазии чужеродных видов могут происходить различным образом, в том числе по негативному для вселенца сценарию с его дальнейшим исчезновением, но основной сценарий предполагает негармонические колебания. 

В моделях видового противоборства $r$-параметры оказываются бифуркационными. Изменение одного влечет метаморфозы фазового портрета. В реальных устойчивых биосистемах все характеристики популяций --- результат коэволюции \cite{JOB}, они меняются не по отдельности, но согласовано. Если резко увеличить репродуктивную активность одной составляющей, то вместо колебаний с низким минимумом просто повысим риск вымирания другого трофического звена. Исчезновение одного вида отразится на устойчивости всей биосистемы. В системах уравнений резкие качественные изменения описываются именно параметрическими бифуркациями одного из параметров.  Непосредственные характеристики видов рыб и насекомых это плодовитость $\lambda$. Для непрерывных моделей вольтерровских трофических цепей не существует отдельных сезонных поколений или разделения убыли по возрастным группам.  Из классической флуктуационной модели Лотки-Вольтерра следует, что колебания у хищника и жертвы должны сдвигаться по фазе. Колебания согласно модели не могут быть полностью согласованы или строго зеркально находиться в противофазе --- сперва пик и жертвы, затем с опозданием следуем максимум хищника, как красиво показано в Википедии. В реальности противоборствующие виды совсем не следуют данному известному стандартному решению. В Черном море в конце 1980 гг. размножился уничтожающий зоопланктон гребневик-пришелец Mnemiopsis leidyi, потом появился его естественный враг Beroe ovata и сразу резко снизил его численность \cite{Finenko}. Возникла надежда, что процесс завершится как в экспериментах с инфузориями Г. Гаузе --- произойдет обоюдное исчезновение гребневиков, но естественная система в море гибче лабораторной. Два инвазионных гребневика в южной части Черного моря в настоящее время показывают синхронизированную динамику, но строго зеркальную с колебаниями популяции анчоуса \cite{Turka}. Вредоносный планктоноядный вселенец  Mnemiopsis leidyi и подавляющий его хищный теплолюбивый гребневик Beroe ovata в результате после переходного режима в неклассической системе <<хищник-жертва>> получили прямую согласованность появления популяционных максимумов рис. 1 из \cite{Bilio} у берегов Турции, что в северных границах их ареалов так отчетливо не наблюдается из больших сезонных перепадов температуры воды.  

\begin{figure}[h!] %
\centering
\includegraphics[scale=0.65]{beroe.eps}
\caption{Согласованная динамика колебаний биомассы двух чужеродных гребневиков в Черном море: 1 --- Mnemiopsis leidyi ; 2 --- Beroe ovata.}
\end{figure}

В Южном Каспии  Mnemiopsis leidyi после вспышки численности демонстрировал затухающие пилообразные осцилляции без влияния хищника Beroe ovata. Зависимость динамики  Mnemiopsis leidyi и планктона явно противофазная, минимум мнемиопсиса четко соотносится с максимумом планктона на юге Каспийского моря \cite{Variability}.

Хищник с течением процесса инвазии подстраивается под динамику вида-ресурса, которая сильно зависит от абиотических факторов, которые неоднородны в ареале гребневиков \cite{Hillis}. Beroe ovata уничтожает взрослых особей Mnemiopsis leidyi, но последний успешно может поедать планктонных личинок своего врага. Данная система отличается от обычной <<хищник$\rightarrow$ жертва>>, можно назвать <<хищник $A\Longleftrightarrow$ хищник $B$>>, где объект <<хищник A>> доминирующий вид, но отнюдь не неуязвимый. На ранних стадиях онтогенеза практически все хищные гидробионты становятся жертвами, потому существует в ихтиологии понятие длинна <<интервала уязвимости>> --- времени в жизненном цикле.

В известных классических лабораторных экспериментах G. Gause не успевала проходить никакая адаптация, хищная инфузория полностью уничтожала жертву, что мы обсуждали ранее.  Циклы легко получит в системе паразит-хозяин, где принципиально иные факторы регуляции, ведь паразит априори должен быть адаптирован к жизненному циклу хозяина. Есть группа факторов, которые нельзя учитывать традиционными уравнениями. Помимо коэволюции значимо стремление к пространственной агрегации на первых стадиях инвазии чужеродного вида. Стремление к объединению групп стабилизирует образующуюся популяцию \cite{Taksis}, что выгода от таксиса превышает потери от конкуренции. Именно создавая пространственную неоднородность с замедленным перемещением хищника удавалось в лабораториях получить колеблющийся двухвидовую биосистему.

Пространственная неоднородность популяций важный фактор для инвазивных процессов расселения чужеродных видов. Образующуюся популяцию агрессивного пришельца трудно рассматривать как усредненное распределение частиц по территории. В таких случаях существуют активные очаги и волны миграции за их пределы. Темпы распространения в среде зависят от наличия сконцентрированных ресурсов. 

Взаимодействие с подавляющими размножение видами-конкурентами не обязательно для появления флуктуаций. Математически мы не можем описать такое невынужденное осциллирующее поведение системой уравнений с бифуркацией Андронова-Хопфа. Существует два применяемых математических способа описать подобные варианты динамики (без импульсных воздействий, дельта-функций и других специальных дискретно-триггерных доопределений). Из многих примеров инвазий агрессивных чужеродных видов в неоднородном пространстве $\Omega$ понятно, что свойства $r$-параметра репродуктивной активности  просто не могут сохраняться
$\forall t,\Omega$. Фазы типичной инвазии предполагают сложную и пороговую зависимость репродуктивной активности и одновременно естественной смертности от численности. Для практики интересны более вероятные варианты --- резкое включение сопротивления биотического окружения в некоторый момент процесса, независимого от $r,\tau$-параметров вида-вселенца.

\section{Методы рассмотрения феномена регуляции с запаздыванием}

Так как в лабораторных экспериментах с изолированными от давления конкурентов популяциями условиях постоянны, то возникла идея в зависимости от предшествующего во времени состояния численности описать ту регуляцию, которая периодически сдерживала размножение. Проще говоря изменяла знак правой части. Некоторый универсальный интервал времени нашей модели, который отбрасывает действие сиюминутной регуляции приращения численности в прошлое состояние. Идея появилась у биологов ранее, чем математическая теория уравнений с $t-\tau$ была полностью разработана \cite{Bacaer}. С запаздыванием (можно включать  более одного) получают для систем малой размерности вычислительную реализацию многих нелинейных эффектов (включая хаотическое движение).  Для практического применения уравнений нужно оценить, что в своей сущности представляет собой данный интервал времени в форме запаздывания $t-\tau$, отбрасывающий на текущие значения системы гнет предшествующих состояний. Для задач оптимизации промыла актуально было бы обосновать биологическую природу подобного поведения, соотнеси интервал $(-\tau, 0]$с некоторой реальной характеристикой биосистемы, указать является ли величина времени отбрасывания вспять $\tau$ независимой константой для вида или появляется только в данной среде.

Запаздывание (точнее временное последействие прошлого состояния) появилось в математической биологии и теоретической экологии по причине того, что в ряде наблюдений и лабораторных экспериментов подтвердилось~\cite{Nicholson1954}, что флуктуации численности могут возникать у изолированных в аквариуме популяций. Искусственно поддерживаемая биологическая система получает ресурсы, не испытывает межвидового трофического взаимодействия и осиллирует --- образуется цикл $N_*(t;r\tau)$ со значительной амплитудной. Если быть скрупулезным, то цикл в экспериментах был получен в экологическом понимании этого термина, а не строго математическом. Для формализации появления популяционных колебаний одновидовой и саморегулируемой системы Хатчинсоном ~\cite{Hutchinson1978}
предложено известное логистическое уравнение Ферхюльста-Пирла просто дополнить запаздыванием, не меняя более ничего. Помимо начальных условий для непрерывной модели там нужно задать некоторую функцию-предысторию, но обычно указывают ее единичной константой. Проблема уравнений, что если нам нужно увеличить амплитуду колебаний, но минимумы цикла станут нереально глубокими $\min N_*(t;r\tau)\to0+\epsilon$, статистически для экологов не отличимыми от нуля. Со времен работ А.Д. Базыкина считается, что  популяционная модель должна сохранять адекватность только для некоторого существенного значения $N(t)>L$~\cite{Berezansky}. Для некоторой популяций необходима для продолжения роста надкритическая численность $L^+$. Само значение $N(t)=L$ в модели Базыкина равновесное $N\prime (t)=0$. Существование жесткого $L$-порога актуально, но не обязательно в конкретной ситуации. В мягкой трактовке порога уровень численности $L^-$ означает возникновение вероятности $\lim_{t\to\infty}N(t)=0$.  Так рыба ротан-головешка распространяется единичными экземплярами, опасные инфекции проникают в кровь штучными вирионами. Однако, не каждая попытка инвазии заканчивается успешно, иначе все вокруг было бы заражено гепатитом и заселено ротаном. Иммунную систему вирусу нужно обмануть, а рыбам найти безопасное убежище. После передачи нового зоонозного вирусного заболевания от животных в человеческую популяцию заразность вируса первое время очень высока, как сейчас происходит с распространением коронавируса 2019-nCoV из Китая, но экстремальная фаза эпидемий всегда заканчивается и затухает, как произошло с <<испанским гриппом>> в 1918 г. Однако, по имеющимся сообщениям стойкий иммунитет к повторному заражению мутантным 2019-nCoV не вырабатывается, потому для такого алгоритма <<иммунные>> клетки окажутся неактуальны, но тогда станет возможно тривиальное итоговое равновесие в игровом поле.

В стохастической трактовке судьбы популяций считается, что на минимальных значениях $N(t)\to L$ эффективность размножения инвазионных организмов это вероятностная величина. С точки зрения теории вероятности удачная рекомбинация вирусных белков из нескольких разных штаммов очень маловероятное событие, нужно несколько редких событий сразу, что бы куски РНК успешно перестроились в одной клетке у летучей мыши и рядом оказался восприимчивый человек --- с которого началась эпидемия в Китае. Прогнозировать такие явления невозможно, вот потому мы предпочитаем детерминированные модели для биологических задач.

Обзор известных решений и существующих интересных проблем популяционных уравнений с запаздыванием дан в \cite{Obzor}, оставшиеся трудности приведены в \cite{Koles2010}. Каждый метод моделирования обладает своими недостатками, все зависит от уровня детализации решаемых задач. В дискретных моделях промысловой ихтиологии логично разделяют убыль и пополнение, не агрегирую их разность одним $r$-параметром. Энтомологи оценивают выживаемость в процентах для каждой стадии развития вредителей от $N(0)$. В непрерывной модели можно заменить константу $r$ периодической функцией $\omega_r (t), \exists t : \omega_r (t) < 0$. Так можно учесть сезонность --- внешний, независимый от биотического взаимодействия фактор, вызывающий периодические изменения в преобладании убыли над рождаемостью.
Масштабные вспышки численности вредителей периодически заканчиваются дефолиацС‚ <<хищник A>> доминирующий РІРёРґ, РЅРѕ отнюдь РЅРµ неуязвимый. РќР° ранних стадиях онтогенеза практически РІСЃРµ хищные гидробионты становятся жертвами, потому существует РІ ихтиологии понятие длинна <<интервала уязвимости>> --- времени РІ жизненном цикле.

Р’ известных классических лабораторных экспериментах G. Gause РЅРµ успевала проходить никакая адаптация, хищная инфузория полностью уничтожала жертву, что РјС‹ обсуждали ранее.  Циклы легко получит РІ системе паразит-С…РѕР·СЏРёРЅ, РіРґРµ принципиально иные факторы регуляции, ведь паразит априори должен быть адаптирован Рє жизненному циклу С…РѕР·СЏРёРЅР°. Есть РіСЂСѓРїРїР° факторов, которые нельзя учитывать традиционными уравнениями. РџРѕРјРёРјРѕ коэволюции значимо стремление Рє пространственной агрегации РЅР° первых стадиях инвазии чужеродного РІРёРґР°. Стремление Рє объединению РіСЂСѓРїРї стабилизирует образующуюся популяцию \cite{Taksis}, что выгода РѕС‚ таксиса превышает потери РѕС‚ конкуренции. Р