%\documentclass[a4paper,11pt,twoside]{article}
%\usepackage{vvmph2}
%\usepackage{rumathbr}
%\usepackage{tabularx}
%\usepackage{multirow}
%\graphicspath{ {images/} }
%
%
%\begin{document}

\udk{004.89}
\bbk{55.6}
\art[Интеллектуальный анализ данных микроволновой радиотермометрии]{Интеллектуальный анализ данных микроволновой радиотермометрии \plb в диагностике рака молочной железы}{Data mining of microwave radiometry data \plb in the diagnosis of breast cancer}
\support{Работа выполнена при финансовой поддержке РФФИ и Администрации Волгоградской области (проект № 15-47-02475-р\_поволжье\_а)}

\fio{Лосев}{Александр}{Георгиевич}
\sdeg{доктор физико-математических наук}{Doctor of Physical and Mathematical Sciences}
\pos{профессор кафедры математического анализа и теории функций}{Professor, Department of Mathematical Analysis and Function Theory}
\emp{Волгоградский государственный университет}{Volgograd State University}
\email{alexander.losev@volsu.ru}
\addr{просп. Университетский, 100, 400062 г. Волгоград, Российская Федерация}{Prosp. Universitetsky, 100, 400062 Volgograd, Russian Federation}

\fio{Левшинский}{Владислав}{Викторович}
\pos{студент}{student}
\emp{Волгоградский государственный университет}{Volgograd State University}
\email{vladi.lev.email@gmail.com}
\addr{просп. Университетский, 100, 400062 г. Волгоград, Российская Федерация}{Prosp. Universitetsky, 100, 400062 Volgograd, Russian Federation}

\maketitle

\begin{abstract}
Работа посвящена разработке нового метода формирования пространства диагностических признаков по данным микроволновой радиотермометрии, предназначенного для создания на его основе консультативной интеллектуальной диагностической системы. Предъявлен метод формирования высокоинформативных признаков на базе количественного описания медицинских знаний и математических моделей поведения температурных полей молочных желез.
\end{abstract}

\keywords{интеллектуальный анализ данных, микроволновая радиотермометрия, консультативные интеллектуальные системы, высокоинформативные признаки, рак молочной железы}{data mining, microwave radiothermometry, intelligent advisory systems}

\section*{Введение}

Одной из актуальных задач медицины и информационных технологий является разработка и внедрение эффективных систем поддержки принятия решений, которые используя методы интеллектуального анализа данных, помогают специалистам в задачах постановки диагнозов, прогнозирования развития заболеваний и т.д.  В подавляющем большинстве случаев, применение современного медицинского оборудования, решая одни проблемы --- порождает другие. В настоящее время трудности диагностики возникают не из-за дефицита информации, а из-за недостаточной эффективности методов ее обработки. В какой-то мере, решение указанных проблем обеспечивает создание систем интерпретации и анализа медицинских данных. При этом наибольший интерес вызывает разработка консультативных интеллектуальных систем, то есть экспертных систем, содержащих механизм объяснения и обоснования предлагаемых решений на языке, понятном пользователю \cite{kobrinsky1}.

Целью данного исследования является разработка некоторых методов анализа и интерпретации медицинских данных, получаемых с помощью микроволновой радиотермометрии.

\section{Описание задачи}

Одним из действенных способов повышения эффективности диагностики является автоматизация обработки медицинских данных с помощью методов искусственного интеллекта \cite{yasnitsky,burke,kononenko1,kononenko2,lesmo}. Однако, большинство экспертных систем свои решения предлагает врачу или в детерминированной форме однозначного заключения, или в виде вероятностных оценок каждого из возможных диагнозов. В настоящее время идет процесс формирования новых подходов к созданию диагностических систем, <<объясняющих>> предлагаемые ими решения \cite{kobrinsky1,kobrinsky2}.

Одним из наиболее перспективных методов функциональной диагностики, на основе которого вполне возможно создание эффективной консультативной интеллектуальной системы является микроволновая радиотермометрия. Это биофизический метод неинвазивного обследования, заключающийся в измерении внутренних и поверхностных температур тканей по интенсивности их теплового излучения, соответственно, в микроволновом (РТМ) и инфракрасном (ИК) диапазонах. В течение последнего десятилетия данный метод получил распространение в различных областях медицины \cite{losevhoperskov}. Однако, существующий на данный момент диагностический комплекс РТМ-01-РЭС является системой поддержки принятия решений специалиста высокой квалификации. Сложность восприятия информации, возникающая у медицинского персонала без специальной длительной подготовки, значительно снижает потенциальную возможность его широкого использования в скрининге. Таким образом, актуальной задачей является создание экспертной системы, обладающей возможностями обоснования предполагаемого диагностического решения.

Особую сложность в данной проблеме вызывает нахождение
высокоинформативных признаков заболеваний. Как отмечается
большинством специалистов, на будущее качество алгоритмов
классификации влияют качественный и количественный составы
пространства информационных признаков. Основной задачей данного
исследования является разработка метода формирования пространства
информационных признаков.

\section{Качественные составляющие информационных признаков}

Вначале опишем подробнее существующую методику диагностики рака молочной железы по данным микроволновой радиотермометрии.
Комплекс РТМ-01-РЭС позволяет оценивать функциональное состояние
тканей путем измерения внутренней температуры (РТМ) на глубине до
5 см. и температуры кожи (ИК). Обследование пациентки (см.,
например, \cite{vaysblat}) проводится в горизонтальном положении,
обнаженной по пояс, руки под головой. Обследование начиналось с
измерения температур в опорных точках T1 и T2, расположенных,
первая ---  в центре грудной клетки сразу под и между молочными
железами, вторая --- непосредственно под мечевидным отростком. Далее
измерения проводятся в 10 точках на каждой железе, и в аксилярной
области. При этом,  получаемые данные сразу выводятся на экран
монитора (см. рис. 1).


\begin{figure}[!h]\begin{center}
\includegraphics[scale=0.5]{met12.jpg}
\caption{Методика обследования молочных желез}
\end{center}\end{figure}

%(схема представлена на Рис. 1. ).

После анализа информации о температурных полях молочных желез, в
частности с помощью термокарт (см., например, рис. 2), врач ставит
диагноз пациентке. Здесь, каждое значение температуры передается
на экране монитора своим цветом. Участки с пониженной
температурой, передаются <<холодными>> цветами (синим), а с
повышенной температурой --- <<теплыми>> цветами (розовым, красным).

\begin{figure}[!h]
\center{\includegraphics[scale=0.5]{termo1.jpg}}
\caption{Термокарты}
\vspace*{5mm}
\end{figure}




%Комплекс РТМ-01-РЭС позволяет оценивать функциональное состояние тканей путем измерения внутренней температуры (РТМ; MW) на глубине до 5 см. и температуры кожи (ИК; IR). Обследование пациентки %начинается с измерения температур в опорных точках T1 и T2, расположенных, первая --- в центре грудной клетки сразу под и между молочными железами, вторая --- непосредственно под мечевидным отростком. %Далее измерения проводятся в 10 точках на каждой железе, и в аксиллярной области (схема представлена на рис. \ref{fig:img1}).

%\begin{figure}[h]
%   \centering
%   \includegraphics[width=\textwidth]{img1}
%   \caption{Схема обследования молочных желез}
%   \label{fig:img1}
%\end{figure}

На основе данных, предоставленных онкологическими центрами России,
была сформирована экспертная база термометрических данных. В
настоящее время она включает в себя информацию о 734 молочных
железах пациенток, которые делятся на два контрольных класса:
<<Здоровые>> --- 148 молочных желез и <<Больные>> --- 586 молочных
желез. Класс <<Больные>> в свою очередь делится на несколько
групп: <<Узловой рак>> (185 молочных желез), <<Диффузный рак>> (13
молочных желез), <<Узловые изменения, но не рак>> (90 молочных
желез), <<Диффузные изменения, но не рак>> (125 молочных желез),
<<Неотдифференцированные гистологически>> (8 молочных желез),
<<Норма-2>> (165 молочных желез). В класс <<Норма-2>> входят
термометрические данные здоровых молочных желез больных пациенток.
В рамках данного исследования молочные железы пациенток в базе
были разделены на два новых класса: <<Рак>> --- 326 молочных желез и
<<Не рак>> --- 408 железы. Статистический анализ используемых
термометрических данных был проведен в работе \cite{losevmazepa}.



В ходе исследований и анализа данных, специалистами были выявлены следующие признаки рака молочной железы \cite{vaysblat,vesnin,losevzamechnik} (далее будем называть их качественными):
\begin{itemize}
    \item повышенная величина термоасимметрии между одноименными точками молочных желез;
    \item повышенный разброс температур между отдельными точками в пораженной молочной железе;
    \item разница температур сосков;
    \item повышенная температура соска в пораженной молочной железе по сравнению со средней температурой молочной железы с учетом возрастных изменений температуры;
    \item соотношение кожной и глубинной температур и некоторые другие.
\end{itemize}

Важным этапом создания эффективной консультационной
интеллектуальной системы является математическое описание данных
признаков, выявление их количественных характеристик, а также
выявление новых признаков.

\begin{enumerate}
    \item Группа признаков, характеризующая асимметрию температурных полей молочных желез.

Данный класс признаков исходит из гипотезы о<<зеркальной>> симметрии температурных полей правой и левой молочных желез
здоровых пациенток. Данная гипотеза используется при анализе
термометрических данных не только молочных желез, но практически
всех парных органов человека \cite{losevhoperskov}. Опишем вначале
известные признаки заболеваний, относящиеся к данному классу.
    \begin{enumerate}
        \item[1.1)] Повышенное значение разности температур между одноименными точками правой и
        левой молочных желез.

        В качестве характеристик, описывающих этот эффект, могут быть использованы функции вида
        \[|t_{i, r} - t_{i, l}|,\] или \[t_{i, r} - t_{i, l},\]
        где \(t_{i, r}\) и \(t_{i, l}\) --- температуры в i-ых точках правой и левой молочных
        желез соответственно.
                \item[1.2)] Повышенная разница температур сосков правой и левой молочных
                желез: \[|t_{0, r} - t_{0, l}|,\] или \[t_{0, r} - t_{0,
                l}.\]
        \item[1.3)] Повышенное среднеквадратичное значение разностей температур между одноименными
        точками правой и левой молочных желез:
        \[\sqrt{\sum_{i=0}^{8}\frac{(t_{i, r} - t_{i,
        l})^2}{9}}.\]
        В общем виде асимметрия полей температур молочных желез может быть описана
        различными функциями вида
        \[g(f(t_{i, r}, \dots, t_{n, r})-f(t_{i, l}, \dots, t_{n,
        l})),\]
        где \(f(t_{i, r}, \dots, t_{n, r})\) --- значение функции температур для
        точек \(t_{0}, \dots, t_{n}\) правой молочной железы,
        \(f(t_{i, l}, \dots, t_{n, l})\) --- значение аналогичной функции температур для
        точек \(t_{0}, \dots, t_{n}\) левой молочный железы, а \(g(x)\) --некоторая функция
        одного переменного. Таким образом, значительно расширяется множество исследуемых
        параметров. В частности, таким способом можно описать следующие характеристики.
        \item[1.4)] Разница средних значений температур<<зеркально-симметрично>> расположенных
        подобластей молочных желез, например:
        \[|\frac{t_{0, MW, r} + t_{i, MW, r} + t_{(i \mod 8) + 1, MW, r}}{3} -
        \frac{t_{0, MW, l} + t_{i, MW, l} + t_{(i \mod 8) + 1, MW, l}}{3}|\]
        \(i = 1, \dots, 8.\)
        \item[1.5)] Разница среднеквадратичных отклонений температур молочных желез:
        \[|(\sum_{i=1}^{8}\frac{(t_{i, MW} - t_{MW, m})^2}{8})_{r} - (\sum_{i=1}^{8}
        \frac{(t_{i, MW} - t_{MW, m})^2}{8})_{l}|,\]
        где \(t_{MW, m} = \sum_{i=1}^{8}\frac{t_{i}}{8}\) --- среднее значение
        точек \(1, \dots, 8\) правой или левой молочных желез соответственно и т.д.
    \end{enumerate}
    \item Группа признаков, характеризующих повышенный разброс температур в пораженной молочной
    железе.
    \begin{enumerate}
        \item[2.1)] Повышенное среднеквадратичное отклонение температур в одной из молочных желез:
        \[\sqrt{\sum_{i=0}^{8}\frac{(t_{i} -
        \overline{t_{m}})^2}{9}},
        \]
        где \(\overline{t_{m}} = \sum_{i=0}^{8}\frac{t_{i}}{9}
        \).
        \item[2.2)] Повышенный разброс температур между отдельными точками в пораженной молочной железе:
        \[t_{m} - t_{i},\]
        где \(t_{m} = \sum_{i=1}^{8}\frac{t_{i}}{8}\), \(t_{i}\) --- температура в i-ой точке
        молочной железы.

        В общем виде разброс температур молочных желез может быть описан различными функциями
        вида
        \[g(f_{1}(t_{0}, \dots, t_{n}) - f_{2}(t_{0}, \dots,
        t_{n})),\]
        где \(f_{1}(t_{0}, \dots, t_{n})\) и \(f_{2}(t_{0}, \dots, t_{n})\) --- функции
        температур точек \(t_{0}, \dots, t_{n}\) соответствующей молочной железы, а \(g(x)\) ---         некоторая функция одного переменного. В частности, таким способом можно описать
        следующие характеристики.
        \item[2.3)] Повышенные средние значения температур соседних точек вплоть до средней температуры молочной железы, то есть функции вида:
        \[t_{m} - \frac{t_{i} + t_{(i \mod 8) + 1} + t_{(i \mod 8) + 2}}{3} \]
        и т.п.
    \end{enumerate}
    \item Группа признаков, характеризующих повышенное значение температуры соска в пораженной молочной железе.
    \begin{enumerate}
        \item[3.1)] Аномальная разность температуры соска и средней температуры молочной железы:
        \[t_{0} - t_{m},\]
        где \(t_{m} = \sum_{i=1}^{8}\frac{t_{i}}{8}\).
        \item[3.2)] Аномальная разность температуры соска и температур отдельных точек молочной железы:
        \[t_{0} - t_{i}, i = 1, \dots, 8. \]
        В целом, аномальные значения температуры соска по отношению к другим параметрам можно описать функциями вида
        \[g(t_{0} - f_{1}(t_{0}, \dots, t_{n})),\]
        где \(f_{1}(t_{0}, \dots, t_{n})\) --- функция температур точек \(t_{0}, \dots, t_{n}\) молочной железы, а \(g(x)\) --- некоторая функция одного переменного. В частности, таким способом можно описать следующие характеристики.
        \item[3.3)] Аномальная разность температуры соска и средней температуры различных подобластей молочной железы, например:
        \[t_{0} - \frac{t_{i} + t_{(i \mod 8) + 1}}{2} \]
        и т.п.
    \end{enumerate}
    \item Группа признаков, характеризующих соотношение кожной и глубинной температур:
    \begin{enumerate}
        \item[4.1)] Аномальное значение разности между кожной и глубинной температурами точки пораженной молочной железы (внутренний градиент):
        \[t_{i, MW}-t_{i, IR},\]
        где \(t_{i, MW}\) --- глубинные и \(t_{i, IR}\) --- кожные температуры в i-ой точке молочной железы.
        Разность температур молочных желез, измеренных в РТМ и ИК диапазонах (так называемый
        внутренний градиент), может быть описана функциями вида
        \[g(f(t_{0, MW}, \dots, t_{n, MW}) - f(t_{0, IR}, \dots, t_{n,
        IR})),\]
        где \(f(t_{0}, \dots, t_{n}\) --- функция температур точек \(t_{0}, \dots, t_{n}\) молочной железы, а \(g(x)\) --- некоторая функция одного переменного.

        Отметим, что сформированный выше набор функций представляет собой расширенное описание известных качественных признаков, и получен на базе уже известных медицинских фактов. Но особенно важной, хотя и достаточно сложной задачей, является выявление новых знаний.
    \end{enumerate}
    \item Признаки, базирующиеся на параметрах физико-математических моделей поведения температурных полей.

    В течение последних лет было построено несколько математических моделей,
    описывающих поведение температурных полей молочных желез с помощью уравнений в
    частных производных второго порядка \cite{losevhoperskov, polhop1}.
    Кроме того отметим, что в предъявленном выше наборе функций присутствуют как функции
    температур, так и разностные аналоги их производных по различным направлениям.
    Например, величина \(t_{0} - t_{i}\) является разностным аналогом производной в
    радиальном направлении (радиальный градиент). Аналогично,
    величина \(t_{i, MW} - t_{i, IR}\) является разностным аналогом производной во внутреннем
    направлении (внутренний градиент).

    Учитывая вышесказанное, в работе было решено рассматривать и разностные аналоги вторых производных функций температур, то есть функции вида:
    \[\nabla(t_{0} - f(t_{0}, \dots, t_{n})) = t_{0, MW} - f(t_{1, MW}, \dots, t_{n, MW}) -
    t_{0, IR} + f(t_{1, IR}, \dots, t_{n, IR}). \]
\end{enumerate}

Перейдем к формальному описанию задачи поиска информативных
признаков. Введем следующие обозначения.

Пусть \(t_{i}^{j}, i = 0, \dots, 9\) --- температуры i-ой точки правой молочной железы j-ой пациентки в диапазоне РТМ;

\(t_{i}^{j}, i = 10, \dots, 19\) --- температура \([i - 10]\) точки правой молочной железы j-ой пациентки в диапазоне ИК;

\(t_{i}^{j}, i = 20, \dots, 29\) --- температура \([i - 20]\) точки левой молочной железы j-ой пациентки в диапазоне РТМ;

\(t_{i}^{j}, i = 30, \dots, 39\) --- температура \([i - 30]\) точки левой молочной железы j-ой пациентки в диапазоне ИК;

\(t_{40}^{j} = T1, t_{41}^{j} = T2\) --- глубинные температуры опорных точек j-ой пациентки;

\(t_{42}^{j} = T1, t_{43}^{j} = T2\) --- кожные температуры опорных точек j-ой пациентки.

Тогда обучающую выборку можно представить в виде матрицы
\[
S =
\begin{pmatrix}
t_{0}^{1} & \dots & t_{43}^{1} \\
\vdots & \ddots & \vdots \\
t_{0}^{k} & \dots & t_{43}^{k} \\
t_{0}^{k + 1} & \dots & t_{43}^{k + 1} \\
\vdots & \ddots & \vdots \\
t_{0}^{n} & \dots & t_{43}^{n} \\
\end{pmatrix}
,\] где \(t_{i}^{j}\) при \(j = 1, \dots, k\) --- температуры
молочных желез пациенток класса <<Не рак>>, \(t_{i}^{j}\) при \(j =
k + 1, \dots, n\) --- температуры молочных желез пациенток класса
<<Рак>>.

Пусть \(f^{q}\) --- q-я исследуемая функция, рассматриваемая на множестве векторов \(\Omega_{q} = {(t_{0}^{j}, \dots, t_{43}^{j})}\), где \(j = 1, \dots, n\). Обозначим \(f_{j}^{q} = f^{q}(t_{0}^{j}, \dots, t_{43}^{j})\).

Высокоинформативным признаком будем называть тройку \((f^{q}, V, X)\), где \(f^{q}\) --- функция, описывающая поведение температурных полей, \(V = I(f^{q}, X)\) --- информативность признака, X --- <<информативная>> область множества значений функции \(f^{q}\).

Определим указанные выше характеристики. Под информативностью
традицционно понимается количественный параметр, определяющий,
насколько хорошо закономерность описывает различия между искомой и
отделяемой группами. В качестве \(I(f^{q}, X)\) в данной работе
использовался ряд характеристик, в том числе следующие
функционалы.

Статистическая информативность, вычисляемая по формуле
\[ST(f^{q}, x) = -\ln(\frac{C_{k}^{h}C_{n - k}^{s}}{C_{k + (n - k)}^{h +
s}}),\] где n --- количество векторов с температурными данными
пациенток в обучающей выборке, k --- количество векторов с
температурными данными пациенток класса <<Не рак>> в обучающей
выборке, h --- количество молочных желез класса <<Не рак>>, для
которых \(f^{q} \in X\), а s --- количество молочных желез класса
<<Рак>>, для которых \(f^{q} \in X.\)

Эвристическая информативность, вычисляемая по формуле
\[HR(f^{q}, x) = \frac{\max(\frac{k}{h}, \frac{n - k}{s})}{\min(\frac{k}{h}, \frac{n - k}{s})}.\]


Энтропийная информативность, вычисляемая по формуле
\begin{align*}
EN(f^{q}, x) = H(\frac{k}{k + (n - k)}, \frac{n - k}{k + (n - k)})\\
 - \frac{k + (n - k) - h - s}{k + (n - k)}H(\frac{k - h}{k + (n - k) - h - s},
 \frac{(n - k) - s}{k + (n - k) - h - s}),
\end{align*}
где \(H(q_{0}, q_{1}) = -q_{0}\log_{2}(q_{0}) -q_{1}\log_{2}(q_{1})\) --- математическое ожидание количества информации.

Комбинированная информативность, которая вычисляется по формуле
\[CI(f^{q}, X) = \sqrt[n]{(I_{1}(f^{q}, X) I_{2}(f^{q}, X) \dots I_{n}(f^{q},
X))},\] где \(I_{1}, I_{2}, \dots, I_{n}\) --- другие
информативности.

Среднее гармоническое информативностей, которое вычисляется по
формуле
\[HM(f^{q}, X) = \frac{n}{\frac{1}{I_{1}(f^{q}, X)} + \dots + \frac{1}{I_{n}(f^{q}, X)}}.\]


Далее, пусть
\[I_{s} = \sup_{X_{\alpha}}I(f^{q}, X),\]
где \(X_{\alpha}\) --- все возможнe{t_{m}} = \sum_{i=0}^{8}\frac{t_{i}}{9}
        \).
        \item[2.2)] Повышенный разброс температур между отдельными точками в пораженной молочной железе:
        \[t_{m} - t_{i},\]
        где \(t_{m} = \sum_{i=1}^{8}\frac{t_{i}}{8}\), \(t_{i}\) --- температура в i-ой точке
        молочной железы.

        В общем виде разброс температур молочных желез может быть описан различными функциями
        РІРёРґР°
        \[g(f_{1}(t_{0}, \dots, t_{n}) - f_{2}(t_{0}, \dots,
        t_{n})),\]
        где \(f_{1}(t_{0}, \dots, t_{n})\) и \(f_{2}(t_{0}, \dots, t_{n})\) --- функции
        температур точек \(t_{0}, \dots, t_{n}\) соответствующей молочной железы, а \(g(x)\) ---         некоторая функция одного переменного. В частности, таким способом можно описать
        следующие характеристики.
        \item[2.3)] Повышенные средние значения температур соседних точек вплоть до средней температуры молочной железы, то есть функции вида:
        \[t_{m} - \frac{t_{i} + t_{(i \mod 8) + 1} + t_{(i \mod 8) + 2}}{3} \]
        Рё С‚.Рї.
    \end{enumerate}
    \item Группа признаков, характеризующих повышенное значение температуры соска в пораженной молочной железе.
    \begin{enumerate}
        \item[3.1)] Аномальная разность температуры соска и средней температуры молочной железы:
        \[t_{0} - t_{m},\]
        РіРґРµ \(t_{m} = \sum_{i=1}^{8}\frac{t_{i}}{8}\).
        \item[3.2)] Аномальная разность температуры соска и температур отдельных точек молочной железы:
        \[t_{0} - t_{i}, i = 1, \dots, 8. \]
        В целом, аномальные значения температуры соска по отношению к другим параметрам можно описать функциями вида
        \[g(t_{0} - f_{1}(t_{0}, \dots, t_{n})),\]
        где \(f_{1}(t_{0}, \dots, t_{n})\) --- функция температур точек \(t_{0}, \dots, t_{n}\) молочной железы, а \(g(x)\) --- некоторая функция одного переменного. В частности, таким способом можно описать следующие характеристики.
        \item[3.3)] Аномальная разность температуры соска и средней температуры различных подобластей молочной железы, например:
        \[t_{0} - \frac{t_{i} + t_{(i \mod 8) + 1}}{2} \]
        Рё С‚.Рї.
    \end{enumerate}
    \item Группа признаков, характеризующих соотношение кожной и глубинной температур:
    \begin{enumerate}
        \item[4.1)] Аномальное значение разности между кожной и глубинной температурами точки пораженной молочной железы (внутренний градиент):
        \[t_{i, MW}-t_{i, IR},\]
        где \(t_{i, MW}\) --- глубинные и \(t_{i, IR}\) --- кожные температуры в i-ой точке молочной железы.
        Разность температур молочных желез, измеренных РІ Р РўРњ Рё Р