\documentclass[a4paper,11pt,twoside]{article}
\usepackage{vvmph2} 

%\documentclass[a4paper,11pt,twoside]{article}
%\usepackage{mpcs}
%\usepackage{rumathbr}


\begin{document}

\udk{519.632.4}
\bbk{22.19}

\art{Исследования в области геометрического анализа в Волгоградском государственном университете}{Research in the field of geometric analysis at Volgograd State University}
\support{Работа выполнена при поддержке Математического Центра в Академгородке, соглашение с Министерством науки и высшего образования Российской Федерации номер 075-15-2019-1613.}
\fio{Клячин}{Алексей}{Александрович}
\sdeg{доктор физико-математических наук}{Doctor of Physical and Mathematical Sciences}
\pos{заведующий кафедрой математического анализа и теории функций}{Head of Department of Mathematical Analysis and Function Theory}
\emp{Волгоградский государственный университет}{Volgograd State University}
\email{Aleksey.klyachin@volsu.ru, klyachin-aa@yandex.ru}
\addr{Просп. Университетский, 100, 400062 г. Волгоград, Российская Федерация}{Prosp. Universitetsky, 100, 400062 Volgograd, Russian Federation}

\fio{Клячин}{Владимир}{Александрович}
%\sdeg{}{}
\sdeg{доктор физико-математических наук}{Doctor of Physical and Mathematical Sciences}
\pos{заведующий кафедрой компьютерных наук и экспериментальной математики}{Head of the Department of Computer Science and Experimental Mathematics}
\emp{Волгоградский государственный университет}{Volgograd State University}
\email{klyachin.va@volsu.ru, klchnv@mail.ru}
\addr{Просп. Университетский, 100, 400062 г. Волгоград, Российская Федерация}{Prosp. Universitetsky, 100, 400062 Volgograd, Russian Federation}

\maketitle

\begin{abstract}
В настоящей статье рассмотрены основные направления исследований по геометрическому анализу, которые проводились и проводятся научной математичексой школой Волгоградского государственного унверситета.
Вкратце изложены результаты основоположника нучной школы доктора физико-математических наук, профессора Владимира Михайловича Миклюкова и его учеников. Эти результаты касаются решения ряда задач в области квазиконформных плоских отображений и отображений с ограниченным искажением поверхностей и римановых многообразий, теории минимальных поверхностей и поверхностей предписанной средней кривизны, поверхностей нулевой средней кривизны в лоренцевых пространствах, а также, задач, связанных с исследованием устойчивости такого рода поверхностей. Кроме этого, отмечеы результаты изучения различных классов триангуляций -- объекта, возникающего на стыке исследований в оласт геометрического анализа и вычислительной математики.
Кроме этого, в данном обзоре рассматриваются работы, в которых дано  применение  метода Фурье разложения 
решений уравнений Лапласа-Бельтрами 
и стационарного уравнения Шредингера  по собственным функциям соответствующих краевых задач. В
частности, приведены результаты
по нахождению емкостных характеристик, которые позволили впервые сформулировать 
и доказать критерии выполнения различных теорем типа Лиувилля и разрешимости краевых задач на 
модельных и квазимодельных римановых многообразиях. Также указывается роль метода эквивалентных функций 
при исследовании подобных задач на многообразиях достаточно общего вида.

В данной статье помимо этого дается обзор результатов, касающихся оценок погрешности вычисления
 интегральных функционалов и сходимостью кусочно-полиномиальных решений нелинейных
 уравнений вариационного типа: уравнения минимальной поверхности, уравнения равновесной
 капиллярной поверхности и уравнения бигармонических функций. 
\end{abstract}

\keywords{геометрический анализ, минимальные поверхности, емкость, гармонические функции, триангуляция, интегральный функционал}{geometric analysis, minimal surfaces, capacity, harmonic functions, triangulation, integral functional}




Одним из направлений развития математики в Волгоградском государственном университете стали исследования в области геометрического анализа -- области математики, возникшей  на стыке геометрии, математического анализа и дифференциальных уравнений. Здесь тесно переплелись методы квазиконформных отображений, задачи геометрического строения минимальных поверхностей и исследования асимптотического поведения решений эллиптических уравнений в неограниченных областях или в окрестности особой точки. Самую важную роль в становлении геометрического анализа в ВолГУ сыграли исследования  доктора физико-математических наук, профессора Владимира Михайловича Миклюкова. Разработанные им методы нелинейной емкости позволили решить широкий круг задач,  связанных с поведением решений нелинейных уравнений в пространстве и на многообразиях \cite{32}, \cite{37}, \cite{38}, \cite{45} -- \cite{49}, \cite{51} -- \cite{54}. Так же возник новый объем задач, который  включал в себя вопросы протяженности минимальных трубок и лент в евклидовом и псевдоевклидовом пространствах, проблемы конформного типа поверхностей, исследования пространственно подобных трубок и лент нулевой средней кривизны, их устойчивости/неустойчивости при малых деформациях, времени существования, точек ветвлений, связей между точками ветвления и лоренцево инвариантными характеристиками поверхностей \cite{34}, \cite{36}, \cite{41} -- \cite{44}, \cite{50}, \cite{55}, \cite{56}. Сюда же мы относим изучение теорем типа Фрагмена-Линделефа для дифференциальных форм, теорем типа Альфорса для дифференциальных форм с конечным/бесконечным числом различных асимптотических трактов; теорем типа теоремы Вимана для квазирегулярных отображений многообразий; применение изопериметрических методов в принципе Фрагмена-Линделефа для квазирегулярных отображений многообразий \cite{33}, \cite{35}, \cite{39}, \cite{40}, \cite{57}.

	Под руководством В.М.Миклюкова получила развитие теория минимальных поверхностей и поверхностей нулевой средней кривизны в псевдоевклидовых пространствах  благодаря применению таких современных методов математического анализа как емкостная и модульная техника. Усовершенствованная емкостная и модульная техника позволила ему решить ряд задач нелинейного анализа в его докторской диссертации.  В частности, им получены значительные результаты о граничных свойствах непараметрических минимальных поверхностей, был дан новый подход к проблеме Бернштейна для уравнения типа минимальной поверхности. Фундаментальным результатом явился результат о суммарном топологическом индексе критических точек для решений уравнения типа минимальной поверхности.
	
	Исследования В.М.Миклюкова, касающиеся строения трубок и лент нулевой средней кривизны в евклидовом пространстве и в  пространстве Минковского содержат такие результаты, как теоремы типа Фрагмена-Линделефа, теоремы Лиувилля на поверхностях нулевой средней кривизны и их приложения к геометрическому строению этих поверхностей. В этих работах нашло, в частности,  развитие теории простых концов на случай поверхностей нулевой средней кривизны в евклидовом и псевдоевклидовом пространствах. В работах В.М.Миклюкова найдены общие подходы к изучению асимптотического поведения решений нелинейных уравнений эллиптического типа, как в окрестности конечной точки, так и в бесконечно удаленной. Разработанными методами им получены различные свойства решений таких уравнений (неравенство Гарнака, принцип Сен-Венана, непрерывность по Гельдеру и др.) В последнее время эти результаты В.М.Миклюковым были распространены на случай почти-решений эллиптических уравнений и систем. В частности, им получено неравенство Гарнака. Найденные им новые подходы к исследованию асимптотического поведения решений эллиптических уравнений были успешно применены его учениками, В.Г. Ткачевым, А.А.Клячиным, В.А. Клячиным в изучении строения минимальных, максимальных поверхностей и поверхностей предписанной средней кривизны  \cite{8}, \cite{10} -- \cite{12}, \cite{14} -- \cite{16}, \cite{19}, \cite{20}, \cite{23}, \cite{58}, \cite{59}, \cite{63}. В основе применения этих методов лежит простой факт, что координатные функции погруженных поверхностей предписанной средней кривизны удовлетворяют эллиптическим дифференциальным уравнениям в метрике поверхности. В ряде работ были получены оценки протяженности минимальных трубок не только для случая гиперповерхностей, но и для случая поверхностей произвольной коразмерности. Одним из замечательных следствий такой оценки является тот факт, что  верхняя оценка достигается только на поверхностях вращения. В случае поверхностей коразмерности больше 1 также полностью описан класс таких поверхностей. Эти результаты также были распространены на случай максимальных поверхностей в пространстве-времени Минковского. Отметим, что в этом пространстве решения уравнения максимальных поверхностей могут иметь изолированные особые точки. Исследованию вопросов существования и единственности решений с заданными особенностями были посвящены работы \cite{dop3}, \cite{dop4}.
	
	Отметим еще одно направление исследований в рамках геометрического анализа, выполненное В.М. Миклюковым и его учениками в Волгоградском государственном университете. Это направление связано с изучением признаков устойчивости и неустойчивости минимальных поверхностей, поверхностей предписанной средней кривизны и максимальных поверхностей в пространствах Лоренца \cite{7}, \cite{13}, \cite{18}, \cite{21}, \cite{22}, \cite{31}. Проблема устойчивости такого рода поверхностей вытекает из их вариационного происхождения. Важным моментом здесь является тот факт, что в природе могут существовать только устойчивые решения.  В ряде работ были получены признаки и свойства устойчивых не только классов поверхностей указанных выше, но и классов поверхностей, являющихся экстремалями более общих функционалов.
	
	С начала 2000-х годов в Волгоградском государственном университете начинаются исследования в области вычислительной геометрии и, в частности, объектом исследования становятся различные классы триангуляций \cite{6}, \cite{9}, \cite{17}, \cite{24}. Метод триангуляций находит широкое применение в различных областях: численные методы, 3D  моделирование и программирование,  машинное обучение и многое другое. Одной из задач в методе триангуляций является задача построения триангуляции с наперед заданными геометрическими характеристиками, такими как минимальный угол в треугольнике, максимальный радиус описанной окружности и т. п. Хорошо известен факт, что наилучшие значения указанных величин достигаются на классе триангуляций Делоне. В работах школы геометрического анализа в ВолГУ было дано строгое обоснование использования триангуляции Делоне в задачах аппроксимации дифференцируемых функций, заданных на нерегулярном множестве точек. А так же были построены соответствующие примеры, объясняющие, почему эти результаты не могут быть распространены на многомерный случай. Кроме этого, в работах ученых ИМИТ ВолГУ был описан класс функционалов и им соответствующий  класс триангуляций, на которых функционалы достигают наименьшего значения.  Одним из таких функционалов является сумма площадей охватывающих выпуклых множеств треугольников триангуляций. Доказан следующий результат. Минимальная сумма таких площадей достигается на триангуляции, удовлетворяющей аналогу свойству Делоне: каждое охватывающее множество треугольника триангуляции не содержит точек заданного конечного множества \cite{24}. Это свойство обобщает свойство классической триангуляции Делоне на более широкий класс триангуляций, включающий класс регулярных триангуляций (триангуляций, которые являются проекциями выпуклых многогранников). В настоящее время группа математиков ВолГУ работает над проблемами, связанными с исследованиями структуры множества данного класса триангуляций и решений соответствующих вариационных задач.
  
Отметим еще одно направление геометрического анализа, в котором работает ряд сотрудников ИМИТ ВолГУ.  На протяжении последних двух десятилетий ими наработана достаточно мощная и специфическая емкостная техника проведения исследований в теории дифференциальных уравнений на некомпактных римановых многообразиях, получен  ряд важных и интересных утверждений. Результаты исследований неоднократно докладывались на международных и всероссийских конференциях. Отметим, что разработанные методы позволили получить решение целого ряда проблем качественной теории однородных линейных и квазилинейных эллиптических уравнений и неравенств на некомпактных римановых многообразиях. Так, активное применение  метода Фурье разложения решений уравнений Лапласа-Бельтрами и стационарного уравнения Шредингера  по собственным функциям соответствующих краевых задач, способствовало нахождению емкостных характеристик, которые позволили впервые сформулировать и доказать критерии выполнения различных теорем типа Лиувилля и разрешимости краевых задач на модельных и квазимодельных многообразиях.  
	Проблеме разрешимости задачи Дирихле о восстановлении решений уравнений на некомпактных римановых многообразиях по граничным данным на <<бесконечности>> посвящены работы ряда математиков.  Для некоторых римановых многообразий (например, модельных  и  квазимодельных) существует естественная геометрическая компактификация, позволяющая поставить задачу Дирихле в ее классическом варианте. На таких многообразиях Лосевым А.Г. и Мазепой Е.А. были введены характеристики емкостного типа, которые позволили  также впервые найти критерии восстановления на них решений различных эллиптических уравнений по непрерывным граничным данным на <<бесконечности>>.
	Следует отметить, что  на произвольном некомпактном многообразии даже сама постановка  задачи Дирихле о восстановлении решений уравнений по граничным данным на <<бесконечности>>  достаточно проблематична. В последние годы в работах Мазепы Е.А. наметился новый подход к постановке краевых задач на произвольных некомпактных римановых многообразиях, основанный на введении понятия классов эквивалентных на многообразии непрерывных ограниченных функций. Удалось получить ряд интересных результатов о взаимосвязи разрешимости краевых и внешне краевых задач, об устойчивости решений при вариациях некоторых коэффициентов для линейных эллиптических уравнений. В дальнейшем, Мазепой Е.А.  данная методика была успешно применена для изучения решений различных краевых задач для некоторых полулинейных  и квазилинейных эллиптических уравнений \cite{27}, \cite{28}. Кроме того, были изучены вопросы об устойчивости теорем типа Лиувилля и об устойчивости разрешимости различных краевых задач (в терминах эквивалентных функций) при вариациях нелинейной части рассматриваемых уравнений.
	Метод эквивалентных функций оказался тесно связан с аппроксимативным подходом построения обобщенных решений краевых задач для эллиптических уравнений, описанный в ранних работах Келдыша М.В. и Ландиса Е.М. Данный подход, основанный на методе разметания (Пуанкаре, 19 в.), позволяет рассматривать решения задачи Дирихле без каких-либо ограничений на области, в которых она решается, оставаясь, при этом, в классических предположениях относительно непрерывности граничных данных. В работах Келдыша М.В. (1941) и Ландиса Е.М. (1971) данный подход был применен для изучения проблемы разрешимости задачи Дирихле и устойчивости ее решения в ограниченных областях, граница которых может иметь сложное геометрическое строение.  Аналогичный подход позволил Лосеву А.Г. и Филатову В.В. \cite{25} установить  взаимосвязь между существованием нетривиальных ограниченных и положительных решений стационарного уравнения Шредингера с конечным интегралом энергии. Мазепой Е.А. данный подход был распространен для построения обобщенного (по Келдышу) решения задачи Дирихле и других краевых задач с граничными данными на <<бесконечности>> для произвольных некомпактных римановых многообразий и доказана теорема единственности для рассматриваемых задач \cite{29}. В последнее десятилетие нашими сотрудниками  метод эквивалентных функций стал активно применяться для изучения поведения решений различных краевых задач для однородных линейных уравнений в неограниченных областях многообразий с некомпактным краем.  Светловым А.В. в соавторстве был получен ряд результатов об однозначной разрешимости рассматриваемых краевых задач \cite{62}.
	Другое направление исследований, в котором были получены новые результаты  в  последнее десятилетие, связано с  изучением  решений некоторых нелинейных эллиптических неравенств.  Лосевым А.Г. и Мазепой Е.А. были  получены условия отсутствия нетривиальных решений, а также условия существования и мощность множества положительных решений эллиптических дифференциальных  неравенств вида   на модельных римановых многообразиях и римановых произведениях \cite{30}, \cite{26}.
	Последние результаты обобщают аналогичные утверждения, полученные ранее в работах Naito. Y. и Usami H. (1997) для евклидова пространства. Аналогичные результаты на модельных и квазимодельных римановых многообразиях были получены Лосевым А.Г. и Вихаревым С.С. для стационарного уравнения Гинзбурга-Ландау и дифференциального неравенства специального вида \cite{2}. 
Отметим, что изучение различных свойств триангуляций и треугольных сеток областей, привело еще к одному типу задач, связанных с оценками вычисления интегральных функционалов и сходимостью кусочно-полиномиальных решений нелинейных уравнений вариационного типа. Некоторые задачи, возникающие при проектировании архитектурных сооружений, сводятся к построению поверхностей минимальной площади. Это достаточно подробно отражено в  книге  Михайленко В.Е.,  Ковалев С.Н.  Конструирование форм современных архитектурных сооружений,  Киев:Будiвельник, 1978. - 138 с., а так же в работе Абдюшев А.А., Мифтахутдинов И.Х., Осипов П.П.   Проектирование непологих оболочек минимальной поверхности. Известия КазГАСУ, Строительные конструкции, здания и сооружения, 2009, № 2 (12), где  изучается проблема разработки тентовых тканевых конструкций. Подробный анализ приведенных там результатов   приводит к задаче разработки эффективных методов для приближенного решения уравнения минимальной поверхности и математическому обоснованию найденных методов в плане устойчивости и сходимости приближенных решений. Основная трудность при исследовании данных вопросов заключается в том, что уравнение минимальной поверхности является нелинейным и поэтому традиционные методы, используемые для линейных уравнений,  не пригодны. Похожая ситуация возникает и  при исследовании других нелинейных уравнений, например, уравнения равновесной капиллярной поверхности.   
Первым шагом решения подобных задач стали результаты, связанные с изучением  кусочно-линейных решений уравнения минимальной поверхности над заданной триангуляцией многоугольной области \cite{3}. Именно, было показано, что при определенных условиях градиенты таких функций остаются по модулю ограниченными при стремлении к нулю максимального диаметра треугольников триангуляции. Подчеркивается, что это свойство выполняется, если кусочно-линейные функции приближают значение площади графика гладкой функции с необходимой точностью. Поэтому, в качестве дополнительного результата, было показано, что кусочно-линейные функции дают второй порядок точности вычисления площади графика достаточно гладкой функции. Следствием полученных свойств стала равномерная сходимость кусочно-линейных решений к точному решению уравнения минимальной поверхности. Аналогичный подход был применен и для уравнения равновесной капиллярной поверхности \cite{4}. Как оказалось, что соответствующий интеграл энергии также приближается с погрешностью $O(h^2)$, где $h$ --   мелкость разбиения.  И, как следствие этого свойства,  в работе была установлена равномерная сходимость кусочно-линейных решений к точному решению уравнения равновесной капиллярной поверхности с краевым условием в виде заданного контактного угла. Еще одним подходом построения приближенных решении, является вариационный метод в классе полиномиальных функций. В работе \cite{5} было сформулировано понятие полиномиального приближенного решения задачи Дирихле  уравнения минимальной поверхности. Основным результатом стало доказательство равномерной сходимости таких решений при определенных условиях на геометрическое строение области.
Следующим этапом исследования проблем аппроксимации функционалов стало изучение вопроса степени приближения интегралов, зависящих от вторых производных (примером может быть функционал полной свободной энергии деформированной пластинки) \cite{6}.  В этом направлении нами доказано следующее свойство. Подобный  функционал может быть вычислен с погрешностью порядка $O(h^{4m+1})$    при мелкости треугольной сетки $h\to 0$, если в качестве приближающих функций  рассматривать специальным образом выбранные кусочно-полиномиальные функции степени   $4m+1$ для $m\geq 1$. В двумерном случае удается показать, что кусочно-квадратичная аппроксимация дает второй порядок точности вычисления функционала  для специального вида триангуляции.  
Отметим, что в настоящее время часть вышеприведенных исследований ведутся на  базе совместной работы в рамках сотрудничества с Математическим Научным Центром при Институте математики имени С.Л. Соболева Сибирского отделения Российской Академии Наук.




\section{Заключение}



\begin{thebibliography}{99}

\bibitem{1}
\baut{Веденяпин}{А. Д.}
\baut{Миклюков}{В. М.}
\btit{Внешние размеры трубчатых минимальных гиперповерхностей }[ External dimensions of tubular minimal hypersurfaces]
\bj{Мат. сб.}[Math. sbornik.]
\byr{1986}
\bvol{131}
\bnum{6}
\bpp{240 -- 250}
\mkpaperr

\bibitem{2}
\baut{Вихарев}{С. С.}
\btit{О некоторых лиувиллевых теоремах для стационарного уравнения Гинзбурга - Ландау на квазимодельных римановых многообразиях}[On some Liouville theorems for the stationary Ginzburg-Landau equation on quasimodel Riemannian manifolds]
\bj{Известия Саратовского университета. Новая серия.}[Izvestiya of Saratov University. New Series. Series: Mathematics. Mechanics. Informatics]
\byr{2015}
\bvol{15}
\bnum{2}
\bpp{127-135}
\mkpaperr



\bibitem{dop3}
\baut{Клячин}{А. А.}
\baut{Миклюков}{В. М.}
\btit{Существование решений с особенностями уравнения 
максимальных поверхностей в пространстве Минковского}[Existence of solutions with singularities 
for the maximal surface equation in Minkowski space]
\bj{Математический сборник}[Mathematical Collection.]
\byr{1993}
\bvol{184}
\bnum{9}
\bpp{103-124}
\mkpaperr



\bibitem{dop4}
\baut{Клячин}{А. А.}
\btit{Описание множества целых решений с особенностями уравнения максимальных поверхностей}[Description of the set of singular entire solutions of the maximal surface equation]
\bj{Математический сборник}[Mathematical Collection.]
\byr{2003}
\bvol{194}
\bnum{7}
\bpp{83-104}
\mkpaperr


\bibitem{3}
\baut{Гацунаев}{М. А.}
\baut{Клячин}{А. А.}
\btit{О равномерной сходимости кусочно-линейных решений уравнения минимальной поверхности}[On the uniform convergence of piecewise linear solutions of the minimal surface equation]
\bj{Уфимский математический журнал.}[Ufa Mathematical Journal]
\byr{2014}
\bvol{6}
\bnum{3}
\bpp{3-16}
\mkpaperr

\bibitem{4}
\baut{Клячин}{А. А.}
\btit{О равномерной сходимости кусочно-линейных решений уравнения равновесной капиллярной поверхности}[On the uniform convergence of piecewise linear solutions of the equilibrium capillary surface equation ]
\bj{Сибирский журнал индустриальной математики}[Journal of Applied and Industrial Mathematics]
\byr{2015}
\bvol{18}
\bnum{2}
\bpp{52-62}
\mkpaperr

\bibitem{5}
\baut{Клячин}{А. А.}
\baut{Трухляева}{И. В.}
\btit{О сходимости полиномиальных приближенных решений уравнения минимальной поверхности}[On the convergence of polynomial approximate solutions of the minimal surface equation]
\bj{Уфимский математический журнал}[Ufa Mathematical Journal]
\byr{2016}
\bvol{8}
\bnum{1}
\bpp{72-83}
\mkpaperr

\bibitem{6}
\baut{Клячин}{А. А.}
\btit{Оценка погрешности вычисления функционала, содержащего производные второго порядка, на треугольной сетке }[Error estimation for calculating a functional containing second-order derivatives on a triangular grid]
\bj{Сиб. электрон. матем. известия}[Siberian electronic mathematical reports]
\byr{2019}
\bvol{16}
\bnum{ }
\bpp{1856 -- 1867}
\mkpaperr

\bibitem{7}
\baut{Клячин}{В. А.}
\btit{О некоторых свойствах устойчивых и неустойчивых поверхностей предписанной средней кривизны}[On some properties of stable and unstable surfaces of prescribed average curvature]
\bj{Изв. РАН. Сер. матем.}[IZVESTIYA MATHEMATICS]
\byr{2006}
\bvol{70}
\bnum{ }
\bpp{77--90}
\mkpaperr

\bibitem{8}
\baut{Клячин}{В. А.}
\btit{Об асимптотических свойствах максимальных трубок и лент в окрестности изолированной особенности в пространстве Минковского}[On the asymptotic properties of maximal tubes and tapes in a neighborhood of an isolated singularity in Minkowski space]
\bj{Сиб. мат. ж.}[Siberian Mathematical Journal]
\byr{2002}
\bvol{43}
\bnum{1}
\bpp{76 – 89}
\mkpaperr

\bibitem{9}
\baut{Клячин}{В. А.}
\btit{Аппроксимация градиента функции на основе специального класса триангуляций}[Function gradient approximation based on a special class of triangulations]
\bj{Изв. РАН. Сер. матем.}[IZVESTIYA MATHEMATICS]
\byr{2018}
\bvol{82}
\bnum{6}
\bpp{65--77}
\mkpaperr

\bibitem{10}
\baut{Клячин}{В. А.}
\btit{Максимальные трубчатые поверхности произвольной коразмерности в пространстве Минковского}[Maximal tubular surfaces of arbitrary codimension in Minkowski space]
\bj{Изв. РАН. Сер. матем.}[IZVESTIYA MATHEMATICS]
\byr{1993}
\bvol{57}
\bnum{4}
\bpp{118 -- 131}
\mkpaperr

\bibitem{11}
\baut{Клячин}{В. А.}
\btit{Новые примеры трубчатых минимальных поверхностей произвольной коразмерности}[New examples of tubular minimal surfaces of arbitrary codimension]
\bj{Матем. заметки.}[Mathematical Notes]
\byr{1997}
\bvol{62}
\bnum{1}
\bpp{154 -- 156}
\mkpaperr

\bibitem{12}
\baut{Клячин}{В. А.}
\btit{Об асимптотических свойствах максимальных трубчатых поверхностей в окрестности изолированной особенности в пространстве Минковского }[On the asymptotic properties of maximal tubular surfaces in a neighborhood of an isolated singularity in Minkowski space]
\bj{Вестник ВолГУ}[Vestnik VoGU]
\byr{2000}
\bvol{1}
\bnum{5}
\bpp{34 -- 42}
\mkpaperr

\bibitem{13}
\baut{Клячин}{В. А.}
\btit{Об устойчивости максимальных поверхностей с изолированной особенностью}[On the stability of maximal surfaces with an isolated singularity]
\bj{Вестник ВолГУ}[Vestnik VoGU]
\byr{1999}
\bvol{1}
\bnum{4}
\bpp{10 -- 12}
\mkpaperr

\bibitem{14}
\baut{Клячин}{В. А.}
\btit{Оценка протяженности трубчатых минимальных поверхностей произвольной коразмерности}[Estimation of the length of tubular minimal surfaces of arbitrary codimension]
\bj{Сиб. мат. ж.}[Siberian Mathematical Journal]
\byr{1992}
\bvol{33}
\bnum{5}
\bpp{201--206}
\mkpaperr

\bibitem{15}
\baut{Клячин}{В. А.}
\btit{Поверхности нулевой средней кривизны со знакопеременной метрикой}[Zero mean curvature surfaces with alternating metric]
\bj{Труды кафедры математического анализа и теории функций Волгоградского государственного университета}[Proceedings of the Department of Mathematical Analysis and Function Theory of Volgograd State University]
\byr{2002}
\bvol{}
\bnum{ }
\bpp{56 -- 77}
\mkpaperr

\bibitem{16}
\baut{Клячин}{В. А.}
\btit{Строение поверхностей нулевой средней кривизны  в окрестности изолированной особой точки}[The structure of surfaces of zero mean curvature in the vicinity of an isolated singular point]
\bj{Докл. РАН}[Dokl. RAN]
\byr{2002}
\bvol{383}
\bnum{6}
\bpp{727 -- 730}
\mkpaperr

\bibitem{17}
\baut{Клячин}{В. А.}
\baut{Григорьева}{Е. Г.}
\btit{Описание функционалов, минимизируемых ?-триангуляциями}[Description of functionals minimized by ?-triangulations]
\bj{Итоги науки и техн. Сер. Соврем. мат. и ее прил. Темат. обз.}[Journal of Mathematical Sciences ]
\byr{2017}
\bvol{139}
\bnum{ }
\bpp{9 -- 14}
\mkpaperr

\bibitem{18}
\baut{Клячин}{В. А.}
\baut{Медведева}{Н. М.}
\btit{Об устойчивости экстремальных поверхностей некоторых функционалов типа площади}[On the stability of extremal surfaces of some functionals such as area]
\bj{Сиб. электрон. матем. изв.,}[Siberian electronic mathematical reports]
\byr{2007}
\bvol{4}
\bnum{}
\bpp{113 -- 132}
\mkpaperr

\bibitem{19}
\baut{Клячин}{В. А.}
\baut{Миклюков}{В. М.}
\btit{Геометрическое строение трубок и лент нулевой средней кривизны в пространстве Минковского}[The geometric structure of tubes and tapes of zero average curvature in Minkowski space]
\bj{В сб. <<Научные школы ВолГУ. Геометрический анализ>>, Волгоград. Изд-во ВолГУ}[Collection of scientific schools of VolSU]
\byr{1999}
\bvol{}
\bnum{}
\bpp{204 -- 244}
\mkpaperr

\bibitem{20}
\baut{Клячин}{В. А.}
\baut{Миклюков}{В. М.}
\btit{Максимальные гиперповерхности трубчатого  типа в пространстве Минковского}[Maximum tube-type hypersurfaces in Minkowski space]
\bj{Изв. АН СССР. Сер.}[IZVESTIYA MATHEMATICS]
\byr{1991}
\bvol{55}
\bnum{1}
\bpp{206 -- 217}
\mkpaperr

\bibitem{21}
\baut{Клячин}{В. А.}
\baut{Миклюков}{В. М.}
\btit{Об одном емкостном признаке неустойчивости минимальных гиперповерхностей}[On a capacitive criterion for instability of minimal hypersurfaces]
\bj{Доклады РАН}[Dokl. RAN]
\byr{1993}
\bvol{330}
\bnum{4}
\bpp{424 -- 426}
\mkpaperr

\bibitem{22}
\baut{Клячин}{В. А.}
\baut{Миклюков}{В. М.}
\btit{Признаки неустойчивости поверхностей нулевой средней кривизны в искривленных лоренцевых произведениях}[Signs of instability of surfaces of zero mean curvature in curved Lorentzian products]
\bj{Матем. сб.}[Math. Sb.]
\byr{1996}
\bvol{187}
\bnum{11}
\bpp{67 -- 88}
\mkpaperr

\bibitem{23}
\baut{Клячин}{В. А.}
\baut{Миклюков}{В. М.}
\btit{Условия конечности времени существования максимальных трубок и лент в искривленных лоренцевых произведениях}[Conditions for the finiteness of the existence time of maximal tubes and tapes in curved Lorentzian products]
\bj{Изв. РАН. Сер. мат.}[IZVESTIYA MATHEMATICS]
\byr{1994}
\bvol{58}
\bnum{3}
\bpp{196 -- 210}
\mkpaperr

\bibitem{24}
\baut{Клячин}{В. А.}
\btit{Экстремальные свойства триангуляции, основанной на условии пустого выпуклого множества}[Extreme properties of a triangulation based on the condition of an empty convex set]
\bj{Сиб. электрон. матем. изв.,}[Siberian electronic mathematical reports]
\byr{2015}
\bvol{12}
\bnum{ }
\bpp{991 -- 997}
\mkpaperr

\bibitem{25}
\baut{Лосев}{А. Г.}
\baut{Филатов}{В. В.}
\btit{Теоремы типа Лиувилля для решений стационарного уравнения Шредингера с конечным интегралом Дирихле}[Liouville-type theorems for solutions of the stationary Schr?dinger equation with finite Dirichlet integral]
\bj{Вестник ВолГУ. Серия}[Vestnik VolGU]
\byr{2016}
\bvol{1}
\bnum{5}
\bpp{13-23}
\mkpaperr

\bibitem{26}
\baut{Лосев}{А. Г.}
\baut{Мазепа}{Е. А.}
\btit{Об асимптотическом поведении положительных решений некоторых квазилинейных неравенств на модельных римановых многообразиях}[On the asymptotic behavior of positive solutions of some quasilinear inequalities on model Riemannian manifolds]
\bj{Уфимский математический журнал}[Ufa Mathematical Journal]
\byr{2013}
\bvol{5}
\bnum{1}
\bpp{83-89}
\mkpaperr

\bibitem{27}
\baut{Мазепа}{Е. А.}
\btit{К вопросу о разрешимости краевых задач для полулинейных эллиптических уравнений на некомпактных римановых многообразиях}[On the solvability of boundary value problems for semilinear elliptic equations on noncompact Riemannian manifolds]
\bj{Вестник Волгоградского государственного университета}[Vestnik VolGU]
\byr{2014}
\bvol{1}
\bnum{4}
\bpp{36-46}
\mkpaperr

\bibitem{28}
\baut{Мазепа}{Е. А.}
\btit{О разрешимости краевых задач для квазилинейных эллиптических уравнений на некомпактных римановых многообразиях}[On the solvability of boundary value problems for quasilinear elliptic equations on noncompact Riemannian manifolds]
\bj{Сибирские электронные математические известия}[Siberian electronic mathematical reports]
\byr{2016}
\bvol{13}
\bnum{ }
\bpp{1026 -- 1034}
\mkpaperr

\bibitem{29}
\baut{Мазепа}{Е. А.}
\btit{Обобщенные решения краевых задач для квазилинейных эллиптических уравнений на некомпактных римановых многообразиях}[Generalized solutions of boundary value problems for quasilinear elliptic equations on noncompact Riemannian manifolds]
\bj{Известия вузов. Математика}[Izvestiya vuzov. Mathematics]
\byr{2018}
\bvol{}
\bnum{1}
\bpp{57-66}
\mkpaperr

\bibitem{30}
\baut{Мазепа}{Е. А.}
\btit{Положительные решения квазилинейных эллиптических неравенств на модельных римановых многообразиях}[Positive solutions of quasilinear elliptic inequalities on model Riemannian manifolds]
\bj{Известия вузов. Математика}[Izvestiya vuzov. Mathematics]
\byr{2015}
\bvol{}
\bnum{9}
\bpp{22-30}
\mkpaperr

\bibitem{31}
\baut{Медведева}{Н. М.}
\btit{Исследование устойчивости экстремальных поверхностей вращения}[Investigation of the stability of extreme surfaces of revolution]
\bj{Известия Саратовского университета}[Izvestiya of Saratov University. New Series. Series: Mathematics. Mechanics. Informatics]
\byr{2007}
\bvol{7}
\bnum{2}
\bpp{25 -- 32}
\mkpaperr

\bibitem{32}
\baut{Миклюков}{В. М.}
\btit{Асимптотические тракты субгармонических функций на многообразии и внешнее строение минимальных поверхностей}[Asymptotic paths of subharmonic functions on a manifold and the external structure of minimal surfaces]
\bj{Тезисы докл. Всесоюзн. конф. по геометрии и анализу.}[Abstracts of Soviet Conference by geometry and analysis ]
\byr{1989}
\bvol{}
\bnum{}
\bpp{54}
\mkpaperr



%33.	Миклюков В.М., Введение в негладкий анализ, 2-е изд., Волгоград: Изд-во ВолГУ,  2008.

\bibitem{33}
\baut{Миклюков}{В. М.}
\btit{Введение в негладкий анализ}[Introduction to nonsmooth analysis]
\bcity{Волгоград}
\bpub{Изд-во ВолГУ}
\byr{2008}
\bpp{750}
\mkbookr 

%34.	Миклюков В.М., Геометрический анализ поверхностей нулевой средней кривизны. - Научные школы Волгогр. гос. ун-та. Геометрический анализ и его приложенния. - Волгоград: Изд-во Волгогр. гос. Ун-та, 1999. - С. 5-21. 

\bibitem{34}
\baut{Миклюков}{ В.~М.}
\btit{Геометрический анализ поверхностей нулевой средней кривизны.}[Geometric analysis of surfaces with zero mean curvature.]
\bj{Научные школы Волгогр. гос. ун-та. Геометрический анализ и его приложенния. - Волгоград: Изд-во Волгогр. гос. Ун-та}[Scientific schools of Volgogr. state univ. Geometric analysis and its applications.]
\byr{1999}
\bnum{1}
\bpp{5--21}
\mkpaperr

%35.	Миклюков В.М., Геометрический анализ; дифференциальные формы, почти решения, почти квазиконформные отображения,  Волгоград: Изд-во ВолГУ, 2007.

\bibitem{35}
\baut{Миклюков}{В. М.}
\btit{Геометрический анализ. Дифференциальные формы, почти-решения, почти квазиконформные отображения}[Geometric analysis. Differential forms, almost-solutions, almost quasiconformal mappings]
\bcity{Волгоград}
\bpub{Изд-во ВолГУ}
\byr{2007}
\bpp{530}
\mkbookr 

%36.	Миклюков В.М., Граничные свойства решений уравнений типа минимальных поверхностей. - Мат. сб. - Т. 192. - 2001. - N. 10. - С. 1491-1513.

\bibitem{36}
\baut{Миклюков}{ В.~М.}
\btit{Граничные свойства решений уравнений типа минимальных поверхностей.}[Boundary properties of solutions of equations of the type of minimal surfaces.]
\bj{Математический сборник.}[Mathematical Collection.]
\byr{2001}
\bnum{192(10)}
\bpp{1491--1513}
\mkpaperr

%37.	Миклюков В.М., Емкость и обобщенный принцип максимума для решений квазилинейных уравнений эллиптического типа. -  ДАН СССР. - Т. 250. - 1980. - N. 6. - С. 1318-1320.

\bibitem{37}
\baut{Миклюков}{ В.~М.}
\btit{Емкость и обобщенный принцип максимума для решений квазилинейных уравнений эллиптического типа.}[Capacity and generalized maximum principle for solutions of quasilinear equations of elliptic type..]
\bj{ДАН СССР.}[DAN USSR.]
\byr{1980}
\bnum{250(6)}
\bpp{1318--1320}
\mkpaperr

%38.	Миклюков В.М., Изотермические координаты на поверхностях с особенностями. - Мат. сб. - Т. 195. - 2004. - N. 1. - С. 69-88.

\bibitem{38}
\baut{Миклюков}{ В.~М.}
\btit{Изотермические координаты на поверхностях с особенностями.}[Isothermal coordinates on surfaces with features.]
\bj{Математический сборник.}[Mathematical Collection.]
\byr{2004}
\bnum{195(1)}
\bpp{69--88}
\mkpaperr

%39.	Миклюков В.М., Конформное отображение нерегулярной поверхности и его применения, Волгоград: Изд-во ВолГУ, 2005; имеется перевод: V.M. Miklyukov, Conformal Maps of Nonsmooth Surfaces and Their Applications. Exlibris Corporation, Philadelphia, 2008.

\bibitem{39}
\baut{Миклюков}{В. М.}
\btit{Конформное отображение нерегулярной поверхности и его применения.}[Conformal mapping of an irregular surface and its application]
\bcity{Волгоград}
\bpub{Изд-во ВолГУ}
\byr{2005}
\bpp{}
\mkbookr 


%40.	Миклюков В.М., Локальное время, сверхмедленные процессы и зоны стагнации,  Записки семинара ''Сверхмедленные процессы'', Волгоград: Изд-во ВолГУ, 2006, 131-137.

\bibitem{40}
\baut{Миклюков}{ В.~М.}
\btit{Локальное время, сверхмедленные процессы и зоны стагнации.}[Local time, super slow processes and stagnation zones.]
\bj{Записки семинара <<Сверхмедленные процессы>>, Волгоград: Изд-во ВолГУ.}[Workshop Notes << Super Slow Processes >>, Volgograd: VolSU Publishing House.]
\byr{2006}
\bnum{}
\bpp{131--137}
\mkpaperr

%41.	Миклюков В.М., Максимальные трубки и ленты в пространстве Минковского. -  Мат. сб. - Т. 183. - 1992. - N. 12. - С. 45-76.

\bibitem{41}
\baut{Миклюков}{ В.~М.}
\btit{Максимальные трубки и ленты в пространстве Минковского.}[Maximum tubes and tapes in Minkowski space.]
\bj{Математический сборник.}[Mathematical Collection.]
\byr{1992}
\bnum{183(12)}
\bpp{45--76}
\mkpaperr

%42.	Миклюков В.М., Минимальные ленты типа геликоида. -   IX-я Всесоюзн. геом. конферен. - Кишинев, 1988, - С. 213.

\bibitem{42}
\baut{Миклюков}{ В.~М.}
\btit{Минимальные ленты типа геликоида.}[Minimal helicoid type tapes.]
\bj{IX-я Всесоюзн. геом. конферен. - Кишинев.}[IXth All-Union. geom. conference - Chisinau.]
\byr{1988}
\bnum{}
\bpp{213}
\mkpaperr


%43.	Миклюков В.М., Множества особенностей решений уравнения максимальных поверхностей в пространстве Минковского. - Сибир. мат. Журн. - Т. 131. - 1992. - N. 6. - С. 131-140.

\bibitem{43}
\baut{Миклюков}{ В.~М.}
\btit{Множества особенностей решений уравнения максимальных поверхностей в пространстве Минковского.}[The set of singularities of solutions of the equation of maximal surfaces in Minkowski space.]
\bj{Сибирский математический журнал}[Siberian Mathematical Journal]
\byr{1992}
\bnum{131(6)}
\bpp{131--140}
\mkpaperr

%44.	Миклюков В.М., Некоторые особенности поведения решений уравнений типа минимальной поверхности в неограниченных областях. -  Мат. сб. - Т. 116 (158). - 1981. - N. 1. - С. 72-86.


\bibitem{44}
\baut{Миклюков}{ В.~М.}
\btit{Некоторые особенности поведения решений уравнений типа минимальной поверхности в неограниченных областях.}[Some features of the behavior of solutions of equations such as a minimal surface in unbounded domains.]
\bj{Математический сборник.}[Mathematical Collection.]
\byr{1981}
\bnum{116(1)}
\bpp{72--86}
\mkpaperr

%45.	Миклюков В.М., Некоторые признаки  параболичности  и гиперболичности  граничных множеств поверхностей. - Изв. РАН.  Сер. мат. - Т. 60. - 1996. - N. 4. - С. 111-158.

\bibitem{45}
\baut{Миклюков}{ В.~М.}
\btit{Некоторые признаки  параболичности  и гиперболичности  граничных множеств поверхностей.}[Some signs of parabolicity and hyperbolicity of boundary sets of surfaces.]
\bj{Известия РАН.  Серия математическая.}[Proceedings of the RAS. The series is mathematical.]
\byr{1996}
\bnum{60(4)}
\bpp{111--158}
\mkpaperr


%46.	Миклюков В.М., О конформном типе концов максимальных пространственно подобных поверхностей с особенностями. -   Актуальные вопросы комплексного анализа: Тезисы докл. школы-семинара. - Ташкент, 1989. - С. 79.

\bibitem{46}
\baut{Миклюков}{ В.~М.}
\btit{О конформном типе концов максимальных пространственно подобных поверхностей с особенностями.}[On the conformal type of ends of maximal spatially similar surfaces with singularities.]
\bj{Актуальные вопросы комплексного анализа: Тезисы докл. школы-семинара.}[Actual issues of complex analysis: Abstracts of the school-seminar.]
\byr{1989}
\bnum{}
\bpp{79}
\mkpaperr

%47.	Миклюков В.М., О конформном типе поверхностей, теореме Лиувилля и теореме Бернштейна. - ДАН СССР. - Т. 242. 1978. - N. 3. -  С. 537-540.

\bibitem{47}
\baut{Миклюков}{ В.~М.}
\btit{О конформном типе поверхностей, теореме Лиувилля и теореме Бернштейна.}[On the conformal type of surfaces, Liouville's theorem, and Bernstein's theorem.]
\bj{ДАН СССР.}[DAN USSR.]
\byr{1978}
\bnum{242(3)}
\bpp{537--540}
\mkpaperr

%48.	Миклюков В.М., О критических точках решений уравнений типа максимальных поверхностей в пространстве Минковского. -   Теория отображ. и приближ. функ. - Киев: Наукова Думка, 1989. - С. 112-125.

\bibitem{48}
\baut{Миклюков}{ В.~М.}
\btit{О критических точках решений уравнений типа максимальных поверхностей в пространстве Минковского.}[On critical points of solutions of equations of the type of maximal surfaces in Minkowski space.]
\bj{Теория отображ. и приближ. функ. - Киев: Наукова Думка.}[Theory of mappings and approximation of functions - Kiev: Naukova Dumka.]
\byr{1989}
\bnum{}
\bpp{112--125}
\mkpaperr


%49.	Миклюков В.М., О некоторых граничных задачах теории конформных отображений. - Сибир. мат. ж. - Т. 18. - 1977. - N. 5. - С. 1111-1124.

\bibitem{49}
\baut{Миклюков}{ В.~М.}
\btit{О некоторых граничных задачах теории конформных отображений.}[On some boundary value problems of the theory of conformal mappings.]
\bj{Сибирский математический журнал}[Siberian Mathematical Journal]
\byr{1977}
\bnum{18(5)}
\bpp{1111--1124}
\mkpaperr


%50.	Миклюков В.М., О некоторых свойствах трубчатых в целом минимальных поверхностей в Rn - ДАН СССР. - Т. 247. - 1979. - N. 3. - С. 549-552.

\bibitem{50}
\baut{Миклюков}{ В.~М.}
\btit{О некоторых свойствах трубчатых в целом минимальных поверхностей в $R^n$.}[On some properties of tubular generally minimal surfaces in $R^n$.]
\bj{ДАН СССР.}[DAN USSR.]
\byr{1979}
\bnum{247(3)}
\bpp{549--552}
\mkpaperr

%51.	Миклюков В.М., Об асимптотических свойствах субрешений квазилинейных уравнений эллиптического типа и отображений с ограниченным искажением. - Мат. сб. - Т. 111. - 1980. - N. 1. - С. 42-66.

\bibitem{51}
\baut{Миклюков}{ В.~М.}
\btit{Об асимптотических свойствах субрешений квазилинейных уравнений эллиптического типа и отображений с ограниченным искажением.}[On the asymptotic properties of sub-solutions of quasilinear equations of elliptic type and mappings with bounded distortion.]
\bj{Математический сборник.}[Mathematical Collection.]
\byr{1980}
\bnum{111(1)}
\bpp{42--66}
\mkpaperr

%52.	Миклюков В.М., Об одной оценке модуля семейства кривых на минимальной поверхности и ее применениях. - Успехи мат. наук. - Т. 34. - 1979. - N. 3. - С. 207-208

\bibitem{52}
\baut{Миклюков}{ В.~М.}
\btit{Об одной оценке модуля семейства кривых на минимальной поверхности и ее применениях.}[An estimate of the modulus of a family of curves on a minimal surface and its applications.]
\bj{Успехи мат. наук.}[Success math. sciences.]
\byr{1979}
\bnum{34(3)}
\bpp{207--208}
\mkpaperr

%53.	Миклюков В.М., Об одном новом подходе к теореме Бернштейна и близким вопросам уравнений типа минимальной поверхности. -   Мат. сб. - Т. 108 (150). - 1979. - N. 2. - С. 268-289.

\bibitem{51}
\baut{Миклюков}{ В.~М.}
\btit{Об одном новом подходе к теореме Бернштейна и близким вопросам уравнений типа минимальной поверхности.}[On a new approach to Bernstein's theorem and related questions of equations of the type of minimal surface.]
\bj{Математический сборник.}[Mathematical Collection.]
\byr{1979}
\bnum{108(2)}
\bpp{268--289}
\mkpaperr


%54.	Миклюков В.М., Суворов Г.Д., О существовании и единственности квазиконформных отображений с неограниченными характеристиками. - Исследов.  по теор. функц. компл. перемен. и ее примен. - Киев: Наукова Думка,  1972. - С. 45-53.

\bibitem{54}
\baut{Миклюков}{ В.~М.}
\baut{ Суворов}{Г.~Д.}
\btit{О существовании и единственности квазиконформных отображений с неограниченными характеристиками.}[On the existence and uniqueness of quasiconformal mappings with unbounded characteristics.]
\bj{Исследования  по теории функций  комплексного переменного и ее применения - Киев: Наукова Думка.}[Research on the theory of functions of a complex variable and its application - Kiev: Naukova Dumka.]
\byr{1972}
\bnum{}
\bpp{45--53}
\mkpaperr

%55.	Миклюков В.М., Ткачев В.Г., Некоторые свойства трубчатых минимальных поверхностей произвольной коразмерности. -  Мат. сб. - Т. 180. - 1989. - N. 9. - С. 1278-1295.

\bibitem{55}
\baut{Миклюков}{ В.~М.}
\baut{Ткачев}{ В.~Г.}
\btit{Некоторые свойства трубчатых минимальных поверхностей произвольной коразмерности.}[Some properties of tubular minimal surfaces of arbitrary codimension.]
\bj{Математический сборник.}[Mathematical Collection.]
\byr{1989}
\bnum{180(9)}
\bpp{1278--1295}
\mkpaperr

%56.	Миклюков В.М., Ткачев В.Г., О строении в целом внешне полных минимальных поверхностей в Rn. - Изв. вузов. Математика. - 1987. - N. 7. - С. 30-36.

\bibitem{55}
\baut{Миклюков}{ В.~М.}
\baut{Ткачев}{ В.~Г.}
\btit{О строении в целом внешне полных минимальных поверхностей в $R^n$.}[On the structure as a whole of externally complete minimal surfaces in $R^n$.]
\bj{Изв. вузов. Математика.}[Izv. universities. Mathematics.]
\byr{1987}
\bnum{9}
\bpp{30--36}
\mkpaperr

%57.	Миклюков В.М., Функции весовых классов Соболева, анизотропные метрики и вырождающиеся квазиконформные отображения, Волгоград: Изд-во ВолГУ, 2010.

\bibitem{57}
\baut{Миклюков}{В. М.}
\btit{Функции весовых классов Соболева, анизотропные метрики и вырождающиеся квазиконформные отображения}[Functions of Sobolev weight classes, anisotropic metrics, and degenerate quasiconformal mappings]
\bcity{Волгоград}
\bpub{Изд-во ВолГУ}
\byr{2010}
\bpp{}
\mkbookr 

%58.	Ткачев В.Г. Минимальные трубки с конечной полной кривизной // Сиб. матем. ж. 1998. Т.39, N1. С.181 – 190.

\bibitem{58}
\baut{Ткачев}{ В.~Г.}
\btit{Минимальные трубки с конечной полной кривизной.}[Minimum tubes with finite total curvature.]
\bj{Сибирский математический журнал}[Siberian Mathematical Journal]
\byr{1998}
\bnum{39(1)}
\bpp{181--190}
\mkpaperr


%59.	Ткачев В.Г.,  Некоторые оценки средней кривизны графиков над областями в Rn, Доклады АН СССР, 314:1 (1990), с. 140-143

\bibitem{59}
\baut{Ткачев}{ В.~Г.}
\btit{Некоторые оценки средней кривизны графиков над областями в $R^n$.}[Some estimates of the average curvature of graphs over regions in $R^n$.]
\bj{Доклады АН СССР}[Reports of the USSR Academy of Sciences]
\byr{1990}
\bnum{314(1)}
\bpp{140--143}
\mkpaperr

%60.	Klyachin V. A., Miklyukov V. M. Geometrical structure of tubes and bands of zero mean curvature in Minkowski space. Annales Academiae Scientiarum Fennicae, Mathematica. V. 28. 2003. P. 239–270. 

\bibitem{60}
\baut{Klyachin}{ V.~A.}
\baut{Miklyukov}{ V.~M.}
\btit{Geometrical structure of tubes and bands of zero mean curvature in Minkowski space.}[]
\bj{Annales Academiae Scientiarum Fennicae, Mathematica.}[]
\byr{2003}
\bnum{28}
\bpp{239--270}
\mkpaperr

%61.	Klyachin V.A., Approximation of the gradient of a function on the basis of a special class of triangulations, Izv. Math., 82:6 (2018),  p. 1136–1147 

\bibitem{61}
\baut{Klyachin}{ V.~A.}
\btit{Approximation of the gradient of a function on the basis of a special class of triangulations.}[]
\bj{Izv. Math.}[]
\byr{2018}
\bnum{82(6)}
\bpp{1136--1147}
\mkpaperr


%62.	Korolkov S., Korolkova E., Svetlov A. On solvability of boundary value problems for solutions of the stationary Schrodinger equation on unbounded domains of Riemannian manifolds// International Journal of Pure and Applied Mathematics. Vol. 97, No. 2, 2014, P. 231-240.

\bibitem{62}
\baut{Korolkov}{S.}
\baut{Korolkova}{E.}
\baut{Svetlov}{A.}
\btit{On solvability of boundary value problems for solutions of the stationary Schrodinger equation on unbounded domains of Riemannian manifolds.}[]
\bj{International Journal of Pure and Applied Mathematics.}[]
\byr{2014}
\bnum{97(2)}
\bpp{231--240}
\mkpaperr

%63.	Tkachev V.G. Minimal tube and coefficients of holomorphic functions // Bull. de la Soc. Sci. de Lodz, v. XX, p.19 – 26.

\bibitem{63}
\baut{Tkachev}{ V.~G.}
\btit{Minimal tube and coefficients of holomorphic functions.}[]
\bj{Bull. de la Soc. Sci. de Lodz}[]
\byr{2018}
\bnum{XX}
\bpp{19--26}
\mkpaperr


\end{thebibliography}


\begin{summary} 
This article discusses the main directions of research in geometric analysis, which were conducted and are being carried out by the scientific mathematical school of Volgograd State University.
The results of the founder of the scientific school, Doctor of Physics and Mathematics, Professor Vladimir Mikhailovich Miklyukov and his students are summarized. These results concern the solution of a number of problems in the field of quasiconformal flat mappings and mappings with bounded distortion of surfaces and Riemannian manifolds, the theory of minimal surfaces and surfaces of prescribed mean curvature, surfaces of zero mean curvature in Lorentz spaces, as well as problems associated with the study of the stability of such surfaces. In addition, the results of the study of various classes of triangulations -- an object that appears at the junction of research in the field of geometric analysis and computational mathematics -- are remarkable.
Besides, this review discusses papers that use the Fourier decomposition method.
solutions of the Laplace-Beltrami equations
and the stationary Schr\"{o}dinger equation with respect to the eigenfunctions of the corresponding boundary value problems. In particular, the results are given
finding capacitive characteristics that allowed for the first time to formulate
and prove the criteria for the fulfillment of various theorems of Liouville type and the solvability of boundary value problems on
model and quasimodel Riemannian manifolds. The role of the equivalent function method is also indicated.
in the study of such problems on manifolds of a fairly general form.

In addition to this, this article gives an overview of the results concerning estimates of calculation error
 integral functionals and convergence of piecewise polynomial solutions of nonlinear
 variational type equations: minimal surface equations, equilibrium equations
 capillary surface and equations of biharmonic functions.

\end{summary}

\end{document}
