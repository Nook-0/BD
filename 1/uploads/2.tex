 %\documentclass[a4paper,11pt,twoside]{article}
%\usepackage{vvmph2}
%\usepackage{rumathbr}


%\begin{document}

\udk{618.19+004.021}
\bbk{55.6}
\art[Алгоритмы классификации заболеваний]{Алгоритмы классификации заболеваний парных органов на основе нейросетей\plb и нечетких множеств}{Algorithms for the classification of diseases of paired organs on the basis of neural networks and fuzzy sets}
\support{Работа выполнена при финансовой поддержке гранта РФФИ и Администрации Волгоградской области
	(проект № 15-47-02475-р_поволжье_а)}

%\support{Здесь могла бы быть информация о поддержке работы грантом}


\fio{Зенович}{Андрей}{Васильевич} \pos{Старший преподаватель кафедры фундаментальной информатики\\ и оптимального управления}{Senior Lecturer,\\
	Department of Fundamental Computer Science and Optimal Control}
\emp{Волгоградский государственный университет}{Volgograd State
	University} \email{zenovich@rambler.ru, fiou@volsu.ru} \addr{просп.
	Университетский, 100, 400062 г. Волгоград, Российская
	Федерация}{\vspace*{-0.5mm}Prosp. Universitetsky, 100, 400062 Volgograd, Russian
	Federation}

\fio{Гребнев}{Виталий}{Игоревич} \pos{Студент института математики и информационных технологий}{Student,\\
	Institute of Mathematics and Information Technologies}
\emp{Волгоградский государственный университет}{Volgograd State
	University} \email{vitalii.grebnev@mail.ru} \addr{просп.
	Университетский, 100, 400062 г. Волгоград, Российская
	Федерация}{Prosp. Universitetsky, 100, 400062 Volgograd, Russian
	Federation}

\fio{Примаченко}{Филипп}{Германович} \pos{Студент института математики и информационных технологий}{Student,\\
	Institute of Mathematics and Information Technologies}
\emp{Волгоградский государственный университет}{Volgograd State
	University} \email{lshadow@mail.ru} \addr{просп.
	Университетский, 100, 400062 г. Волгоград, Российская
	Федерация}{Prosp. Universitetsky, 100, 400062 Volgograd, Russian
	Federation}



\maketitle

\begin{abstract} Предложены два алгоритма диагностики заболеваний парных органов методом микроволновой термометрии: на основе нейросетей и на базе аппарата нечетких множеств.

При построении нейросетевого алгоритма рассматривается новая модификация  архитектуры нейросетей, заключающаяся в автоматическом добавлении нейронов на выходной слой в процессе обучения нейросети.  Проводились вычислительные эксперименты по диагностике варикозных заболеваний ног и заболеваний молочных желез. Эти эксперименты показали, что данная модификация улучшает эффективность алгоритма на 10--12\,\%.

Алгоритм диагностики на основе нечетких множеств по признакам диагностики строит нечеткие множества, после чего диагноз ставится методом, аналогичным методу некомпенсаторного агрегирования. Также проводилось тестирование алгоритма на варикозных заболеваниях и на заболеваниях молочных желез.

\end{abstract}

\keywords{интеллектуальный анализ данных, микроволновая радиотермометрия, консультативные интеллектуальные системы, маммология, флебология}{data mining, microwave radiothermometry, intelligent advisory systems, mammalogy, phlebology}

\section*{Введение}



В последнее время активно разрабатываются методы 	 диагностики заболеваний, основанные на изучении тепловых полей организма человека. Одним из таких методов является метод микроволновой радиотермометрии (см., например, [3]), заключающийся в диагностике по набору внутренних и поверхностных температур тканей, измеренных  в микроволновом и инфракрасном диапазоне, и набору некоторых дополнительных характеристик.

Метод микроволновой радиотермометрии  достаточно широко применяется в маммологии (см.: \cite{vka1,los2,los3}), флебологии (см.: \cite{an1,los1,los4,los5}), хирургии и других областях медицины. Например, при диагностике онкологических заболеваний молочных желез по температурным полям выявляются тепловые аномалии, что способствует обнаружению опухолей на начальной стадии быстрого размножения раковых клеток (см. \cite{vka1}). Этим микроволновая радиотермометрия выгодно отличается от классических методов диагностики онкологических заболеваний.

Существующий диагностический комплекс РТМ-01-РЭС способен измерять температуру кожи (ИК) и внутреннюю температуру тканей (РТМ) на глубине до 5 см. Разработано программное обеспечение, которое позволяет визуализировать соответствующие температурные поля, что облегчает медику постановку правильного диагноза. При этом постановка диагноза на основе температурных карт доступна только специалисту высокой квалификации, в совершенстве владеющему методами микроволновой термометрии. Данное обстоятельство серьезно затрудняет использование метода микроволновой термометрии в первичной диагностике (например, во время ежегодных профилактических медицинских осмотров). Выход из данной ситуации видится в создании консультативной информационной системы с элементами искусственного интеллекта, которая поможет врачу поставить правильный диагноз. Для разработки такой системы необходимы эффективные  и интуитивно понятные медику алгоритмы машинной диагностики.

Диагностика пациента начинается с измерения достаточно большого количества поверхностных и глубинных температур. Медиками разработаны методики таких измерений (см.: \cite{vka1,los4,los5}) для различных видов заболеваний, в которых определены точки измерения температур, указана последовательность данных измерений и условия, при которых их можно проводить. При этом для парных органов желательно точки измерения выбирать симметричными, чтобы была возможность сравнить значения температур в этих точках. В качестве примера на рисунке 1 показаны точки измерения температур для диагностики варикозных болезней ног.

\begin{figure}[h!]
	\centering
    \includegraphics[scale=0.6]{nogi.jpg}
 	\caption{Точки измерения температур для диагностики варикозных заболеваний ног}
 \vspace*{5mm}
\end{figure}

По результатам измерений каждому пациенту будет поставлен в соответствие $m$-мерный вектор  температур. Можно проводить первичную диагностику непосредственно по набору температур, но эффективность такой диагностики оставляет желать лучшего. Для улучшения качества диагностики необходимо предварительно сформировать пространство высокоинформативных диагностических признаков. В работах \cite{los6,los7} подробно описана методика формирования такого пространства как для парных, так и для непарных органов. Анализ данных начинается с исследования качественных признаков, выявленных специалистами-медиками.
 Например, для онкологических заболеваний молочных желез такими признаками являются повышенная величина термоасимметрии между одноименными точками молочных желез,  повышенный разброс температур между отдельными точками в пораженной молочной железе, соотношение кожной и глубинной температур и некоторые другие (см.: \cite{an1,los2}). По каждому такому признаку строится набор числовых моделирующих функций, при этом число таких функций может быть достаточно большим. Например, как отмечено в \cite{los6}, термоасимметрия полей температур парных органов может быть описана различными функциями вида $$
 g(f(t_{i,\text{пр}},\dots,t_{n,\text{пр}})-f(t_{i,\text{л}},\dots,t_{n,\text{л}})),
 $$
 где $f(t_{i,\text{пр}},\dots,t_{n,\text{пр}})$ --- значение функции температур для точек $t_{i},\dots,t_{n}$ правого парного органа; $f(t_{i,\text{л}},\dots,t_{n,\text{л}})$ --- значение аналогичной функции температур для точек $t_{i},\dots,t_{n}$ левого парного органа, а $g(x)$ --- некоторая функция одного переменного.

 Введем следующие обозначения. Пусть $t_{i}^{j}$,$j=1,\dots,m$ -- температура $i$-й точки $j$-го пациента. Тогда обучающую выборку можно представить в виде матрицы

 $$
 \left(
 \begin{array}{lcr}
 t_{1}^{1} & \cdots & t_{m}^{1} \\
 \vdots & \ddots & \vdots \\
 t_{1}^{k}  & \cdots & t_{m}^{k} \\
 t_{1}^{k+1}  & \cdots & t_{m}^{k+1} \\
 \vdots & \ddots & \vdots \\
 t_{1}^{n}  & \cdots & t_{m}^{n} \\
 \end{array}
 \right),
 $$
 где $t_{i}^{j}$ при $j=1,\dots,k$ --- температуры здоровых пациентов; $t_{i}^{j}$ при $j=k+1,\dots,n$ --- температуры больных пациентов.

Пусть $f=f(t_{1},t_{2},\dots,t_{m})$ --- моделирующая функция, рассматриваемая на множестве векторов $\Omega=\{(t_1^j, \dots , t_{m}^j)\},$ где $j=1, \dots , n.$
 Обозначим $f_{j}=f(\vec{t_{j}})$, где $\vec{t_{j}}=(t_{1}^{j}, t_{2}^{j},\dots,t_{m}^{j})$, $j=1,\dots,n$.
Назовем характеристическим признаком пару $(f, X),$ где $X \subset R$ --- максимальное (в каком-то смысле) подмножество области изменения функции $f$ такое, что
$f_j \in X$ для некоторых $j=k+1, \dots , n$ и $f_j \in R\setminus X$ для всех $j=1, \dots , k.$
Если характеристический признак для данного пациента выполняется, пациента автоматически считаем больным.

Назовем высокоинформативным признаком  тройку $(f,X,I)$, где  $X$ --- область изменения функции $f$; $I$ --- величина информативности. Информативность -- количественный параметр, определяющий, насколько хорошо признак описывает отличие одной группы (искомой) от другой (отделяемой). Чем выше информативность признака классификации, тем лучше данный признак отделяет одну группу от другой. Существует несколько видов информативностей: статистическая эвристическая, энтропийная, комбинированная информативность (см.: \cite{los6,los7}).  Если выполняется высокоинформативный признак, пациента нельзя автоматически считать больным, но выполнение каждого высокоинформативного признака повышает вероятность того, что пациент болен.

По небольшому набору качественных медицинских признаков можно построить достаточно много высокоинформативных признаков (например, в задаче диагностики заболеваний молочных желез их более тысячи). В работах \cite{,los7} изложены алгоритмы, позволяющие выбрать для диагностики минимизированные наборы признаков.  Исследованы простые алгоритмы диагностики на основе таких наборов.
Качество алгоритма медицинской диагностики обычно характеризуется  специфичностью $Sp$ (долей верно диагностированных здоровых)  и чувствительностью $Se$ (долей верно диагностированных больных). В качестве критерия эффективности будем рассматривать среднее геометрическое этих величин $D=\sqrt{Sp\cdot Se}$.


Целью данной работы является создание, реализация и проверка эффективности двух алгоритмов диагностики на базе пространства высокоинформативных признаков.  Первый алгоритм для диагностики использует нейронные сети. Второй алгоритм представляет собой вариант алгоритма некомпенсаторного агрегирования, при этом для выставления  экспертных оценок используется аппарат нечетких множеств. В процессе многочисленных вычислительных экспериментов данные алгоритмы были протестированы на двух задачах: диагностике варикозных заболеваний нижних конечностей и диагностике заболеваний молочных желез.


\section{Нейросетевой алгоритм и его модификация}

Для выбора оптимальной конфигурации нейронной сети был разработан и реализован на языке C\# программный модуль, который строит, обучает, а затем тестирует нейронную сеть. Проводились вычислительные эксперименты для задачи диагностики заболеваний молочных желез. Рассматривались  различные виды архитектуры и методы обучения нейросети. На входы подавались нормированные наборы моделирующих функций и некоторые дополнительные показатели (число точек боли и т. п.).

Исследовались показатели многослойного персептрона с двумя нейронами на выходном слое, пользующегося популярностью в большом спектре задач. Каждый нейрон выходного слоя принадлежит одному из классов, на которые нужно разделить примеры из обучающего множества. Выходной нейрон, набравший наибольший выходной сигнал при получении примера на входной слой, считается победителем, и нейронная сеть относит полученный пример к тому классу, за который отвечал победивший нейрон. Число скрытых слоев и нейронов на них варьировалось в процессе проведения экспериментов.

\begin{figure}[h]
	\centering
	\includegraphics[scale=0.5]{Perc.png}
	\caption{Многослойный перцептрон}
\end{figure}

Также в процессе разработки была апробирована нейронная сеть каскадной корреляции, способная в процессе обучения изменять количество нейронов на скрытом слое.

Для обучения нейронной сети были использованы три различных алгоритма: метод обратного распространения ошибки, метод Хука --- Дживса и метод имитации отжига (см. \cite{Ya1}). Как видно из таблиц 1--2, все эти методы показали не очень высокую эффективность.

С целью повышения эффективности была разработана модификация архитектуры нейронной сети каскадной корреляции, позволяющая учитывать особенности и изолировать аномалии среди примеров из обучающего множества путем увеличения размерности выходного слоя.  Модификация заключается в возможности добавления нейронов выходного слоя в процессе обучения нейросети. Так, данная нейронная сеть может иметь на выходном слое сразу несколько нейронов, отвечающих за один класс из всего числа классов, на которые нужно разделить примеры из обучающего множества.

Обозначим $K$ количество классов, на которые нужно разбить поступающие входные векторы. (В наших задачах полагаем $K=2$.) Модифицированный алгоритм построения нейронной сети имеет следующий вид:

Шаг 1: создание и обучение стандартного персептрона. Каждый $k$-й выходной нейрон из числа всех выходных нейронов $K$ получает метку с индексом $k$, которая показывает принадлежность этого нейрона одному из классов.

Шаг 2: для каждого $k$-го класса из числа всех классов $K$ выполняются шаги 3--4. После прохода по всем классам перейти на шаг 5.

Шаг 3: вернуть изначальные веса и добавить на выходной слой нейрон с меткой $k$-го класса, который соединится со всеми входными нейронами, после чего обучить нейронную сеть.

Шаг 4: если при добавлении нейрона результат улучшается незначительно --- закончить добавление нейронов и вернуться на шаг 2, выбрав следующий класс, иначе вернуться на шаг 3.

Шаг 5: построить скрытый слой методом каскадной корреляции.

Мы видим, что по сравнению со стандартным алгоритмом обучения сети каскадной корреляции добавлен шаг 3, что позволяет для каждого класса иметь несколько нейронов на выходном слое. Пример из обучающего множества, поданный на входы нейронной сети, будет отнесен к тому классу, метку которого имеет выходной нейрон, набравший наибольший выходной сигнал, независимо от того, каким по порядку этот нейрон был добавлен. Данная архитектура  изолирует в отдельных нейронах примеры, которые относятся к одному классу, но не имеют общих свойств или их свойства частично совпадают с примерами, принадлежащими другому классу.

Для ускорения процесса обучения применялись методы распараллеливания вычислений. В процессе обучения участвовало порядка 30 нейронных сетей одной архитектуры, но имеющих различные начальные веса. Как только обучение всех сетей завершалось, из них выбиралась та, что показала лучшие результаты.

Для анализа качества разработанного программного модуля использовалось сравнение точности его работы с точностью диагностики нейронных сетей, включенных в программный пакет AForge.

Перед обучением изначальное обучающее множество было разбито на обучающую и тестовую выборку. При этом тестовая выборка составила 60\,\% обучающего множества. В таблицах 1--2 приведены результаты диагностики.

\begin{table}[h]
	\caption{Результаты диагностики <<рак --- не рак>>}
	\centering
	\label{tabluar:tabResults}\small
	\begin{tabular}{|l|l|c|}
	
		\hline
	Архитектура & {Метод обучения} &  $D$, \%  \\
		\hline
	Перцептрон & Обратное  распространение & 55,9   \\
		\hline
	Перцептрон & Метод Хука --- Дживса & 48,3   \\
		\hline
	Перцептрон & Имитация отжига & 54,6   \\
		\hline
	Каскадная сеть & Имитация отжига & 55,8   \\
		\hline
	
	Каскадная сеть & Обратное распространение & 52,7   \\
		\hline
		Нейронная сеть с построением выходного слоя & Имитация отжига & 73,4   \\
		\hline
	
	\end{tabular}
\end{table}


\begin{table}[h]
	\caption{Результаты диагностики <<здоров --- болен>>}
	\centering
	\label{tabluar:tabResults}\small
	\begin{tabular}{|l|l|c|}
		
		\hline
		{Архитектура} & {Метод обучения} &  $D$, \%  \\
		\hline
		Перцептрон & Обратное  распространение & 57,9   \\
		\hline
		Перцептрон & Метод Хука --- Дживса & 49,6   \\
		\hline
		Перцептрон & Имитация отжига & 59,7   \\
		\hline
		Каскадная сеть & Имитация отжига & 57,3   \\
		\hline
		
		Каскадная сеть & Обратное распространение & 56   \\
		\hline
		Нейронная сеть с построением выходного слоя & Имитация отжига & 72,7  \\
		\hline
		
	\end{tabular}
\end{table}

Анализ полученных результатов позволяет сделать вывод, что максимальная эффективность диагностики достигается при использовании нейронной сети с построением выходного слоя, обученной методом имитации отжига. При этом предложенная авторами модификация сети, заключающаяся в добавлении дополнительных нейронов на выходной слой сети каскадной архитектуры, увеличивает эффективность на 10--12\,\%.

В качестве второй тестовой задачи рассматривалась диагностика варикозных заболеваний ног. Запуск программы проводился на каскадной нейронной сети с построением выходного слоя и скрытого слоя.

В качестве входных параметров использовались нормированные температуры пациентов, информация о наличии точек боли, варикозных расширений сосудов, кожных изменений и отека.

База данных пациентов была поделена на обучающую и тестирующую выборку (см. табл. 3):

\begin{table}[h]
	\caption{Варикоз. Количество пациентов в тестовой и обучающей выборке}
	\centering
	\label{tabluar:tabResults}\small
	\begin{tabular}{|c|c|c|}
		
		\hline
		Пациенты & {Варикоз} &  {Не варикоз}  \\
		\hline
		Обучающая выборка & 34 & 92   \\
		\hline
		Тестовая выборка & 22 & 62   \\
		\hline
		
		
	\end{tabular}
\end{table}

Результаты запусков показали, что в процессе обучения достигалась 100\,\%-я эффективность на обучающей выборке даже без увеличения выходного слоя.

\begin{table}[h]
	\caption{Результаты диагностики <<варикоз --- не варикоз>>}
	\centering
	\label{tabluar:tabResults}\small
	\begin{tabular}{|c|c|c|}
		
		\hline
		{Номер запуска} & {Количество нейронов выходного слоя} &  $D$, \%  \\
		\hline
		1 & 4 & 70,3   \\
		\hline
		2 & 3 & 53,6   \\
		\hline
		3 & 4 & 71,9   \\
		\hline
		4 & 2 & 49,9   \\
		\hline
		
		5 & 3 & 66,2   \\
		\hline
		
		
	\end{tabular}
\end{table}

При этом увеличение числа нейронов выходного слоя приводило к повышению эффективности на тестовой выборке (см. табл. 4). В связи с этим предпринималась попытка искусственно добавлять нейроны на выходной слой, но увеличение эффективности не было достигнуто.

\section{Гибридный алгоритм нечеткой классификации\\ и некомпенсаторного агрегирования}

Предлагаемый алгоритм состоит из двух этапов. На первом этапе по каждой моделирующей функции $f$ из минимизированного набора признаков строится нечеткое множество. На втором этапе методом, аналогичным методу некомпенсаторного агрегирования (см. \cite{al1}), ставится диагноз.

Чтобы построить по функции $f$ нечеткое множество, разбиваем множество значений функции на промежутки. Для статистической значимости потребуем, чтобы на каждом промежутке находилось не менее $N$ значений функции на элементах обучающей выборки. Начинаем с равномерного разбиения, но если условие статистической значимости при этом нарушается, переходим к неравномерному, увеличивая длины соответствующих отрезков.  Пусть $P_1$ --- доля здоровых пациентов, попавших в наш промежуток,  $P_2$ --- доля больных. Для каждого промежутка $(a;\,b)$ находим степень принадлежности $y = P_2 / (P_1 + P_2)$. Фиксируем точку с координатами ($(a+b)/2$;\,$y$). По отмеченным на каждом промежутке точкам строим методом наименьших квадратов многочлен четвертой степени. Данный многочлен и задает функцию принадлежности нечеткого множества, соответствующего $f$. На рисунке 3 показан пример нечеткого множества для одной из функций, моделирующих термоасимметрию молочных желез.


\begin{figure}[h]
	\centering
	\includegraphics[scale=0.5]{kart.png}
	\caption{Функция принадлежности нечеткого множества}
\end{figure}

По набору функций принадлежности строится алгоритм классификации, зависящий от скалярного параметра $\alpha$ и  вектора $\beta = (\beta_{1},\dots,\beta_{k})$, которые подбираются в процессе вычислительных экспериментов с целью обеспечения наибольшей эффективности алгоритма.

\begin{table}[h]
	\caption{Эффективность диагностики варикозных заболеваний ног}\small
	\centering
	\label{tabluar:tabResults}
	\begin{tabular}{|c|c|c|c|}
		
		\hline
		{Количество признаков} & {Специфичность, \%} & {Чувствительность, \%} &  $D$, \%  \\
		\hline
		8 & 77 & 63 & 69  \\
		\hline
		8 & 68 & 80 & 73 \\
		\hline
		10 & 56 & 80 & 66  \\
		\hline
		10 & 61 & 74 & 67 \\
		\hline
		
		
		
	\end{tabular}
\end{table}

 Алгоритм состоит из трех шагов:

1)	Если для некоторого $p=1,\dots,k$ можно найти $p$ моделирующих функций из нашего набора, у которых значения функций принадлежности больше $\beta_p$, то относим пациента к группе больных.

2)	Если значения всех моделирующих функций меньше $\alpha$, считаем пациента здоровым.

3)	В противном случае пациент остается без диагноза.





\begin{table}[h]
	\caption{Эффективность диагностики онкологических заболеваний молочных желез}
	\centering
	\label{tabluar:tabResults}\small
	\begin{tabular}{|c|c|c|c|}
		
		\hline
		{Количество признаков} & {Специфичность, \%} & {Чувствительность, \%} &  $D$, \%  \\
		\hline
		10 & 81 & 71 & 76  \\
		\hline
		10 & 75 & 71 & 73 \\
		\hline
		9 & 85 & 60 &71  \\
		\hline
		9 & 84 & 62 & 72 \\
		\hline
		
		
		
	\end{tabular}
\end{table}

Данный алгоритм также тестировался на задаче диагностики варикозных заболеваний ног и на задаче диагностики онкологических заболеваний молочных желез. Как видно из таблиц 5 и 6, эффективность диагностики на тестовой выборке составила около 70\,\% для заболеваний молочных желез и около 75\,\% для варикозных болезней ног.

\section*{Заключение}

В настоящее время коллективом специалистов института математики и информационных технологий ВолГУ разрабатывается консультативная информационная система диагностики онкологических заболеваний молочной железы, помогающая врачу диагностировать пациенток методом комбинированной термометрии.  В \cite{zen1} описана предполагаемая архитектура такой системы. Главной частью системы является блок принятия решений, который делится на консультативный и экспертный блок. При этом задача консультативного блока --- предложить медику дополнительную информацию о температурных аномалиях, помогающих поставить диагноз, задача алгоритмов экспертного блока --- участвовать в постановке предварительного диагноза. В настоящее время реализован на языке С\# прототип информационной системы. В экспертный блок прототипа встроен модуль, реализующий предложенную авторами модификацию нейросетевого алгоритма.

Завершена разработка модуля, реализующего гибридный алгоритм, в ближайшее время планируется его внедрение в консультативный блок системы.

Авторы выражают благодарность В.В. Левшинскому за предоставленные наборы признаков.



\begin{thebibliography}{99}
	

	

	\bibitem{zen1}
	\baut{Зенович}{А.В.}\baut{Глазунов}{В.А.}\baut{Опарин}{А.С.}\baut{Примаченко}{Ф.Г.}\btit{Алгоритмы принятия решений в консультативной интеллектуальной системе диагностики молочных
		желез}[Algorithms of decision-making in intelligent Advisory system for diagnostics of the mammary
	glands] \bj{Вестник Волгоградского
		государственного университета. Серия 1, Математика.
		Физика}\byr{2016}\bnum{6 (31)}\bpp{129--142}\mkpaperr
	
	\bibitem{al1}
	\baut{Алескеров}{Ф.Т.}\baut{Юзбашев}{Д.А.}\baut{Якуба}{В.И.}\btit{Пороговое агрегирование трехградационных ранжировок}[Intelligent analysis of the thermometric data in the diagnosis of breast] \bj{Автоматика и телемеханика}[Automation and Remote Control]\byr{2007}\bnum{1}\bpp{147--152}\mkpaperr
	
	
	\bibitem{los1}
	\baut{Веденяпин}{Д.А.}\baut{Лосев}{А.Г.}\btit{Об одной нейросетевой модели диагностики венозных заболеваний}[About one neural network model of diagnosis of venous diseases] \bj{Управление большими системами}\byr{2012}\bnum{39}\bpp{219--229}\mkpaperr
	
	\bibitem{vka1}
	\baut{Веснин}{С.Г.}\baut{Каплан}{А.М.}\baut{Авакян}{Р.С.}\btit{Современная
		микроволновая радиотермометрия молочных желез}[Modern microwave
	radiothermometry of mammary Glands] \bj{Медицинский альманах}\byr{2008}\bnum{3}\bpp{82--87}\mkpaperr
	
	
	\bibitem{los4}
	\baut{Замечник}{Т.В.}\baut{Овчаренко}{Н.С.}\baut{Ларин}{С.И.}\baut{Лосев}{А.Г.}\btit{Изучение надежности комбинированной термографии как метода диагностики состояния вен нижних конечностей}[Reliability assessment of combined thermography as a method for the characteristic of the lower limb venous system] \bj{Флебология}\byr{2010}\bvol{3}\bnum{4}\bpp{23--26}\mkpaperr
	
	\bibitem{kob1}
	\baut{Кобринский}{Б.А.}\btit{Консультативные интеллектуальные
		медицинские системы: классификация, принципы построения,
		эффективность}[Consulting intelligent medical systems:
	classification, principles of construction, efficiency] \bj{Врач и
		информационные технологии}\byr{2008}\bnum{2}\bpp{38--47}\mkpaperr
	
	\bibitem{los7}
	\baut{Лосев}{А.Г.}\baut{Левшинский}{В.В.}\btit{Интеллектуальный анализ данных микроволновой радиотермометрии в диагностике рака молочной железы}[Intelligent analysis of microwave radiometry data in the diagnosis of breast cancer] \bj{Математическая физика и компьютерное моделирование}[Mathematical Physics and Computer Simulation]\byr{2017}\bnum{5}\bpp{49-62}\mkpaperr
	
\bibitem{los6}
\baut{Лосев}{А.Г.}\baut{Левшинский}{В.В.}\btit{Интеллектуальный анализ термометрических данных в диагностике молочных желез}[Intelligent analysis of the thermometric data in the diagnosis of breast] \bj{Управление большими системами}\byr{2017}\biss{70}\bpp{113--135}\mkpaperr


\bibitem{los3}
\baut{Лосев}{А.Г.}\baut{Мазепа}{Е.А.}\baut{Сулейманова}{Х.М.}\btit{О
	взаимосвязи некоторых признаков РТМ-диагностики заболеваний
	молочных желез}[The relationship between some signs of RTM-
diagnosis of breast diseases] \bj{Вестник Волгоградского
	государственного университета. Серия 1, Математика.
	Физика}\byr{2015}\bnum{4 (29)}\bpp{35-44}\mkpaperr

\bibitem{los2}
\baut{Лосев}{А.Г.}\baut{Мазепа}{Е.А.}\baut{Замечник}{Т.В.}\btit{О
	некоторых характерных признаках в диагностике патологии молочных
	желез по данным микроволновой радиотермометрии}[Some typical
symptoms in the diagnosis of breast pathology according to
microwave radiometry] \bj{Современные проблемы науки и
	образования}\byr{2014}\bnum{6}\baid{16587}\mkpaperr

\bibitem{an1}
\baut{Анисимова}{Е.В.}\baut{Замечник}{Т.В.}\baut{Лосев}{А.Г.}\baut{Мазепа}{Е.А.}\btit{О
	некоторых характерных признаках в диагностике заболеваний нижних
	конечностей методом комбинированной термографии}[Some characteristic signs in diagnostics of venous diseases of lower extremities by the method of combined thermography] \bj{Вестник новых
	медицинских технологий}\byr{2011}\bnum{18~(2)}\bpp{329--330}\mkpaperr



\bibitem{los4}
\baut{Лосев}{А.Г.}\baut{Хоперсков}{А.В.}\baut{Астахов}{А.С.}\baut{Сулейманова}{Х.М.}\btit{Проблемы измерения и моделирования тепловых и радиационных полей в биотканях: анализ данных микроволновой радиотермометрии}[Problems of measurement and modeling of thermal and radiation fields in biological tissues: analysis of microwave thermometry data] \bj{Вестник Волгоградского
	государственного университета. Серия 1, Математика.
	Физика}\byr{2015}\bnum{6 (31)}\bpp{31-71}\mkpaperr

\bibitem{los5}
\baut{Замечник}{Т.В.}\baut{Ларин}{С.И.}\baut{Лосев}{А.Г.}\baut{Овчаренко}{Н.С.}\btit{Способ комбинированной термометрии и математические модели вероятностной диагностики заболеваний вен нижних конечностей}[The Way of the Combined Thermometry and Mathematical Models of Probabilistic Diagnostics of Diseases of the Lower Limbs Veines] \bj{Вестник новых медицинских технологий}\byr{2009}\bvol{16}\bnum{4}\bpp{14--16}\mkpaperr






%\bibitem{kor1}
%\baut{Кореневский}{Н.А.}\baut{Лукашов}{М.И.}\baut{Артеменко}{М.В.}\baut{Агарков}{Н.М.}
%\btit{Синтез гибридных нечетких решающих правил для классификации%
%	клинических вариантов течения генитального герпеса на основ%е
%	моделей системных взаимосвязей}[Synthesis of hybrid fuzzy decision
%rules for classification Clinical variants of the genital herpes
%based models of systemic linkages] \bj{Фундаментальные
%	исследования}[Basic
%research]\byr{2014}\bnum{10}\bpp{901--907}\mkpaperr







	\bibitem{Ya1}
	\baut{Ясницкий}{Л.Н.}\btit{Введение в искусственный интеллект :
		учеб. пособие для студентов вузов}[Introduction to artificial intelligence:
	textbook for University students] \bcity{М.}\bpub{Academia}
	\byr{2005}\bpp{176}\mkbookr
	
	
	
	


\end{thebibliography}

\begin{summary}  In this paper we introduce two algorithms for diagnosing diseases of paired organs by the method of combined radio thermometry. The first one is based on neural networks and the second one is based on the apparatus of fuzzy sets.

 We consider a new modification of the neural network architecture for constructing the neural network algorithm, which involves the automatic addition of neurons to the output layer during the learning of the neural network. Compu\-ta\-ti\-o\-nal experiments were carried out to diagnose varicose leg diseases and breast diseases. These experiments showed that this modification improves the efficiency of the algorithm by 10--12\,\%.

 The diagnostic algorithm based on fuzzy sets on the grounds of diagnosis builds fuzzy sets, after which the diagnosis is set by a method analogous to the method of non-compensatory aggregation. Besides, the algorithm was tested for varicose diseases and breast diseases.


\end{summary}



%\end{document}
