%\documentclass[a4paper,11pt,twoside]{article}
%\usepackage{vvmph2}
%\usepackage{rumathbr}
%
%%\newcommand{\grad}{\mathop\mathrm{grad}\nolimits}
%%\newcommand{\diverge}{\mathop\mathrm{div}\nolimits}
%
%\begin{document}

\udk{524.3-17}
\bbk{22.66}
\art[Гидродинамический механизм формирования]{Гидродинамический механизм формирования \plb и коллимации струй \plb в молодых звездных объектах}{The hydrodinamical mechanism of jets' formation \plb and collimation  in young star objects}
\support{Работа выполнена при финансовой поддержке РФФИ: гранты № 16-07-01037, 15-02-06204}

\fio{Кузьмин}{Николай}{Михайлович}
\sdeg{кандидат физико-математических наук}{Candidate of Physical and Mathematical Sciences}
\pos{доцент кафедры информационных систем\\ и компьютерного моделирования}{Associate Professor,\\ Department of Information Systems and Computer Simulation}
\emp{Волгоградский государственный университет}{Volgograd State University}
\email{nikolay.kuzmin@volsu.ru}
\addr{просп. Университетский, 100, 400062 г. Волгоград, Российская Федерация}{\vspace*{-0.5mm}Prosp. Universitetsky, 100, 400062 Volgograd, Russian Federation}

\fio{Мусцевой}{Виктор}{Васильевич}
\sdeg{доктор физико-математических наук}{Doctor of Physical and Mathematical Sciences}
\emp{временно безработный}{Temporarily Unemployed}
%\emp{Волгоградский государственный университет}{Volgograd State University}
\email{vvmusts@mail.ru}
%\addr{просп. Университетский, 100, 400062 г. Волгоград, Российская Федерация}{Prosp. Universitetsky, 100, 400062 Volgograd, Russian Federation}

\fio{Храпов}{Сергей}{Сергеевич}
\sdeg{кандидат физико-математических наук}{Candidate of Physical and Mathematical Sciences}
\pos{доцент кафедры информационных систем\\ и компьютерного моделирования}{Associate Professor,\\ Department of Information Systems and Computer Simulation}
\emp{Волгоградский государственный университет}{Volgograd State University}
\email{xss-ip@mail.ru, infomod@volsu.ru}
\addr{просп. Университетский, 100, 400062 г. Волгоград, Российская Федерация}{Prosp. Universitetsky, 100, 400062 Volgograd, Russian Federation}

\maketitle

\begin{abstract}
Представлены результаты численного гидродинамического моделирования
эволюции ударных оболочек в молодых звездных объектах.
Показано, что в процессе расширения такой оболочки внутри нее
посредством развития последовательности <<выброс --- тор --- торнадо --- джет>>
формируется медленно вращающаяся сверхзвуковая коллимированная струя,
сформированная из вещества околозвездного диска. Механизм возникновения
и коллимации струйного течения является чисто
гидродинамическим и основывается на выполнении закона сохранения
углового момента; он должен работать для всех аккреционно-струйных
систем, а наличие дополнительных факторов (например, магнитных полей)
способно модифицировать такой механизм, но не устранить его.
\end{abstract}

\keywords{струйные истечения, молодые звездные объекты, объекты Хербига --- Аро, численное моделирование, механизмы коллимации}{jet outflows, young star objects, Herbig --- Haro objects, numerical simulation, mechanisms of collimation}

%*********************************************************************
\section*{Введение}
\label{sect:kmx:intro}
%*********************************************************************


Струйные выбросы материи из молодых звезд уже давно являются объектом
пристального изучения астрономов и астрофизиков (см., например,
работы~\cite{kmx:mundt-1986, kmx:mundt-etal-1987, kmx:mundt-1987, kmx:raga-etal-1991, kmx:mundt-etal-1991, kmx:little-1994}).
По современным представлениям, эти выбросы являются следствием
закономерного этапа эволюции молодых звезд "--- этапа аккреционно-струйных
систем, типичными представителями которых являются HH-объекты
(например, HH1/HH2, HH30, HH34, HH47). В настоящий момент не
существует общепринятого самосогласованного сценария возникновения
и коллимации струйных истечений в таких объектах. В данной работе
мы описываем гидродинамический механизм, приводящий
к формированию и коллимации узкой сверхзвуковой вращающейся струи
внутри головной ударной волны (оболочки), формирующейся
вследствие начального сверхзвукового выброса вещества из молодой
звезды.


%*********************************************************************
\section{Модель и основные уравнения}
\label{sect:kmx:model}
%*********************************************************************


Рассматриваемая в работе модель молодого звездного объекта включает
в себя центральное ядро протозвезды, создающее
сферически-симметричный ньютоновский гравитационный
потенциал, и вращающийся вокруг него квазикеплеровский околозвездный
диск постоянного угла раствора. Оба эти компонента погружены в
окружающую среду, представляющую собой остаток протозвездного облака.
Мы предполагаем, что когда в молодой звезде <<включаются>> термоядерные
реакции и скачком возрастают давление и температура, происходит начальный
двусторонний выброс вещества, коллимируемый околозвездным диском,
вдоль оси симметрии системы из околоядерной области.

Рассмотрение проводим в сферической системе координат
${(r, \theta, \varphi)}$, ось ${\theta = 0}$ которой совпадает
с осью симметрии системы, а начало координат "--- с положением центра
масс протозвезды.
Околозвездный диск и окружающую среду моделируем идеальным невязким
нетеплопроводным несамогравитирующим газом, для которого
справедливо уравнение состояния, выписанное в одной из следующих
эквивалентных форм:
\begin{equation}
\frac{p}{\rho} = \frac{c^2}{\gamma} = (\gamma - 1)\varepsilon
= \frac{R}{\mu}T,
\label{eq:kmx:EOS}
\end{equation}
где $p$~--- давление; $\rho$~--- плотность; $c$~--- адиабатическая
скорость звука; $\gamma$~--- показатель адиабаты;
$\varepsilon$~--- массовая плотность внутренней энергии газа;
$R$~--- универсальная газовая постоянная; $\mu$~--- молярная масса;
$T$~--- абсолютная температура. Считаем газ
одноатомным с ${\gamma = 5/3}$, а течение во всех областях полагаем
осесимметричным: ${\partial/\partial\varphi \equiv 0}$.

Осесимметричные стационарные радиальные распределения параметров
газа (обозначаемые нижним индексом <<$0$>>) в этом случае имеют
следующий вид~\cite{kmx:levin-etal-1999}:
\begin{equation}
\rho_0(r, \theta) = \hat\rho_0(\theta) r^{-3/2},\quad
\mathbf{V}_0(r, \theta) = \hat{\mathbf{V}}_0(\theta) r^{-1/2},\quad
p_0(r, \theta) = \hat p_0(\theta) r^{-5/2},
\label{eq:kmx:r-distribution}
\end{equation}
где ${\mathbf{V} = (u, v, w)}$~--- скорость газа; а шляпками
обозначены нормировочные константы радиальных распределений.
Отметим, что такие же точно степенные распределения этих
параметров были независимо получены в несколько другой постановке
задачи в работе~\cite{kmx:ouyed-etal-1997}.

Эволюция рассматриваемой системы описывается уравнениями
газовой динамики с учетом гравитационного потенциала ядра протозвезды,
нагрева газа излучением центрального объекта (молодой
звезды) и охлаждения за счет высвечивания:
\begin{equation}
\left\{
\begin{array}{l}
{\displaystyle \frac{\partial\rho}{\partial t} + \mathop\mathrm{div}\nolimits(\rho\mathbf{V}) = 0},
\\ \\
{\displaystyle \frac{\partial}{\partial t}(\rho\mathbf{V})
+ \mathop\mathrm{div}\nolimits(\rho\mathbf{V}\otimes\mathbf{V})
+ \mathop\mathrm{grad}\nolimits p = -\rho\mathop\mathrm{grad}\nolimits\Psi},
\\ \\
{\displaystyle \frac{\partial e}{\partial t} + \mathop\mathrm{div}\nolimits[(e + p)\mathbf{V}]
= -\rho\mathbf{V} \mathop\mathrm{grad}\nolimits\Psi + \rho(\Gamma - \rho\Lambda)},
\end{array}
\right.
\label{eq:kmx:HD-system}
\end{equation}
здесь $\Psi$~--- гравитационный потенциал центрального объекта;
${e = p/(\gamma-1) + \rho\mathbf{V}^2/2}$~--- объемная плотность
энергии газа; $\Gamma$ и $\Lambda$~--- функции радиационного
нагрева и охлаждения соответственно. В численных расчетах
для функции нагрева использовалось выражение
${\Gamma = C_\Gamma \varepsilon_0^{1/2} r^{-2}}$, полученное
в работе~\cite{kmx:kuzmin-etal-2002} из условия стационарного
баланса, здесь $C_\Gamma$~--- нормировочная постоянная,
$\varepsilon_0$~--- равновесное значение массовой плотности
внутренней энергии газа на данном радиусе. В стационарном
состоянии при этом выполнялось условие
${\Gamma - \rho\Lambda = 0}$. При численном моделировании эволюции
системы использовалась функция охлаждения из
работ~\cite{kmx:hardee-etal-1997, kmx:macdonald-etal-1981},
аппроксимируемая при расчетах кубическими сплайнами.


%*********************************************************************
\section{Численное моделирование}
\label{sect:kmx:numerical}
%*********************************************************************


На основе описанной в пункте~\ref{sect:kmx:model} стационарной модели
нами было проведено численное моделирование эволюции системы после
начального сверхзвукового выброса вещества из околоядерной области.

Нами было проведено несколько серий моделирования, различающихся
значениями безразмерных параметров подобия: масса ядра протозвезды
${M_* / M_\odot = 10}$, число Маха выброса по отношению к окружающей
среде ${M_j = 3 \div 10}$, число Маха на границе между диском
и окружающей средой ${M_d = 10 \div 40}$, перепад плотностей от
вещества выброса к окружающей среде составлял
${\rho_j/\rho_0 = 5 \div 20}$, перепад температур~---
${T_j / T_0 = 3 \div 5}$, угол полураствора диска варьировался
в пределах ${\theta_d = 10^\circ \div 40^\circ}$, угол полураствора
начального выброса составлял ${\theta_j = 5^\circ \div 10^\circ}$.

Для перехода к реальным значениям физических величин были
введены следующие характерные масштабы обезразмеривания:
${l_r = 1\text{ а. е.} \simeq 1.5 \times 10^{11}\text{ м}}$~---
единица длины,
${l_t = 9.9 \times 10^6\text{ с} \simeq 0.3\text{ года}}$~---
единица времени,
${l_\rho = 8 \times 10^{-20}\text{ кг/м$^3$}}$~---
единица плотности (что примерно соответствует концентрации
${50\text{ см}^{-3}}$).

В соответствии с предположением об азимутальной симметрии
системы вычислительная область выбиралась в виде
${\{r_{min} \leq r \leq r_{max}, 0 \leq \theta \leq \theta_{max}\}}$.
Для случая симметричного биполярного (двустороннего) выброса
в силу наличия дополнительной симметрии относительно плоскости
${\theta = \pi/2}$ выбиралось ${\theta_{max} = \pi/2}$, а для
случая униполярного (одностороннего) выброса, представляющего
по большей части академический интерес, выбиралось
${\theta_{max} = \pi}$; кроме того, для обоих случаев
полагалось ${r_{min} = 5\text{ а. е.}}$ и
${r_{max} = 250\text{ а. е.}}$

Для численного интегрирования системы уравнений~\eqref{eq:kmx:HD-system}
использовалась схема типа MUSCL~\cite{kmx:muscl}, адаптированная
нами для применения в сферической системе координат с начальными
распределениями параметров вида~\eqref{eq:kmx:r-distribution}.
Численный код подробно описан в работе~\cite{kmx:kuzmin-etal-2007}.
Максимальный размер равномерной расчетной сетки составлял
${L \times M = 2\,500 \times 720}$, что позволило достаточно хорошо
разрешить структуру течения.

Начальное распределение газодинамических величин задавалось
в соответствии со стационарной моделью~\cite{kmx:levin-etal-1999} по
формулам~\eqref{eq:kmx:r-distribution}. Вдоль линий ${\theta = 0}$
(ось симметрии системы) и ${\theta = \pi/2}$ (плоскость симметрии
системы для случая биполярного выброса) задавались стандартные граничные
условия для оси и плоскости симметрии соответственно. Вдоль линии
${r = r_{max}}$ (внешняя граница) ставились условия свободного
протекания, вдоль линии ${r = r_{min}}$ (внутренняя граница)
поддерживалось стационарное распределение газодинамических величин,
а в диапазоне углов ${0 \le \theta \le \theta_j}$ задавался выброс
вещества с параметрами, описанными выше. Характер выброса определялся
следующим образом:
\begin{equation*}
f(\theta, t) = f_0 + A_f
\frac{1}{2}
\bigg[1 - \tanh\bigg(\frac{\theta - \theta_j}{\delta_\theta}\bigg)\bigg]
\cosh^{-2}\bigg(\frac{t-t_p}{\delta_t}\bigg),
\end{equation*}
где ${f = \{\rho, u, T\}}$;
${A_f = \{\rho_j - \rho_0, u_j, T_j - T_0\}}$; $\delta_\theta$ и
$\delta_t$~--- величины, характеризующие степень гладкости
распределения по $\theta$-координате и времени соответственно;
$t_p$~--- момент времени, соответствующий максимальной амплитуде
выброса.

Далее на рисунках мы приводим наиболее характерные результаты,
полученные для следующих значений параметров: ${M_j = 10}$,
${M_d = 20}$, ${\rho_j/\rho_a = 10}$, ${T_j / T_0 = 5}$,
${\theta_d = 20^\circ}$, ${\theta_j = 7.5^\circ}$ в случае
двустороннего выброса и ${M_j = 7}$, ${M_d = 20}$,
${\rho_j/\rho_a = 10}$, ${T_j / T_0 = 5}$,
${\theta_d = 10^\circ}$, ${\theta_j = 10^\circ}$ в случае
одностороннего выброса.


%*********************************************************************
\section{Обсуждение результатов}
\label{sect:kmx:discussion}
%*********************************************************************


Структура потоков вещества внутри оболочки, созданной ударной волной,
оказывается очень сложной. Наряду с волной, прошедшей внутрь
аккреционного диска и отраженной диском в оболочку, в системе
также присутствуют иерархические по пространственным и временной
координатам структуры в виде короткоживущих
и долгоживущих вихрей, отражающих тенденцию к турбулизации вещества.

Результаты численного моделирования позволяют заключить, что
эволюция потоков вещества внутри оболочки может быть разделена
на три этапа.

На первом этапе вещество начального выброса, обладающее
более высоким давлением по отношению к окружающей среде и уже
сформировавшее ударную волну (оболочку), начинает расширяться по
широтному углу, обтекая эту плотную оболочку, и приобретает
угловой момент, ориентированный вдоль азимутальной координаты.

На втором этапе, хотя оболочка к этому времени уже
распространилась вдоль радиальной координаты, вещество, обладающее
азимутальным угловым моментом, формирует тороидальный вихрь
(см. рис.~\ref{fig:kmx:vectors}). Этот долгоживущий вихрь возникает в оболочке
над самой внутренней частью диска. Внутренняя поверхность этого
вихря представляет собой сопло Лаваля (конфигурация
конфузор --- диффузор) и в нем происходит ускорение газа до
сверхзвуковых скоростей.

\begin{figure}[!htb]
\centering
\includegraphics[height=0.55\textwidth, keepaspectratio]{hydrojet1a}
\includegraphics[height=0.45\textwidth, keepaspectratio]{hydrojet1b}
\caption{Примеры векторного поля скоростей в области долгоживущего торообразного вихря\plb для случая биполярного выброса в различные моменты времени: ${t = 5}$ (\emph{слева}) и ${t = 15}$ (\emph{справа})}
\label{fig:kmx:vectors}\vspace*{5mm}
\end{figure}

Отметим, что движения ускоренного газа и вещества, вращающегося
в вихре, сонаправлены, поэтому такая конфигурация не размывается
неустойчивостью Кельвина --- Гельмгольца.
%, в отличие от ситуации
%прорыва плотного газового облака, описанной
%в работе Норман и др. 1988.

Результатом этого процесса является то, что во всех сериях
численных экспериментов внутри оболочки формировалась узкая
сверхзвуковая вращающаяся струя, состоящая из вещества диска,
увлекаемая в оболочку тороидальным вихрем. С формированием таких
сверхзвуковых струй мы связываем удлинение головной части оболочки
(а в последующем и грушевидную форму оболочки) "---
рисунок~\ref{fig:kmx:density2}. Струя догоняет оболочку и вторгается в ее газ,
что приводит к формированию внутренней ударной волны, направленной
фронтом к ядру системы, к передаче энергии головной части оболочки,
ее ускорению и существенному разогреву.

На третьем этапе выброшенное через сопло Лаваля
вращающееся вокруг оси симметрии системы вещество струи под
действием центробежной силы оттекает от оси симметрии тем дальше,
чем дальше оно от центра симметрии системы. В результате это вещество
образует смерчеподобную воронку (см. рис.~\ref{fig:kmx:omega2}), и эта воронка,
в свою очередь, затягивает вещество и выбрасывает его в сторону
головной части оболочки (эффект торнадо). Существенно, что выброшенное
смерчем вещество диска <<надстраивает>> воронку. Таким образом, описанное
течение является самоподдерживающимся. Поэтому, хотя в дальнейшем
торообразный вихрь уже распадается, генерация струи продолжалась
во всех сериях экспериментов до конца моделирования.



Сильно вытянутая форма оболочки, наблюдавшаяся во всех численных
экспериментах на больших временах (см. рис.~\ref{fig:kmx:density2}
и~\ref{fig:kmx:density1}), очень напоминает морфологию головных ударных
волн в молодых звездных объектах с джетами (таких как, например, HH1/HH2).
Вероятно, именно так эти объекты и образуются на более поздних стадиях
эволюции молодой звезды, чем рассмотренные нами. В результате описанного
выше процесса происходит значительный разогрев головной части оболочки.
Поскольку давление в этой части и окружающих ее областях достаточно быстро
выравнивается, плотность в головной части оболочки, соответственно,
значительно уменьшается, что визуально на рисунках воспринимается
как тенденция к прорыву оболочки. Однако, как показывает наше
моделирование, такого прорыва не происходит даже на очень больших
временах.

\begin{figure}[!h]
\centering
\includegraphics[height=0.55\textwidth, keepaspectratio]{hydrojet2a}
\includegraphics[height=0.55\textwidth, keepaspectratio]{hydrojet2b}
\caption{Распределения логарифма относительной плотности ${\lg(\rho/\rho_0)}$ для случая\plb биполярного выброса в различные моменты времени: ${t = 25}$ (\emph{слева}) и ${t = 50}$ (\emph{справа})}
\label{fig:kmx:density2}
\end{figure}

\begin{figure}[!h]
\centering
\includegraphics[height=0.55\textwidth, keepaspectratio]{hydrojet3a}
\includegraphics[height=0.55\textwidth, keepaspectratio]{hydrojet3b}
\caption{Распределения относительной угловой скорости газа ${\Omega/\Omega_0}$ для случая\plb биполярного выброса в различные моменты времени: ${t = 25}$ (\emph{слева}) и ${t = 50}$ (\emph{справа})}
\label{fig:kmx:omega2}
\end{figure}

\begin{figure}[!h]
\centering
\includegraphics[height=0.55\textwidth, keepaspectratio]{hydrojet4a}
\includegraphics[height=0.55\textwidth, keepaspectratio]{hydrojet4b}
\caption{Распределения логарифма относительной плотности
${\lg(\rho/\rho_0)}$ для случая\plb униполярного выброса в различные
моменты времени: ${t = 25}$ (\emph{слева}) и ${t = 50}$ (\emph{справа})}
\label{fig:kmx:density1}
\end{figure}

Достаточно интересным эффектом является наблюдавшийся в экспериментах
с односторонним выбросом прогиб аккреционного диска и его смещение
вдоль оси симметрии в направлении, противоположном начальному
выбросу, "--- рисунок~\ref{fig:kmx:omega1}. Этот эффект обусловлен, очевидно,
импульсом, передаваемым веществу диска прошедшей в него ударной волной,
создаваемой ударной волной оболочки. Во внутренней околозвездной
области в данной ситуации, напротив, основная доля вещества диска
затягивается в оболочку в направлении начального выброса, из-за
чего в реальных объектах должен резко уменьшаться темп аккреции.
Вероятно, аналогичный эффект будет наблюдаться и не при
одностороннем, а вообще при несимметричном выбросе.

\begin{figure}[!h]
\centering
\includegraphics[height=0.3\textwidth, keepaspectratio]{hydrojet5a}
\includegraphics[height=0.3\textwidth, keepaspectratio]{hydrojet5b}
\caption{Распределения относительной угловой скорости газа ${\Omega/\Omega_0}$ для случая\plb униполярного выброса в различные моменты времени: ${t = 25}$ (\emph{слева}) и ${t = 50}$ (\emph{справа})}
\label{fig:kmx:omega1}
\end{figure}

Как следует из рисунка~\ref{fig:kmx:density1}, в случае одностороннего выброса
и тонкого диска ударная волна проходит через околозвездный диск и создает
ударную волну "--- оболочку, "--- в противоположной относительно
начального выброса полусфере. Форма этой оболочки несколько иная
(практически параболическая). Как представляется, некоторые
несимметричные объекты могут возникать именно таким образом.

\begin{figure}[!h]
\centering
\includegraphics[width=\textwidth, keepaspectratio]{hydrojet6}
\caption{Зависимость скорости расширения оболочки от ее размера для случая\plb
униполярного выброса}
\label{fig:kmx:vr}
\end{figure}

Анализ зависимостей скорости расширения оболочки от ее размера
(см. рис.~\ref{fig:kmx:vr}) и размера оболочки от времени ее расширения
(см. рис.~\ref{fig:kmx:rt}) показывает, что общая картина расширения
ударной оболочки может быть условно разбита на четыре характерных этапа.

\begin{enumerate}
\item Первый этап "--- в это время происходит начальный выброс, который
формирует начальную структуру ударной оболочки и разгоняет ее до
скороcти ${\approx 50}$~км/с, при этом оболочка расширяется до размера
${\approx 0.007}$~пк. На этом этапе, кроме того, начинается формирование
узкой вращающейся сверхзвуковой струи внутри оболочки (см. выше).
Длительность этого этапа ${\approx 250}$~лет.

\item Следующим этапом является дальнейшее расширение ударной оболочки
до размера ${\approx 0.023}$~пк. При этом происходит вызванное вторжением
оболочки в невозмущенную окружающую среду постепенное уменьшение скорости
ее расширения до значения ${\approx 12}$~км/с. Начальный выброс
к этому моменту уже полностью завершился, а сформировавшаяся внутри
оболочки сверхзвуковая струя начинает постепенно догонять фронт
ударной оболочки. Длительность данного этапа ${\approx 850}$~лет.

\item К началу этого этапа оболочку догоняет сформировавшаяся на втором
этапе узкая вращающаяся сверхзвуковая струя, которая бьет в ее головную
часть, передавая импульс и энергию, и тем самым ускоряет ее
и вызывает удлинение головной части оболочки. К концу этого этапа
скорость расширения оболочки и достигнутый ею размер составляют
${\approx 37}$~км/с и ${\approx 0.027}$~пк соответственно. Длительность
этапа ${\approx 150}$~лет.

\item Завершающим этапом является дальнейшее расширение ударной
оболочки с уменьшением его скорости до значения ${\approx 10}$~км/с.
\end{enumerate}

\begin{figure}[!h]
\centering
\includegraphics[width=\textwidth, keepaspectratio]{hydrojet7}
\caption{Зависимость размера оболочки от времени ее расширения для случая\plb
униполярного выброса}
\label{fig:kmx:rt}
\end{figure}

Очень важно отметить, что описанный гидродинамический механизм
формирования струи посредством развития последовательности
<<выброс --- тор --- торнадо --- джет>> основан на выполнении закона сохранения
углового момента, поэтому он должен работать для всех
аккреционно-струйных систем. Существование других дополнительных
факторов (например, магнитные поля и т.\,д.) способно
модифицировать этот механизм, но не устранить его.

%*********************************************************************
\section*{Заключение}
\label{sect:kmx:conclusion}
%*********************************************************************


Проведенное нами моделирование позволяет сделать следующие выводы:

\begin{enumerate}
\item Результаты однозначно свидетельствуют о том, что в струю
попадает вещество диска, обладающее угловым моментом, сонаправленным
с осью симметрии \hbox{системы.}

\item Если в силу некоторых причин происходит не биполярный, а
односторонний выброс вещества, то вызванная им ударная волна
проходит через тонкий околозвездный диск и формирует ударную волну
(оболочку) с другой стороны диска.

\item Закономерностью при высоких начальных скоростях выброса
является значительное удлинение головной части оболочки, что
характерно для многих наблюдаемых молодых звездных объектов.

\item Еще одной закономерностью является то, что формирование струи
обусловлено наличием долгоживущего торообразного вихря,
в котором происходит вращение
газа как вдоль средней линии тора, так и вокруг нее.
\end{enumerate}

Авторы признательны А.\,В.~Моисееву за полезные обсуждения.

\begin{thebibliography}{99}
\bibitem{kmx:kuzmin-etal-2002}
\baut{Кузьмин}{Н.~М.}
\baut{Мусцевой}{В.~В.}
\baut{Храпов}{С.~С.}
\btit{Исследование дисперсионных свойств малых возмущений в струйных выбросах из молодых звезд}[The Investigation of Dispersion Properties of Small Perturbations in Jet Outflows from Young Stars]
\bj{Вестник Волгоградского государственного университета. Серия 1, Математика. Физика}
\byr{2002}
\bnum{7}
\bpp{76-94}
\mkpaperr

\bibitem{kmx:kuzmin-etal-2007}
\baut{Кузьмин}{Н.~М.}
\baut{Мусцевой}{В.~В.}
\baut{Храпов}{С.~С.}
\btit{Численное моделирование эволюции неустойчивых мод джетов, выходящих из молодых звездных объектов}[Numerical Modeling of the Evolution of Unstable Modes of Jets from Young Stellar Objects]
\bj{Астрономический журнал}
\byr{2007}
\bvol{84}
\bnum{12}
\bpp{1089-1098}
\mkpaperr

\bibitem{kmx:levin-etal-1999}
\baut{Левин}{К.~А.}
\baut{Мусцевой}{В.~В.}
\baut{Храпов}{С.~С.}
\btit{Диски и джеты вокруг молодых звезд}[Jets and Disks around Young Stars]
\bj{Астрономический журнал}
\byr{1999}
\bvol{76}
\bnum{2}
\bpp{126-135}
\mkpaperr

\bibitem{kmx:hardee-etal-1997}
\baut{Hardee}{P.~E.}
\baut{Stone}{J.~M.}
\btit{The Stability of Radiatively Cooling Jets I. Linear Analysis}
\bj{Astrophysical Journal}
\byr{1997}
\bvol{483}
\bpp{121-335}
\mkpapere

\bibitem{kmx:muscl}
\baut{Leer}{B.~van}[][.]
\btit{Towards the Ultimate Conservative Difference Scheme V. A Second Order Sequel to Godunov's Method}
\bj{Journal of Computational Physics}
\byr{1979}
\bvol{32}
\bnum{1}
\bpp{101--136}
\mkpapere

\bibitem{kmx:little-1994}
\baut{Little}{L.~T.}
\btit{Interstellar Molecular Discs around Young Stars}
\bj{Quarterly Journal of the Royal Astronomical Society}
\byr{1994}
\bvol{35}
\bpp{11-42}
\mkpapere

\bibitem{kmx:macdonald-etal-1981}
\baut{MacDonald}{J.}
\baut{Bailey}{M.~E.}
\btit{The Evolution of Flows of Stellar Mass Loss in Active Galaxies}
\bj{Monthly Notices of the Royal Astronomical Society}
\byr{1981}
\bvol{197}
\bpp{995-1019}
\mkpapere

\bibitem{kmx:mundt-etal-1991}
\baut{Mundt}{R.}
\baut{Ray}{T.~P.}
\baut{Raga}{A.~C.}
\btit{Collimation of Stellar Jets --- Constraints from the Observed Spatial Structure II. Observational Results}
\bj{Astronomy and Astrophysics}
\byr{1991}
\bvol{252}
\bpp{740-761}
\mkpapere

\bibitem{kmx:mundt-1986}
\baut{Mundt}{R.}
\btit{Jets from Young Stars --- Estimates of their Physical Parameters}
\bj{Canadian Journal of Physics}
\byr{1986}
\bvol{64}
\bpp{407-413}
\mkpapere

\bibitem{kmx:mundt-1987}
\baut{Mundt}{R.}
\btit{Jets from Young Stars}
\bj{Mitteilungen der Astronomischen Gesellschaft Hamburg}
\byr{1987}
\bvol{70}
\bpp{100-115}
\mkpapere

\bibitem{kmx:mundt-etal-1987}
\baut{Mundt}{R.}
\baut{Brugel}{E.~W.}
\baut{Buehrke}{T.}
\btit{Jets from Young Stars --- CCD Imaging, Long-Slit Spectroscopy, and Interpretation of Existing Data}
\bj{Astrophysical Journal}
\byr{1987}
\bvol{319}
\bpp{275-303}
\mkpapere

\bibitem{kmx:ouyed-etal-1997}
\baut{Ouyed}{R.}
\baut{Pudritz}{R.~E.}
\btit{Numerical Simulations of Astrophysical Jets from Keplerian Disks I. Stationary Models}
\bj{Astrophysical Journal}
\byr{1997}
\bvol{482}
\bpp{712-732}
\mkpapere

\bibitem{kmx:raga-etal-1991}
\baut{Raga}{A.~C.}
\baut{Mundt}{R.}
\baut{Ray}{T.~P.}
\btit{Collimation of stellar jets --- Constraints from the Observed Spatial Structure I. Data Analysis Methods}
\bj{Astronomy and Astrophysics}
\byr{1991}
\bvol{252}
\bpp{733-739}
\mkpapere
\end{thebibliography}

\begin{summary} \!\!\!The results of numerical hydrodynamical simulation of shock shells
evolution in young star objects are presented. We have shown that during the
expanding process of such shell, a slowly rotating supersonic collimated jet made of accretion disk substance forms inside the shell through the development of consequence ``ejection --- thorus --- tornado --- jet''. The mechanism
of outflow's forming and collimation is only hydrodynamical and
based on the conservation law of angular momentum. It must work
for all accretion-jet systems. Additional factors (such as magnetic fields)
can modify such mechanism, but not eliminate it.

Let's list the main conclusions:
\begin{enumerate}
\item The jet is formed from the substance of the circumstellar disc. It has an angular momentum co-ordinated with the symmetry axis of the system.

\item If a one-sided ejection of matter takes place, then the shock wave, caused by it, passes through a thin circumstellar disk. Then it forms a shock wave (shell) on the other side of the disk.

\item At high initial ejection velocities, there is a significant elongation of the head of the shell. This is typical for many observable young star objects.

\item The formation of the jet is due to the presence of a long-lived torus-like vortex, in which the gas rotates both along the toroidal axis and around it.
\end{enumerate}
\end{summary}

%\end{document}
