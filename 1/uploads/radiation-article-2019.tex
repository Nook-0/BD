\documentclass[a4paper,11pt,twoside]{article}
\usepackage{mpcs}
\usepackage{rumathbr}

\begin{document}
\udk{524.5-7}
\bbk{22.66}

\art[Численное моделирование термической турбулентности в оптически толстых облаках]{Численное моделирование термической турбулентности в оптически толстых облаках межзвездной среды}{Numerical simulation of the thermal turbulence in the optical thick clouds of the interstellar medium}
\support{Работа поддержана грантом РФФИ 18-42-340002 <<Турбулентность в оптически толстых газопылевых облаках межзвёздной среды>>}

\fio{Королев}{Виталий}{Владимирович}
\sdeg{кандидат физико-математических наук}{Candidate of Physical and Mathematical Sciences}
\pos{доцент кафедры теоретической физики и волновых процессов}{Associate Professor, Department of Theoretical Physics and Wave Phenomena}
\emp{Волгоградский государственный университет}{Volgograd State University}
\email{korolev.vv@volsu.ru, vitokorolev@gmail.com}
\addr{просп. Университетский, 100, 400062 г. Волгоград, Российская Федерация\vspace*{-5mm}}{Prosp. Universitetsky, 100, 400062 Volgograd, Russian Federation}

\fio{Еремин}{Михаил}{Анатольевич}
\sdeg{кандидат физико-математических наук}{Candidate of Physical and Mathematical Sciences}
\pos{доцент кафедры теоретической физики и волновых процессов}{Associate Professor, Department of Theoretical Physics and Wave Phenomena}
\emp{Волгоградский государственный университет}{Volgograd State University}
\email{eremin@volsu.ru, ereminmikhail@gmail.com}
\addr{просп. Университетский, 100, 400062 г. Волгоград, Российская Федерация\vspace*{-5mm}}{Prosp. Universitetsky, 100, 400062 Volgograd, Russian Federation}

\fio{Жукова}{Елена}{Владимировна}
%\sdeg{кандидат физико-математических наук}{Candidate of Physical and Mathematical Sciences}
\pos{научный сотрудник кафедры теоретической физики и волновых процессов}{Junior Researcher, Department of Theoretical Physics and Wave Phenomena}
\emp{Волгоградский государственный университет}{Volgograd State University}
\email{tf@volsu.ru, zhu4ok88@mail.ru}
\addr{просп. Университетский, 100, 400062 г. Волгоград, Российская Федерация\vspace*{-5mm}}{Prosp. Universitetsky, 100, 400062 Volgograd, Russian Federation}

\fio{Коваленко}{Илья}{Геннадьевич}
\sdeg{доктор физико-математических наук}{Doctor of Physical and Mathematical Sciences}
\pos{профессор кафедры теоретической физики и волновых процессов}{Professor, Department of Theoretical Physics and Wave Phenomena}
\emp{Волгоградский государственный университет}{Volgograd State University}
\email{i.kovalenko@volsu.ru, ilya.g.Kovalenko@gmail.com}
\addr{просп. Университетский, 100, 400062 г. Волгоград, Российская Федерация\vspace*{-5mm}}{Prosp. Universitetsky, 100, 400062 Volgograd, Russian Federation}

\fio{Занкович}{Андрей}{Михайлович}
%\sdeg{кандидат физико-математических наук}{Candidate of Physical and Mathematical Sciences}
\pos{научный сотрудник кафедры теоретической физики и волновых процессов}{Junior Researcher, Department of Theoretical Physics and Wave Phenomena}
\emp{Волгоградский государственный университет}{Volgograd State University}
\email{tf@volsu.ru, zed81@list.ru}
\addr{просп. Университетский, 100, 400062 г. Волгоград, Российская Федерация}{Prosp. Universitetsky, 100, 400062 Volgograd, Russian Federation}

\maketitle

\begin{abstract}
В работе представлены результаты численного моделирования развивающейся в протозвёздном облаке конвективной неустойчивости в рамках двухмерной самосогласованной оптико-гидродинамической модели турбулентного газопылевого облака, учитывающей движение среды под действием собственной гравитации и радиационного давления. Показано, что при характерных для наблюдаемых диффузных облаков межзвёздной среды параметрах облако переходит в конвективно неустойчивое состояние с инверсным распределением концентрации и температуры. Конвекция возникает в этом инверсном слое в виде маломасштабных вихрей. Со временем размеры вихрей достигают $\sim (0.05-0.1)\lambda_J$, а течение приобретает турбулентный характер по всей толще облака. Развивающаяся в облаке турбулентность является трансзвуковой со скоростями $\sim 600$ м/с и числами Маха до $1.2$, однако сверхзвуковые области занимают малую часть объёма облака.
\end{abstract}

\keywords{межзвёздная среда, диффузные облака, перенос излучения, турбулентность, конвекция}{interstellar medium, diffuse clouds, radiation transfer, turbulence, convection}

\section*{Введение}

Как показывают наблюдения, межзвёздная газопылевая среда, включая диффузные и молекулярные облака и протозвёздные туманности, сильно турбулизована \cite{Elmegreen2004,Snow2006}. Однако в вопросе о физических механизмах, ответственных за возбуждение и долговременное поддержание трансзвуковой турбулентности, до сих нет полной ясности. В числе таких процессов рассматривают аккрецию, струйные течения, гидро- и магнитогидродинамические неустойчивости, ударные волны, звёздный ветер и радиационные механизмы.

В работе \cite{Zhukova2012} было показано, что при наличии постоянно действующего источника подогрева в центральной части самогравитирующего газопылевого облака распределение газа в нём приобретает стратифицированный характер. Под действием ультрафиолетового излучения, для которого облако является оптически толстой средой, вещество выметается световым давлением из центральной части облака. На периферии вещество снова охлаждается, высвечивая тепло в инфракрасном диапазоне, в котором облако оптически прозрачно. В результате в облаке реализуется стационарное распределение оболочечного типа: в центральной части расположен горячий разреженный газ, затем область с плотным холодным газом, и далее на периферии газ с асимптотически спадающими к краю концентрацией и температурой. Такое инверсное распределение вещества в облаке является неустойчивым. Если локальные нарушения баланса сил давления газа и излучения, с одной стороны, и собственного тяготения, с другой, вызовут конвекцию, то с течением времени в облаке может развиться и глобальная турбулентность.

%Такое инверсное распределение вещества в облаке, удерживаемое за счёт сил давления газа и излучения от сжатия под действием собственного тяготения, является конвективно неустойчивым. 
%Например, обычные для молодых звёзд  вариации излучения могут послужить причиной появления конвективных движений и, как следствие, турбулентности.

Связанная с неоднородным прогревом облаков термическая турбулентность, поддерживаемая за счёт источников излучения, исследована слабо.
Сценарии поддержания равновесия межзвёздного газа или развития неустойчивостей под влиянием излучения на масштабах от нескольких парсек до килопарсек предлагались во многих работах \cite{Bianci2005, Ferrara1993, Murray2010}, но, как правило, анализ носил полукачественный характер без расчёта детальной структуры течения. В работах \cite{Freytag2009,Heofner1995,Kreuger1994,Kreuger1997,Mastrodemos1996,Ochsendorf2014,Ochsendorf2014-2,Simis2001,Sorrell2002,Woitke2006} рассматривались газопылевые течения масштаба парсеков в эмиссионных туманностях и газопылевых оболочках звёзд, однако в них не учитывалась самогравитация.

Целью данной работы является исследование конвективно неустойчивых газопылевых облаков, поддерживаемых в стратифицированном состоянии источниками излучения и самогравитацией, и моделирование развития турбулентности по сценарию, предложенному в работах \cite{Zhukova2012,Zhukova2015}.

\section{Физическая модель и основные уравнения}

%Чтобы сверхзвуковая турбулентность возникала в облаке <<самостоятельно>> вследствие развития физических неустойчивостей, необходимо построить опорную оптико-динамическую модель самогравитирующей оптически плотной газопылевой среды. Поскольку в уравнениях переноса присутствуют дополнительные размерности –- направления распространения излучения и частоты, расчёт переноса излучения в многомерном конфигурационном пространстве представляет собой сверхгромоздкую задачу, требующую  адекватного упрощения.

В основном мы отталкиваемся от идей и моделей работы \cite{Zhukova2012}, предполагая, что турбулентность развивается благодаря внутренним механизмам, самогравитации и источникам излучения в облаке, без участия магнитных полей. При этом задача рассматривается в рамках двумерного приближения и не предполагает стационарности.
%При этом полагая, что источник излучения не точечный, а протяжённый и располагается вдоль оси $x$ (рисунок ref ???).
Кратко перечислим основные допущения, использованные в работе.

Облако моделируется как бесконечно протяжённый неоднородный газопылевой слой, оптически непрозрачный и зеркально-симметричный относительно экваториальной плоскости $z = 0$ (рис. \ref{pic_scheme}). Вдоль этой плоскости расположены звёзды -- источники излучения, их пространственное распределение однородно, а излучение изотропно. Это позволяет ограничить построение решения только в одной из половин облака ($z > 0$).
%{\color{red}Кроме того, при таком расположении источников можно пользоваться упрощённой одномерной моделью переноса излучения, полагая в первом приближении, что динамика среды главным образом зависит от переноса излучения в направлении $z$.}

\begin{figure}[t]
\includegraphics[width=0.7\linewidth]{scheme.png}
\caption{Схема постановки задачи: пунктирными линиями условно показано предполагаемое конвективное движение среды, серыми звёздочками -- расположение источников излучения, вертикальными стрелками -- направление излучения.}
\label{pic_scheme}
\end{figure}

Пылевую компоненту облака предполагаем менее массивной в сравнении с газовой и считаем, что скорости частиц пыли подстраиваются к скорости течения газа за времена, существенно меньшие динамических времён эволюции облака.
%{\color{red} cослаться на наблюдения, по наблюдениям пыли мало даже ГМО}.
Фактически это означает, что пыль вморожена в газ. Поэтому среду облака <<газ + пыль>> далее можно характеризовать общими значениями плотности $\rho$, концентрации $n$, скорости $\mathbf{u} = \{u_x, u_z\}$, температуры $T$ и давления $p$.

Облако будем считать состоящим из водорода в атомарной форме (H\,I), не учитывая молекулярного водорода, гелия и других элементов, содержание которых в в газовой фазе мало \cite{Snow2006}. При характерных для облаков H\,I значениях плотностей можно рассматривать газопылевую среду как сплошную, а сам газ как идеальный c показателем адиабаты $\gamma = 5/3$ и уравнением состояния в форме
\begin{equation}\label{eq_gas_5}
p = \frac{\rho k_{B} T}{m_0},
\end{equation}
где $k_{B}$ -- постоянная Больцмана, $m_0$ -- масса атома водорода.

Облако предполагается существенно более массивным по сравнению
с излучающими звёздами, поэтому в модели учитывается только собственная гравитация газопылевой среды. Силы тяготения $\mathbf{f}$ определяются по гравитационному потенциалу $\Phi(x, z)$, который находится из уравнения Пуассона
\begin{equation}\label{eq_grav_1}
\Delta\Phi = 4\pi G \rho,
\end{equation}
\begin{equation}\label{eq_grav_2}
\mathbf{f} = -\nabla\Phi,
\end{equation}
где $G$ -- гравитационная постоянная.

В модели переноса излучения в газопылевом облаке используется двухканальное приближение. Будем считать, что максимум излучения в спектре звёзд приходится на ультрафиолетовый диапазон $0.1-0.5$ мкм. Так как для этих длин волн пылевая компонента среды имеет большую оптическую толщу, энергия проходящих через неё квантов частично передаётся пылинкам и, как следствие, газу через давление излучения, преобразуясь в кинетическую энергию среды, а частично высвечивается пылью в виде инфракрасного излучения на длинах волн $\sim 100$ мкм, для которых среда является прозрачной.

Рассеяние проходящего через среду излучения в каждом малом объёме среды должно происходить в разных направлениях. Однако моделирование многомерного переноса излучения в неоднородной среде является крайне ресурсоёмкой процедурой.
%Для её упрощения мы далее будем предполагать, что кванты распространяются преимущественно параллельно оси $z$, а доля рассеяных в поперечном направлении квантов существенно меньше. Тогда перенос УФ-излучения в среде можно рассматривать как одномерный для восходящего и нисходящего потоков диффузного излучения: в каждом газопылевом столбе, ориентированном вдоль оси $z$, излучение взаимодействует со слоями среды, рассеиваясь либо в положительном, либо в отрицательном направлении $z$.
Для существенного упрощения алгоритма вычислений будем рассматривать одномерную модель переноса для восходящего и нисходящего вдоль оси $z$ потоков диффузного излучения. Такой подход означает, что в каждом столбе газопылевой среды шириной в одну ячейку расчётной области и ориентированном вдоль $z$, излучение, частично поглощаясь или рассеиваясь, взаимодействует со слоями среды только в своём столбе независимо от соседних.

%Фактически такой подход позволяет использовать одномерную двухканальную модель переноса излучения, предложенную ранее в работах cite //Жукова Е. В., Занкович А. М., Коваленко И. Г., Фирсов К. М., Гидростатическая модель самогравитирующего оптичски плотного межзвездного облака // Вестник Волгоградского государственного университета. Серия 1. Математика. Физика Вып. 1(16), 57-73 (2012).

%Характерные масштабы $L$ межзвездного облака -- парсеки или десятки парсек. %Длину свободного пробега атомов и молекул $l$ в межзвездном облаке можно оценить как $l\approx 1/n\sigma_H$, где $\sigma_H$ -- сечение взаимодействия атомов или молекул водорода.
%Беря для оценки сверху атомарный водород, получаем $l\sim 3\cdot 10^{12}\div 3\cdot 10^{16}$ см, что на два-шесть порядков меньше минимального размера облака $L$.
%Это означает, что в одной элементарной ячейке размером $r^3$ ($l \ll r\ll L$) содержится большое число частиц.
%Следовательно, вещество в облаке  на масштабах задачи можно рассматривать как сплошную среду.

%Если газ находится в гидростатическом равновесии, то в  неподвижном облаке %весу вышележащих слоев противостоят сила трения о пыль и давление газа.

В итоге система уравнений радиативной газовой динамики, описывающая движение среды с учётом перечисленных выше факторов и допущений, запишется в виде:
\begin{equation}\label{eq_gas_1}
\frac{\partial{\rho}}{\partial{t}} + \nabla{(\rho \mathbf{u})}=0,
\end{equation}
\begin{equation}\label{eq_gas_2}
\frac{\partial{(\rho \mathbf{u})}}{\partial{t}} + \nabla{(\rho \mathbf{u} \otimes \mathbf{u} + p)} = \rho\mathbf{f} + \frac{4\pi}{c}\rho \alpha_{\nu} \mathbf{H_{\nu}},
\end{equation}
\begin{equation}\label{eq_gas_3}
\frac{\partial{E}}{\partial {t}} + \nabla((E + p)\mathbf{u}) = \rho\mathbf{f}\mathbf{u} + 4\pi\rho(k_{UV}J_\nu - k_{IR}\sigma_{SB} T^4) + \frac{4\pi}{c}\rho \alpha_{\nu}\mathbf{H_\nu}\mathbf{u}.
\end{equation}
%\begin{equation}\label{eq_gas_4}
%E=\frac{\rho u^{2}}{2} + \frac{p}{\gamma - 1}.
%\end{equation}
Здесь $E = \rho u^{2}/2 + p/(\gamma - 1)$ -- полная энергия единицы объёма среды,
$\mathbf{H_{\nu}} = H_{\nu}(z)\mathbf{e}_z$ -- полный поток излучения,
$J_\nu = J_\nu(z)$ -- средняя интенсивность излучения,
$\alpha_\nu$ -- непрозрачность среды,
$k_{UV}$ -- коэффициент поглощения среды для УФ-излучения,
$k_{IR}$ -- коэффициент поглощения среды для ИК-излучения,
$\sigma_{SB}$ -- постоянная Стефана-Больцмана,
$c$ -- скорость света.

Интенсивность диффузного излучения $I_\nu = I_\nu(\mu; z)$ определяется из уравнения переноса излучения:
\begin{equation}\label{eq_rad_11}
\mu\frac{d I_{\nu}}{d z} = -\alpha_{\nu} I_{\nu} + \frac{\sigma_{\nu}}{2} \int\limits_{-1}^{1} I_{\nu}(\mu,\mu';z)p(\mu,\mu') d\mu' + \alpha_{\nu}\varepsilon_{\nu},
%\frac{d I}{d z} = - I + \frac{\omega}{2} \int\limits_{-1}^{1}
%I(\mu,\mu'; z) p(\mu,\mu') d\mu',
\end{equation}
\begin{equation} \label{eq_rad_2}
\varepsilon_{\nu}(\tau)=
\frac{1}{2}F_0\delta(\mu-\mu_0)e^{-\tau/\mu_0}, \qquad \tau \equiv \tau(z) = \int_0^z{\alpha_{\nu}(z)dz},
\end{equation}
а функции $J_\nu(z)$ и $H_\nu(z)$ вычисляются, соответственно, как нулевой и первый моменты функции $I_\nu(\mu; z)$:
\begin{equation}\label{eq_rad_3}
J_\nu(z) = \int\limits_{-1}^{1} I_\nu(\mu; z) d\mu, \qquad H_\nu(z) = \int\limits_{-1}^{1} I_\nu(\mu; z)\mu d\mu.
\end{equation}
Здесь $\varepsilon_{\nu}(\tau)$ -- коэффициент  излучения, учитывающий излучение звёзд и зависящий от плотности источников УФ-излучения $F_0$,
$\tau$ -- оптическая толща, $p(\mu,\mu')$ -- индикатриса, описывающая рассеяние излучения, поступившего из направления  $\mu'=\cos\vartheta'$, в направлении $\mu=\cos\vartheta$, $\mu_0 = 1$ -- косинус угла между осью $z$ и направлением эмиссии фотонов источника, $\delta(\mu)$ -- $\delta$-функция Дирака. %$\omega$ -- альбедо однократного рассеяния,

В расчётах использовалась модельная индикатриса Хеньи-Гринстейна \cite{Briegleb2007}:
\begin{equation} \label{eq_rad_4}
p(\cos\vartheta) = \frac{1-g_{HG}^2}{(1 + g_{HG}^2 - 2g_{HG}\cos\vartheta)^{3/2}},
\end{equation}
где параметр $0\le |g_{HG}| \le 1$ характеризует степень вытянутости индикатрисы. В расчётах он принят равным $g_{HG} = 0.6$.

\section{Параметры моделей и методы решения}

В начальном состоянии распределение вещества предполагалось изотермическим (с начальной температурой $T_0 = 10$ К), однородным вдоль направления $x$ и стационарным, удерживаемым в равновесии давлением и собственным тяготением вдоль направления $z$. Значения плотности и давления при этом определяются интегрированием системы обыкновенных дифференциальных уравнений:
\begin{equation}
\frac{d \Phi}{d z} = 4\pi G m_0 n,
\end{equation}
\begin{equation}
k_B \frac{d n}{d z} = -n \frac{d \Phi}{d z},
\end{equation}
которая получается из уравнений \eqref{eq_grav_1}--\eqref{eq_gas_3} при условии стационарности и отсутствия в начальный момент источников излучения. В качестве начального условия использовалось значение концентрации в плоскости $z=0$ $n(x, z=0) = n_0$. В разных моделях значение параметра $n_0$ выбиралось в пределах $10^2-10^5$ см$^{-3}$, что соответствует характерным значениями концентрации газа в облаках межзвёздной среды. Характерное время задачи определяется величиной $t_0 = \lambda_J/c_{s0}$, где $\lambda_J$ -- джинсовский масштаб, а $c_{s0}$ -- адиабатическая скорость звука в плоскости $z = 0$ в начальном состоянии. При $n_0 = 10^3$ см$^{-3}$ и $T_0 = 10$ К эти параметры равны
$t_0 \approx 1.6\cdot10^6$ лет и $\lambda_J \approx 0.8$ пк.

Для приходящего от источников в ближнем ультрафиолетовом диапазоне излучения, нагревающего среду, была принята длина волны $\lambda_{UV} = 2 \cdot 10^{-5}$ см, на которой излучение эффективно взаимодействует с пылью. Собственное инфракрасное излучение пылинок имеет непрерывный спектр с характерной длиной волны $\lambda_{IR} = 10^{-2}$ см, что соответствует температуре частиц в диапазоне $T_d = 10-40$ К. Поток излучения $F_0$ от источников на границе $z=0$ подбирался таким образом, чтобы вклад слагаемого с давлением излучения был сопоставим с начальным давлением газа, но не настолько велик, чтобы стать доминирующим. Типичное в наших расчётах значение $F_0$ в плоскости источников $z=0$ в пересчёте на размерные величины даёт светимость порядка $10^4 L_\odot$ на площадку в 1 пк$^2$.

Расчёт коэффицентов поглощения, ослабления и рассеяния в газопылевой среде производился согласно теории Ми \cite{Hulst1961}. При этом мы полагали аналогично работе \cite{Zhukova2012}, что пылинки ведут себя как пассивная примесь одинаковых сферических частицами из графита с диаметром $d_g = 10^{-5}$ см и плотностью материала $\rho_g = 2.23$~г/см$^3$. Комплексный показатель преломления для графитовой пылинки на разных длинах волн принимался равным $m_{UV} = 2.1 + 1.5i$, $m_{IR} = 10.39 + 9.592 i$  \cite{Michel1996}, действительная часть этих параметров характеризует рассеяние, мнимая -- поглощение.

Для решения уравнений \eqref{eq_grav_1}--\eqref{eq_rad_11} была использована численная схема с расщеплением по физическим процессам \cite{Koven1981,Marchuk1988}. Оригинальная расчётная программа включает три распараллеленных модуля, отвечающих за решение основных подзадач: расчёт динамики течения в заданном поле сил, вычисление гравитационного потенциала по известному распределению вещества и моделирование переноса излучения в газопылевой среде.  Гидродинамическая часть программы, аналогичная коду AstroChemHydro \cite{Eremin2012}, была реализована на основе схемы MUSCL TVD \cite{Kolgan1972,Toro1999, vanleer1, vanleer2} третьего порядка точности по пространству и второго по времени. Решение уравненупрощения алгоритма вычислений будем рассматривать одномерную модель переноса для восходящего Рё нисходящего вдоль РѕСЃРё $z$ потоков диффузного излучения. Такой РїРѕРґС…РѕРґ означает, что РІ каждом столбе газопылевой среды шириной РІ РѕРґРЅСѓ ячейку расчётной области Рё ориентированном вдоль $z$, излучение, частично поглощаясь или рассеиваясь, взаимодействует СЃРѕ слоями среды только РІ своём столбе независимо РѕС‚ соседних.

%Фактически такой РїРѕРґС…РѕРґ позволяет использовать одномерную двухканальную модель переноса излучения, предложенную ранее РІ работах cite //Р–СѓРєРѕРІР° Р•. Р’., Занкович Рђ. Рњ., Коваленко Р