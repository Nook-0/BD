\documentclass[a4paper,11pt,twoside]{article}
\usepackage{vvmph2} 

%\documentclass[a4paper,11pt,twoside]{article}
%\usepackage{mpcs}
%\usepackage{rumathbr}


\begin{document}

\udk{519.632.4}
\bbk{22.19}

\art{Исследования в области геометрического анализа в Волгоградском государственном университете}{Research in the field of geometric analysis at Volgograd State University}
\support{Работа выполнена при поддержке Математического Центра в Академгородке, соглашение с Министерством науки и высшего образования Российской Федерации номер 075-15-2019-1613.}
\fio{Клячин}{Алексей}{Александрович}
\sdeg{доктор физико-математических наук}{Doctor of Physical and Mathematical Sciences}
\pos{заведующий кафедрой математического анализа и теории функций}{Head of Department of Mathematical Analysis and Function Theory}
\emp{Волгоградский государственный университет}{Volgograd State University}
\email{Aleksey.klyachin@volsu.ru, klyachin-aa@yandex.ru}
\addr{Просп. Университетский, 100, 400062 г. Волгоград, Российская Федерация}{Prosp. Universitetsky, 100, 400062 Volgograd, Russian Federation}

\fio{Клячин}{Владимир}{Александрович}
%\sdeg{}{}
\sdeg{доктор физико-математических наук}{Doctor of Physical and Mathematical Sciences}
\pos{заведующий кафедрой компьютерных наук и экспериментальной математики}{Head of the Department of Computer Science and Experimental Mathematics}
\emp{Волгоградский государственный университет}{Volgograd State University}
\email{klyachin.va@volsu.ru, klchnv@mail.ru}
\addr{Просп. Университетский, 100, 400062 г. Волгоград, Российская Федерация}{Prosp. Universitetsky, 100, 400062 Volgograd, Russian Federation}

\maketitle

\begin{abstract}
В настоящей статье рассмотрены основные направления исследований по геометрическому анализу, которые проводились и проводятся научной математичексой школой Волгоградского государственного унверситета.
Вкратце изложены результаты основоположника нучной школы доктора физико-математических наук, профессора Владимира Михайловича Миклюкова и его учеников. Эти результаты касаются решения ряда задач в области квазиконформных плоских отображений и отображений с ограниченным искажением поверхностей и римановых многообразий, теории минимальных поверхностей и поверхностей предписанной средней кривизны, поверхностей нулевой средней кривизны в лоренцевых пространствах, а также, задач, связанных с исследованием устойчивости такого рода поверхностей. Кроме этого, отмечеы результаты изучения различных классов триангуляций -- объекта, возникающего на стыке исследований в оласт геометрического анализа и вычислительной математики.
Кроме этого, в данном обзоре рассматриваются работы, в которых дано  применение  метода Фурье разложения 
решений уравнений Лапласа-Бельтрами 
и стационарного уравнения Шредингера  по собственным функциям соответствующих краевых задач. В
частности, приведены результаты
по нахождению емкостных характеристик, которые позволили впервые сформулировать 
и доказать критерии выполнения различных теорем типа Лиувилля и разрешимости краевых задач на 
модельных и квазимодельных римановых многообразиях. Также указывается роль метода эквивалентных функций 
при исследовании подобных задач на многообразиях достаточно общего вида.

В данной статье помимо этого дается обзор результатов, касающихся оценок погрешности вычисления
 интегральных функционалов и сходимостью кусочно-полиномиальных решений нелинейных
 уравнений вариационного типа: уравнения минимальной поверхности, уравнения равновесной
 капиллярной поверхности и уравнения бигармонических функций. 
\end{abstract}

\keywords{геометрический анализ, минимальные поверхности, емкость, гармонические функции, триангуляция, интегральный функционал}{geometric analysis, minimal surfaces, capacity, harmonic functions, triangulation, integral functional}




Одним из направлений развития математики в Волгоградском государственном университете стали исследования в области геометрического анализа -- области математики, возникшей  на стыке геометрии, математического анализа и дифференциальных уравнений. Здесь тесно переплелись методы квазиконформных отображений, задачи геометрического строения минимальных поверхностей и исследования асимптотического поведения решений эллиптических уравнений в неограниченных областях или в окрестности особой точки. Самую важную роль в становлении геометрического анализа в ВолГУ сыграли исследования  доктора физико-математических наук, профессора Владимира Михайловича Миклюкова. Разработанные им методы нелинейной емкости позволили решить широкий круг задач,  связанных с поведением решений нелинейных уравнений в пространстве и на многообразиях \cite{32}, \cite{37}, \cite{38}, \cite{45} -- \cite{49}, \cite{51} -- \cite{54}. Так же возник новый объем задач, который  включал в себя вопросы протяженности минимальных трубок и лент в евклидовом и псевдоевклидовом пространствах, проблемы конформного типа поверхностей, исследования пространственно подобных трубок и лент нулевой средней кривизны, их устойчивости/неустойчивости при малых деформациях, времени существования, точек ветвлений, связей между точками ветвления и лоренцево инвариантными характеристиками поверхностей \cite{34}, \cite{36}, \cite{41} -- \cite{44}, \cite{50}, \cite{55}, \cite{56}. Сюда же мы относим изучение теорем типа Фрагмена-Линделефа для дифференциальных форм, теорем типа Альфорса для дифференциальных форм с конечным/бесконечным числом различных асимптотических трактов; теорем типа теоремы Вимана для квазирегулярных отображений многообразий; применение изопериметрических методов в принципе Фрагмена-Линделефа для квазирегулярных отображений многообразий \cite{33}, \cite{35}, \cite{39}, \cite{40}, \cite{57}.

	Под руководством В.М.Миклюкова получила развитие теория минимальных поверхностей и поверхностей нулевой средней кривизны в псевдоевклидовых пространствах  благодаря применению таких современных методов математического анализа как емкостная и модульная техника. Усовершенствованная емкостная и модульная техника позволила ему решить ряд задач нелинейного анализа в его докторской диссертации.  В частности, им получены значительные результаты о граничных свойствах непараметрических минимальных поверхностей, был дан новый подход к проблеме Бернштейна для уравнения типа минимальной поверхности. Фундаментальным результатом явился результат о суммарном топологическом индексе критических точек для решений уравнения типа минимальной поверхности.
	
	Исследования В.М.Миклюкова, касающиеся строения трубок и лент нулевой средней кривизны в евклидовом пространстве и в  пространстве Минковского содержат такие результаты, как теоремы типа Фрагмена-Линделефа, теоремы Лиувилля на поверхностях нулевой средней кривизны и их приложения к геометрическому строению этих поверхностей. В этих работах нашло, в частности,  развитие теории простых концов на случай поверхностей нулевой средней кривизны в евклидовом и псевдоевклидовом пространствах. В работах В.М.Миклюкова найдены общие подходы к изучению асимптотического поведения решений нелинейных уравнений эллиптического типа, как в окрестности конечной точки, так и в бесконечно удаленной. Разработанными методами им получены различные свойства решений таких уравнений (неравенство Гарнака, принцип Сен-Венана, непрерывность по Гельдеру и др.) В последнее время эти результаты В.М.Миклюковым были распространены на случай почти-решений эллиптических уравнений и систем. В частности, им получено неравенство Гарнака. Найденные им новые подходы к исследованию асимптотического поведения решений эллиптических уравнений были успешно применены его учениками, В.Г. Ткачевым, А.А.Клячиным, В.А. Клячиным в изучении строения минимальных, максимальных поверхностей и поверхностей предписанной средней кривизны  \cite{8}, \cite{10} -- \cite{12}, \cite{14} -- \cite{16}, \cite{19}, \cite{20}, \cite{23}, \cite{58}, \cite{59}, \cite{63}. В основе применения этих методов лежит простой факт, что координатные функции погруженных поверхностей предписанной средней кривизны удовлетворяют эллиптическим дифференциальным уравнениям в метрике поверхности. В ряде работ были получены оценки протяженности минимальных трубок не только для случая гиперповерхностей, но и для случая поверхностей произвольной коразмерности. Одним из замечательных следствий такой оценки является тот факт, что  верхняя оценка достигается только на поверхностях вращения. В случае поверхностей коразмерности больше 1 также полностью описан класс таких поверхностей. Эти результаты также были распространены на случай максимальных поверхностей в пространстве-времени Минковского. Отметим, что в этом пространстве решения уравнения максимальных поверхностей могут иметь изолированные особые точки. Исследованию вопросов существования и единственности решений с заданными особенностями были посвящены работы \cite{dop3}, \cite{dop4}.
	
	Отметим еще одно направление исследований в рамках геометрического анализа, выполненное В.М. Миклюковым и его учениками в Волгоградском государственном университете. Это направление связано с изучением признаков устойчивости и неустойчивости минимальных поверхностей, поверхностей предписанной средней кривизны и максимальных поверхностей в пространствах Лоренца \cite{7}, \cite{13}, \cite{18}, \cite{21}, \cite{22}, \cite{31}. Проблема устойчивости такого рода поверхностей вытекает из их вариационного происхождения. Важным моментом здесь является тот факт, что в природе могут существовать только устойчивые решения.  В ряде работ были получены признаки и свойства устойчивых не только классов поверхностей указанных выше, но и классов поверхностей, являющихся экстремалями более общих функционалов.
	
	С начала 2000-х годов в Волгоградском государственном университете начинаются исследования в области вычислительной геометрии и, в частности, объектом исследования становятся различные классы триангуляций \cite{6}, \cite{9}, \cite{17}, \cite{24}. Метод триангуляций находит широкое применение в различных областях: численные методы, 3D  моделирование и программирование,  машинное обучение и многое другое. Одной из задач в методе триангуляций является задача построения триангуляции с наперед заданными геометрическими характеристиками, такими как минимальный угол в треугольнике, максимальный радиус описанной окружности и т. п. Хорошо известен факт, что наилучшие значения указанных величин достигаются на классе триангуляций Делоне. В работах школы геометрического анализа в ВолГУ было дано строгое обоснование использования триангуляции Делоне в задачах аппроксимации дифференцируемых функций, заданных на нерегулярном множестве точек. А так же были построены соответствующие примеры, объясняющие, почему эти результаты не могут быть распространены на многомерный случай. Кроме этого, в работах ученых ИМИТ ВолГУ был описан класс функционалов и им соответствующий  класс триангуляций, на которых функционалы достигают наименьшего значения.  Одним из таких функционалов является сумма площадей охватывающих выпуклых множеств треугольников триангуляций. Доказан следующий результат. Минимальная сумма таких площадей достигается на триангуляции, удовлетворяющей аналогу свойству Делоне: каждое охватывающее множество треугольника триангуляции не содержит точек заданного конечного множества \cite{24}. Это свойство обобщает свойство классической триангуляции Делоне на более широкий класс триангуляций, включающий класс регулярных триангуляций (триангуляций, которые являются проекциями выпуклых многогранников). В настоящее время группа математиков ВолГУ работает над проблемами, связанными с исследованиями структуры множества данного класса триангуляций и решений соответствующих вариационных задач.
  
Отметим еще одно направление геометрического анализа, в котором работает ряд сотрудников ИМИТ ВолГУ.  На протяжении последних двух десятилетий ими наработана достаточно мощная и специфическая емкостная техника проведения исследований в теории дифференциальных уравнений на некомпактных римановых многообразиях, получен  ряд важных и интересных утверждений. Результаты исследований неоднократно докладывались на международных и всероссийских конференциях. Отметим, что разработанные методы позволили получить решение целого ряда проблем качественной теории однородных линейных и квазилинейных эллиптических уравнений и неравенств на некомпактных римановых многообразиях. Так, активное применение  метода Фурье разложения решений уравнений Лапласа-Бельтрами и стационарного уравнения Шредингера  по собственным функциям соответствующих краевых задач, способствовало нахождению емкостных характеристик, которые позволили впервые сформулировать и доказать критерии выполнения различных теорем типа Лиувилля и разрешимости краевых задач на модельных и квазимодельных многообразиях.  
	Проблеме разрешимости задачи Дирихле о восстановлении решений уравнений на некомпактных римановых многообразиях по граничным данным на <<бесконечности>> посвящены работы ряда математиков.  Для некоторых римановых многообразий (например, модельных  и  квазимодельных) существует естественная геометрическая компактификация, позволяющая поставить задачу Дирихле в ее классическом варианте. На таких многообразиях Лосевым А.Г. и Мазепой Е.А. были введены характеристики емкостного типа, которые позволили  также впервые найти критерии восстановления на них решений различных эллиптических уравнений по непрерывным граничным данным на <<бесконечности>>.
	Следует отметить, что  на произвольном некомпактном многообразии даже сама постановка  задачи Дирихле о восстановлении решений уравнений по граничным данным на <<бесконечности>>  достаточно проблематична. В последние годы в работах Мазепы Е.А. наметился новый подход к постановке краевых задач на произвольных некомпактных римановых многообразиях, основанный на введении понятия классов эквивалентных на многообразии непрерывных ограниченных функций. Удалось получить ряд интересных результатов о взаимосвязи разрешимости краевых и внешне краевых задач, об устойчивости решений при вариациях некоторых коэффициентов для линейных эллиптических уравнений. В дальнейшем, Мазепой Е.А.  данная методика была успешно применена для изучения решений различных краевых задач для некоторых полулинейных  и квазилинейных эллиптических уравнений \cite{27}, \cite{28}. Кроме того, были изучены вопросы об устойчивости теорем типа Лиувилля и об устойчивости разрешимости различных краевых задач (в терминах эквивалентных функций) при вариациях нелинейной части рассматриваемых уравнений.
	Метод эквивалентных функций оказался тесно связан с аппроксимативным подходом построения обобщенных решений краевых задач для эллиптических уравнений, описанный в ранних работах Келдыша М.В. и Ландиса Е.М. Данный подход, основанный на методе разметания (Пуанкаре, 19 в.), позволяет рассматривать решения задачи Дирихле без каких-либо ограничений на области, в которых она решается, оставаясь, при этом, в классических предположениях относительно непрерывности граничных данных. В работах Келдыша М.В. (1941) и Ландиса Е.М. (1971) данный подход был применен для изучения проблемы разрешимости задачи Дирихле и устойчивости ее решения в ограниченных областях, граница которых может иметь сложное геометрическое строение.  Аналогичный подход позволил Лосеву А.Г. и Филатову В.В. \cite{25} установить  взаимосвязь между существованием нетривиальных ограниченных и положительных решений стационарного уравнения Шредингера с конечным интегралом энергии. Мазепой Е.А. данный подход был распространен для построения обобщенного (по Келдышу) решения задачи Дирихле и других краевых задач с граничными данными на <<бесконечности>> для произвольных некомпактных римановых многообразий и доказана теорема единственности для рассматриваемых задач \cite{29}. В последнее десятилетие нашими сотрудниками  метод эквивалентных функций стал активно применяться для изучения поведения решений различных краевых задач для однородных линейных уравнений в неограниченных областях многообразий с некомпактным краем.  Светловым А.В. в соавторстве был получен ряд результатов об однозначной разрешимости рассматриваемых краевых задач \cite{62}.
	Другое направление исследований, в котором были получены новые результаты  в  последнее десятилетие, связано с  изучением  решений некоторых нелинейных эллиптических неравенств.  Лосевым А.Г. и Мазепой Е.А. были  получены условия отсутствия нетривиальных решений, а также условия существования и мощность множества положительных решений эллиптических дифференциальных  неравенств вида   на модельных римановых многообразиях и римановых произведениях \cite{30}, \cite{26}.
	Последние результаты обобщают аналогичные утверждения, полученные ранее в работах Naito. Y. и Usami H. (1997) для евклидова пространства. Аналогичные результаты на модельных и квазимодельных римановых многообразиях были получены Лосевым А.Г. и Вихаревым С.С. для стационарного уравнения Гинзбурга-Ландау и дифференциального неравенства специального вида \cite{2}. 
Отметим, что изучение различных свойств триангуляций и треугольных сеток областей, привело еще к одному типу задач, связанных с оценками вычисления интегральных функционалов и сходимостью кусочно-полиномиальных решений нелинейных уравнений вариационного типа. Некоторые задачи, возникающие при проектировании архитектурных сооружений, сводятся к построению поверхностей минимальной площади. Это достаточно подробно отражено в  книге  Михайленко В.Е.,  Ковалев С.Н.  Конструирование форм современных архитектурных сооружений,  Киев:Будiвельник, 1978. - 138 с., а так же в работе Абдюшев А.А., Мифтахутдинов И.Х., Осипов П.П.   Проектирование непологих оболочек минимальной поверхности. Известия КазГАСУ, Строительные конструкции, здания и сооружения, 2009, № 2 (12), где  изучается проблема разработки тентовых тканевых конструкций. Подробный анализ приведенных там результатов   приводит к задаче разработки эффективных методов для приближенного решения уравнения минимальной поверхности и математическому обоснованию найденных методов в плане устойчивости и сходимости приближенных решений. Основная трудность при исследовании данных вопросов заключается в том, что уравнение минимальной поверхности является нелинейным и поэтому традиционные методы, используемые для линейных уравнений,  не пригодны. Похожая ситуация возникает и  при исследовании других нелинейных уравнений, например, уравнения равновесной капиллярной поверхности.   
Первым шагом решения подобных задач стали результаты, связанные с иµС‚ствующие примеры, объясняющие, почему эти результаты РЅРµ РјРѕРіСѓС‚ быть распространены РЅР° многомерный случай. РљСЂРѕРјРµ этого, РІ работах ученых Р