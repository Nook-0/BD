%\documentclass[a4paper,11pt,twoside]{article}
%\usepackage{vvmph2}
%\usepackage{rumathbr}
%
%
%\begin{document}

\udk{517.9}
\bbk{22.161}
\art[Асимптотика собственных значений разностного оператора]{Асимптотика собственных значений разностного оператора с растущим потенциалом и полугруппы операторов}{The asymptotic of eigenvalues for
difference operator\plb with growing potential}
\support{Работа выполнена при финансовой поддержке Российского фонда фундаментальных
исследований (проект № 16-01-00197)}

\fio{Гаркавенко}{Галина}{Валериевна}
\sdeg{кандидат физико-математических наук}{Candidate of Physical and Mathematical Sciences}
\pos{доцент кафедры информатики\\ и методики преподавания математики}{Associate Professor,\\ Department of Informatics and Mathematics Teaching Methodology}
\emp{Воронежский государственный педагогический университет}{Voronezh State Pedagogical University}
\email{g.garkavenko@mail.ru}
\addr{ул. Ленина, 86, 394043 г. Воронеж, Российская Федерация}{Lenina St., 86, 394043 Voronezh, Russian Federation}

\fio{Ускова}{Наталья}{Борисовна}
\sdeg{кандидат физико-математических наук}{\vspace*{-1mm}Candidate of Physical and Mathematical Sciences}
\pos{доцент кафедры высшей математики\\ и физико-математического моделирования}{Associate Professor,\\ Department of Mathematics and Physical-Mathematical Modeling}
\emp{Воронежский государственный технический университет}{Voronezh State Technical University}
\email{nat-uskova@mail.ru}
\addr{просп. Московский, 14, 394026 г. Воронеж, Российская Федерация}{Prosp. Moskovsky, 14, 394026 Voronezh, Russian Federation}

\maketitle

\begin{abstract} Изучаются разностные операторы, соответствующие опера\-тору Штурма --- Лиувилля с растущим потенциалом общего
вида. Методом исследования служит метод подобных операторов, обычно применяемый в спектральном анализе различных классов
дифференциальных операторов. В зависимости от условий, накладываемых на
потенциал, сформулированы теоремы о том, что исследуемый оператор является генератором полугруппы (группы)
операторов и выписан вид подобной полугруппы (группы).
\end{abstract}

\keywords{метод подобных операторов, разностный оператор, собственные значения, полугруппа операторов,
генератор полугруппы операторов}{method of similar operators, difference operator, eigenvalues, semigroup of operators, generator of operator
semigroup}

\section*{Введение}

Рассмотрим гильбертово пространство двусторонних комплексных последовательностей $l_2(\mathbb{Z})$ со скалярным
произведением $(x, y)=\sum\limits_{n\in\mathbb{Z}}x(n)\overline{y(n)}$, $x, y\in l_2(\mathbb{Z})$,
$x: \mathbb{Z}\to\mathbb{C}$, $y: \mathbb{Z}\to\mathbb{C}$, и нормой
$\|x\|=\big(\sum\limits_{n\in\mathbb{Z}}|x(n)|^2\big)^{1/2}$, порожденной этим скалярным произведением. В пространстве
$l_2(\mathbb{Z})$ зададим линейный оператор $A: D(A)\subset l_2(\mathbb{Z})\to l_2(\mathbb{Z})$ формулой
\begin{equation}\label{uskova_1}
(Ax)(n)=a(n)x(n), \quad n\in \mathbb{Z}, \quad x \in D(A),
\end{equation}
с областью определения $D(A)\subset l_2(\mathbb{Z})$ вида
$$
D(A)=\{x\in l_2(\mathbb{Z}): \sum_{n\in\mathbb{Z}}|a(n)|^2|x(n)|^2<\infty\},
$$
где $a: \mathbb{Z}\to\mathbb{C}$~--- последовательность, обладающая свойствами:

1) $a(i)\ne a(j)$ при $i\ne j$;

2) $\lim\limits_{|n|\to\infty}|a(n)|=\infty$;

3) $0<d_i=\inf_{i\ne j}|a(i)-a(j)|\to\infty$, $|i|\to\infty$.

Символом $\rho(A)$ обозначим резольвентное множество оператора $A$, а символом $\sigma(A)$~--- его спектр. Из условий на
последовательность $a: \mathbb{Z}\to\mathbb{C}$ следует, что $\sigma(A)=\{a(n), n\in\mathbb{Z}\}$, то есть спектр оператора
$A$ состоит из простых изолированных собственных значений. Если число $\lambda_0$ не совпадает ни с одним $a(n)$, то
$\lambda_0\in\rho(A)$ и оператор $(A-\lambda_0I)^{-1}: l_2(\mathbb{Z})\to l_2(\mathbb{Z})$ действует по формуле
$((A-\lambda_0I)^{-1}x)(n)=((a(n)-\lambda_0)^{-1}x)(n)$, $n\in\mathbb{Z}$. Такой оператор является нормальным компактным
оператором. Поэтому оператор $A$ также является нормальным оператором. Собственными векторами $e_n$, $n\in\mathbb{Z}$, оператора
$A$ являются векторы стандартного базиса пространства $l_2(\mathbb{Z})$, то есть векторы $e_n$, $n\in\mathbb{Z}$, такие, что
$e_n(m)=\delta_{nm}$, $n, m\in\mathbb{Z}$, где $\delta_{nm}$~--- символ Кронекера.

Пусть самосопряженный ограниченный оператор $B: l_2(\mathbb{Z})\to l_2(\mathbb{Z})$ действует по формуле
\begin{equation}\label{uskova_3}
(Bx)(n)=-2x(n)+x(n+1)+x(n-1), \quad n\in\mathbb{Z}, \quad x\in l_2(\mathbb{Z}).
\end{equation}
В работе изучается возмущенный оператор $\mathcal{A}=A-B: D(\mathcal{A})\subset l_2(\mathbb{Z})\to l_2(\mathbb{Z})$,
$D(\mathcal{A})=D(A)$. Рассматриваемый класс разностных операторов и их матриц соответствует уравнениям Штурма --- Лиувилля при
их дискретизации \cite{Musilimov}.

Основные результаты статьи содержатся в теоремах~\ref{uskova_TH1}--\ref{uskova_TH2}. Для их формулировки нам потребуется
асимптотика собственных значений оператора $A-B$, которая получена в~\cite{9Uskova_Garkavenko_VSU,10Uskova_Garkavenko_Sar}.
Приведем ниже без доказательства соответствующую теорему.
\begin{thm}[{\cite[теорема~1]{10Uskova_Garkavenko_Sar}}]\label{uskova_lh1}
Существует такое целое число $k\geqslant 0$, что спектр $\sigma(\mathcal{A})$ оператора $\mathcal{A}$ представим в виде
$$
\sigma(\mathcal{A})=\sigma_{(k)}\bigcup\bigg(\bigcup_{|i|>k}\sigma_i\bigg),
$$
где $\sigma_{(k)}$ содержит не более чем $2k+1$ собственных значений, $\sigma_i=\{\mu_i\}$, $|i|>k$,~--- одноточечные множества, и
имеют место следующие асимптотические формулы:
\begin{equation}\label{uskova_5}
\mu_i=a(i)+2+O(d_i^{-1}), \quad |i|>k,
\end{equation}
\begin{equation}\label{uskova_6}
\mu_i=a(i)+2-\frac{a(i+1)-2a(i)+a(i-1)}{(a(i+1)-a(i))(a(i-1)-a(i))}+O(d_i^{-3}), \quad |i|>k.
\end{equation}
\end{thm}

Обозначим через $P_n=P(\{a(n)\}, A)$ спектральный проектор, построенный по одноточечному множеству $\{a(n)\}$, $n\in\mathbb{Z}$,
невозмущенного оператора $A$, и через $Q_k$ оператор $Q_k=\sum\limits_{|i|\leqslant k}P_i$. Таким образом, $P_nx=(x, e_n)e_n=x(n)e_n$, $n\in\mathbb{Z}$,
где $x=(x(n))\in l_2(\mathbb{Z})$.

Определения и используемые результаты из теории полугрупп можно найти\plb в~\cite{Baskakov_Harm,Hille,Nagel}.

В следующих теоремах используются обозначения теоремы~\ref{uskova_lh1}, и ее условия считаются выполненными.
Обозначение $\mathrm{Im}\,C$ означает образ некоторого линейного оператора~$C$.

Пусть $\mathcal{H}$~--- комплексное гильбертово пространство. Символом $\mathrm{End}\,\mathcal{H}$ обозначим
банахову алгебру ограниченных линейных операторов, действующих в $\mathcal{H}$ с нормой $\|X\|_\infty=\sup\limits_{\|x\|=1}\|Xx\|$,
$x\in\mathcal{H}$.
\begin{thm}\label{uskova_TH1}
Пусть последовательность $a:\mathbb{Z}\to\mathbb{C}$ удовлетворяет условию $\mathrm{Re}\,a(n)\leqslant\beta$
для всех $n\in\mathbb{Z}$ и некоторого $\beta\in\mathbb{R}$. Тогда оператор $\mathcal{A}$ является генератором полугруппы
операторов $T: \mathbb{R}_+\to\mathrm{End}\,l_2(\mathbb{Z})$, и эта полугруппа подобна полугруппе
$\widetilde{T}: \mathbb{R}_+\to\mathrm{End}\,l_2(\mathbb{Z})$ вида
\begin{equation}\label{uskova_7}
\widetilde{T}(t)=\widetilde{T}_{(k)}(t)\oplus \widetilde{T}^{(k)}(t), \quad t\in\mathbb{R}_+,
\end{equation}
действующей в $l_2(\mathbb{Z})=\mathcal{H}_{(k)}\oplus\mathcal{H}^{(k)}$, где $\mathcal{H}_{(k)}=\mathrm{Im}\,Q_k$ и
$\mathcal{H}^{(k)}=\mathrm{Im}\,(I-Q_k)$. Полугруппа $\widetilde{T}^{(k)}: \mathbb{R}_+\to\mathrm{End}\,\mathcal{H}^{(k)}$
определяется формулой
$$
\widetilde{T}^{(k)}(t)x=\sum_{|n|>k}e^{\mu_nt}P_nx, \quad x\in\mathcal{H}^{(k)}, \quad t\in\mathbb{R}_+,
$$
где числа $\mu_n$, $|n|>k$, определены равенствами (\ref{uskova_5}), (\ref{uskova_6}).
\end{thm}

\begin{thm}\label{uskova_TH3}
Пусть $\alpha\leqslant \mathrm{Re}\,a(n)\leqslant\beta$, $\alpha$, $\beta\in\mathbb{R}$, для всех $n\in\mathbb{Z}$.
Тогда оператор $\mathcal{A}: D(\mathcal{A})\subset l_2(\mathbb{Z})\to l_2(\mathbb{Z})$ является производящим оператором группы операторов
$T: \mathbb{R}\to \mathrm{End}\,l_2(\mathbb{Z})$ и эта группа подобна группе
$\widetilde{T}: \mathbb{R}\to \mathrm{End}\,l_2(\mathbb{Z})$ вида (\ref{uskova_7}), где
$$
\widetilde{T}^{(k)}(t)x=\sum_{|n|>k}e^{\mu_nt}P_nx, \quad x\in\mathcal{H}^{(k)}, \quad t\in\mathbb{R}.
$$
\end{thm}

\begin{thm}\label{uskova_TH2}
Пусть оператор $\mathcal{A}: D(\mathcal{A})\subset l_2(\mathbb{Z})\to l_2(\mathbb{Z})$ самосопряжен. Тогда оператор
$i\mathcal{A}$ является генератором группы изометрий $T: \mathbb{R}\to \mathrm{End}\,l_2(\mathbb{Z})$, эта группа подобна
группе вида (\ref{uskova_7}), где
$$
\widetilde{T}^{(k)}(t)x=\sum_{|n|>k}e^{i\mu_nt}P_nx, \quad x\in\mathcal{H}^{(k)}, \quad t\in\mathbb{R}.
$$
\end{thm}

Отметим, что в работах \cite{Uskova_Garkavenko_Tavr,9Uskova_Garkavenko_VSU,10Uskova_Garkavenko_Sar,11Uskova_Garkavenko_Bel}
исследовались спектральные свойства указанного разностного оператора $A-B$. Вопросы, связанные с построением генератора
полугруппы (группы) операторов, в этих работах не рассматривались. В \cite{11Uskova_Garkavenko_Bel} изучался случай
четной последовательности $a: \mathbb{Z}\to\mathbb{C}$, что соответствует четному потенциалу.

\section{О методе подобных операторов}

Метод подобных операторов берет начало с работ А.~Пуанкаре, А.М.~Ляпунова, Н.М.~Крылова, Н.Н.~Боголюбова, К.О.~Фридрихса,
Р.~Тернера и окончательно оформляется в работах А.Г.~Баскакова
\cite{Baskakov1983,Baskakov1986,Baskakov_Harm,Baskakov1994,Baskakov2011}. Мы будем придерживаться
идеологии и методологии работы \cite{Baskakov2011}. Обычно метод подобных операторов применяется для получения спектральных
характеристик дифференциальных операторов (см., например, недавно вышедшие работы~\cite{Baskakov2015,7Baskakov_Polyakov,
PolyakovVSU,UskovaDE_2016,UskovaMN_2004,Shelkovoj}).

К исследованию разностных операторов метод подобных операторов применялся в
\cite{Uskova_Garkavenko_Tavr,9Uskova_Garkavenko_VSU,10Uskova_Garkavenko_Sar,11Uskova_Garkavenko_Bel}.
\begin{dfn}[\cite{Baskakov2011}]
Два линейных оператора $A_i:D(A_i)\subset\mathcal{H}\to\mathcal{H}$, $i=1, 2$, называются \emph{подобными}, если существует
непрерывно обратимый оператор $U\in\mathrm{End}\,\mathcal{H}$ такой, что $UD(A_2)=D(A_1)$ и $A_1Ux=UA_2x$, $x\in D(A_2)$.
Оператор $U$ называется \emph{оператором преобразования} оператора $A_1$ в $A_2$.
\end{dfn}
Подобные операторы обладают рядом совпадающих спектральных свойств. Соответствующее утверждение удобно формулировать
в виде следующей леммы.
\begin{lem}[\cite{Baskakov2011}]\label{uskova_lh2}
Пусть $A_i:D(A_i)\subset\mathcal{H}\to\mathcal{H}$, $i=1, 2$,~--- два подобных оператора, и
$U\in\mathrm{End}\,\mathcal{H}$~--- оператор преобразования оператора $A_1$ в оператор $A_2$. Тогда
справедливы следующие утверждения:

1) $\sigma(A_1)=\sigma(A_2)$, $\sigma_d(A_1)=\sigma_d(A_2)$, $\sigma_c(A_1)=\sigma_c(A_2)$, $\sigma_r(A_1)=\sigma_r(A_2)$,
где $\sigma(A_i)$, $\sigma_d(A_i)$, $\sigma_c(A_i)$, $\sigma_r(A_i)$, $i=1,2,$ "--- спектр, дискретный, непрерывный и остаточный
спектры операторов $A_i$, $i=1,2$, соответственно;

2) если оператор $A_2$ допускает разложение $A_2=A_{21}\oplus A_{22}$, где $A_{2k}=A|\mathcal{H}_k$, $k=1, 2$,~--- сужение
$A_2$ на $\mathcal{H}_k$ относительно прямой суммы $\mathcal{H}=\mathcal{H}_1\oplus\mathcal{H}_2$ инвариантных относительно $A_2$
подпространств $\mathcal{H}_1$, $\mathcal{H}_2$, то подпространства $\widetilde{\mathcal{H}}_k=U(\mathcal{H}_k)$, $k=1, 2$,
инвариантны относительно оператора $A_1$ и $A_1=A_{11}\oplus A_{12}$, где $A_{1k}=A|\widetilde{\mathcal{H}}_k$, $k=1, 2$,
относительно разложения $\mathcal{H}=\widetilde{\mathcal{H}}_1\oplus \widetilde{\mathcal{H}}_2$. Кроме того, если $P$~---
проектор, осуществляющий разложение $\mathcal{H}=\mathcal{H}_1\oplus\mathcal{H}_2$ (то есть $\mathcal{H}_1=\mathrm{Im}\,P$~---
образ проектора $P$, $\mathcal{H}_2=\mathrm{Im}\,(I-P)$~--- образ дополнительного проектора), то проектор
$\widetilde{P}\in\mathrm{End}\,\mathcal{H}$, осуществляющий разложение
$\mathcal{H}=\widetilde{\mathcal{H}}_1\oplus \widetilde{\mathcal{H}}_2$, определяется формулой
$$
\widetilde{P}=UPU^{-1};
$$

3) если оператор $A_2$ является генератором сильно непрерывной полугруппы операторов
$T_2: \mathbb{R}_+=[0, \infty)\to\mathrm{End}\,\mathcal{H}$ (полугруппы класса $C_0$), то оператор $A_1$ также является
генератором сильно непрерывной полугруппы операторов
$$
T_1(t)=UT_2(t)U^{-1}, \quad t\geqslant 0, \quad T_1: \mathbb{R}_+\to\mathrm{End}\,\mathcal{H}.
$$
\end{lem}

Линейные операторы, действующие в пространстве операторов согласно терминологии М.Г.~Крейна будем называть трансформаторами.

Наиболее важным понятием метода подобных операторов является понятие допустимой тройки.
\begin{dfn}[\cite{Baskakov2011}]\label{uskova_def1}
Пусть $J:\mathrm{End}\,\mathcal{H}\to \mathrm{End}\,\mathcal{H}$, $\Gamma:\mathrm{End}\,\mathcal{H}\to\mathrm{End}\,\mathcal{H}$,
являются трансформаторами. Тройку $(\mathrm{End}\,\mathcal{H}, J, \Gamma)$ назовем допустимой для невозмущенного оператора $A$
тройкой, и $\mathrm{End}\,\mathcal{H}$~--- допустимым пространством возмущений, если

1) $J$ и $\Gamma$~--- непрерывные трансформаторы, причем $J$~--- проектор;

2) $(\Gamma X)D(A) \subset D(A)$, при этом
\begin{equation}\label{uskova_8}
A\Gamma X - (\Gamma X)A=X - JX, \quad X\in\mathrm{End}\,\mathcal{H},
\end{equation}
и $Y=\Gamma X\in\mathrm{End}\,\mathcal{H}$~--- единственное решение уравнения $AY-YA=X-JX$,
удовлетворяющее условию $JY=0$;

3) существует постоянная $\gamma>0$ такая, что
$$
\|\Gamma\|\leqslant\gamma, \quad \max\{\|X\Gamma Y\|, \|(\Gamma X)Y\|\}\leqslant \gamma\|X\|\|Y\|;
$$

4) для любого $X\in\mathrm{End}\,\mathcal{H}$ и $\varepsilon>0$ существует $\lambda_\varepsilon\in\rho(A)$ такое, что
$$
\|X(A- \lambda_\varepsilon I)^{-1} \|<\varepsilon.
$$
\end{dfn}
\begin{thm}[\cite{Baskakov2011}]\label{uskova_th4}
Пусть $(\mathrm{End}\,\mathcal{H}, J, \Gamma)$~--- допустимая для оператора $A:D(A)\subset\mathcal{H}\to\mathcal{H}$ тройка
и $B$~--- некоторый оператор из $\mathrm{End}\,\mathcal{H}$. Тогда, если
\begin{equation}\label{uskova_10}
4\gamma\|B\|<1,
\end{equation}
то оператор $A-B$ подобен оператору $A-JX_*$, где оператор $X_*\in\mathrm{End}\,\mathcal{H}$ является решением
нелинейного операторного уравнения
\begin{equation}\label{uskova_11}
X=B\Gamma X - (\Gamma X)(JB)-(\Gamma X)J(B\Gamma X)+B.
\end{equation}
Решение $X_*$ может быть найдено методом простых итераций, положив $X_0=0$, $X_1=B, \dots$\ Преобразование подобия оператора
$A-B$ в оператор $A-JX_*$ осуществляет оператор $I+\Gamma X_*\in\mathrm{End}\,\mathcal{H}$.
\end{thm}

Из леммы~\ref{uskova_lh2} и теоремы~\ref{uskova_th4} следует следующая теорема.
\begin{thm}
Пусть $(\mathrm{End}\,\mathcal{H}, J, \Gamma)$~--- допустимая тройка для оператора $A: D(A)\subset\mathcal{H}\to\mathcal{H}$,
оператор $B\in\mathrm{End}\,\mathcal{H}$ удовлетворяет условию (\ref{uskova_10}), и оператор $A-JX_*$, где
$X_*$~--- решение нелинейного операторного уравнения (\ref{uskova_11}), является генератором полугруппы операторов
$\widetilde{T}: \mathbb{R}_+\to\mathrm{End}\,\mathcal{H}$. Тогда оператор $A-B$ является генератором полугруппы операторов
$T: \mathbb{R}_+\to\mathrm{End}\,\mathcal{H}$, определенной равенствами
$$
T(t)=(I+\Gamma X_*)\widetilde{T}(t)(I+\Gamma X_*)^{-1}, \quad t\in\mathbb{R}_+.
$$
\end{thm}

\section{Преобразование подобия исходного оператора}

В этом параграфе символом $\mathcal{H}$ будем обозначать гильбертово пространство $l_2(\mathbb{Z})$. Отметим, что в
дальнейшем удобно будет пользоваться матричным представлением операторов $A$ и $B$, определенных формулами (\ref{uskova_1})
и (\ref{uskova_3}) соответственно.

Представим оператор $A-B$ в виде $A-B=\widetilde{A}-\widetilde{B}$, где $(\widetilde{A}x)(n)=a(n)x(n)+2x(n)$,
$(\widetilde{B}x)(n)=x(n-1)+x(n+1)$. Тогда $\sigma(\widetilde{A})=\{a(n)+2, n\in\mathbb{Z}\}$ --- собственные векторы и спектральные
проекторы те же, что и у оператора $\widetilde{A}$, оператор $\widetilde{B}\in\mathrm{End}\,\mathcal{H}$, $\|\widetilde{B}\|=2$ и
главная диагональ у матрицы оператора $\widetilde{B}$ нулевая.

Перейдем к определению трансформаторов $J_k: \mathrm{End}\,\mathcal{H}\to\mathrm{End}\,\mathcal{H}$ и
$\Gamma_k: \mathrm{End}\,\mathcal{H}\to\mathrm{End}\,\mathcal{H}, k \geqslant 0$. Положим
$$
J_kX=\sum_{|n|>k}P_nXP_n+Q_kXQ_k, \quad X\in\mathrm{End}\,\mathcal{H}.
$$

Очевидно, что трансформатор $J_k$ блочно диагонализирует матрицу оператора $X$ и $J_k\widetilde{B}$ --- оператор конечного ранга.

Перепишем равенство (\ref{uskova_8}) для матричных элементов $y_{nm}$, $n$, $m\in\mathbb{Z}$, матрицы $Y$ ($Y=\Gamma_k X$):
$$
a(n)y_{nm}-y_{nm}a(m)=x_{nm}, \quad n\ne m, \max\{|n|, |m|\}>k,
$$
откуда
\begin{equation}\label{uskova_13}
y_{nm}=\frac{x_{nm}}{a(n)-a(m)},
\end{equation}
и $y_{nm}=0$ в противном случае. Так как $a(n)\ne a(m)$ при $n\ne m$, то формула (\ref{uskova_13}) корректна. Таким образом,
матричные элементы оператора $\Gamma_k X$ определены. При этом $Y\in\mathrm{End}\,\mathcal{H}$.

\begin{lem}[{\cite[лемма 2]{10Uskova_Garkavenko_Sar}}]\label{uskova_lh3}
Тройка $(\mathrm{End}\,l_2(\mathbb{Z}), J_k, \Gamma_k)$ является допустимой для оператора $\widetilde{A}$ тройкой
при любом $k\geqslant 0$.
\end{lem}

Отметим, что в уже упомянутых работах
\cite{Uskova_Garkavenko_Tavr,9Uskova_Garkavenko_VSU,10Uskova_Garkavenko_Sar,11Uskova_Garkavenko_Bel}
для разностного оператора были построены различные допустимые тройки. Они отличались выбором пространства допустимых
возмущений. Наряду с $\mathrm{End}\,\mathcal{H}$ в качестве пространства допустимых возмущений можно рассматривать пространство
$\mathrm{End}_1\,\mathcal{H}$ операторов из $\mathrm{End}\,\mathcal{H}$, имеющих матрицы с суммируемыми диагоналями. При
этом возмущение $\widetilde{B}$ принадлежит пространству $\mathrm{End}_1\,\mathcal{H}$. Также в качестве пространства
допустимых возмущений может выступать двусторонний идеал операторов Гильберта --- Шмидта $\mathfrak{S}_2(\mathcal{H})$
из алгебры $\mathrm{End}\,\mathcal{H}$. В этом случае на последовательность $d: \mathbb{Z}\to\mathbb{C}$ необходимо
наложить дополнительное условие $\sum\limits_{i\in\mathbb{Z}}d_i^{-2}<\infty$. Отметим также, что
$\widetilde{B}\notin \mathfrak{S}_2(\mathcal{H})$. Поэтому придется проводить вначале предварительное преобразование
подобия (см.:~\cite{Baskakov2011,PolyakovVSU,UskovaDE_2016,UskovaMN_2004}).

При построении генераторов групп (полугрупп) операторов можно использовать любую из перечисленных допустимых троек. Выбор
пространств допустимых возмущений несущественен, поэтому остановимся на $\mathrm{End}\,\mathcal{H}$.
\begin{thm}[{\cite[теорема 4]{10Uskova_Garkavenko_Sar}}]\label{uskova_th5}
Существует такое $k>0$, что оператор $\widetilde{A}-\widetilde{B}$ подобен оператору блочно-диагонального вида
$\widetilde{A}-J_kX_*$, то есть
$$
(\widetilde{A}-\widetilde{B})(I+\Gamma_kX_*)=(I+\Gamma_kX_*)(\widetilde{A}-J_kX_*),
$$
где оператор $X_*$ есть решение уравнения (\ref{uskova_11}) с $\Gamma_k$ и $J_k$ и возмущением $\widetilde{B}$.
\end{thm}

Отметим, что теорема~\ref{uskova_th5} вытекает из леммы~\ref{uskova_lh3} и теоремы~\ref{uskova_th4}, если учесть, что
$d_n\to\infty$ при $|n|\to\infty$.

\section{Доказательство основных результатов}

Пусть $\mathcal{H}$~--- абстрактное комплексное гильбертово пространство, причем оно представимо в виде взаимно ортогональных
замкнутых подпространств $\mathcal{H}_n$, $n\in\mathbb{Z}$, то есть
\begin{equation}\label{uskova_9}
\mathcal{H}=\bigoplus\limits_{n\in\mathbb{Z}}\mathcal{H}_n,
\end{equation}
где $\mathcal{H}_i$ ортогонально $\mathcal{H}_j$ при $i\ne j$, $i, j\in\mathbb{Z}$, и $x=\sum\limits_{n\in\mathbb{Z}}x_n$,
$x_n\in\mathcal{H}_n$, $\|x\|^2=\sum\limits_{n\in\mathbb{Z}}\|x_n\|^2$. Такое представление пространства $\mathcal{H}$ ведет
к существованию разложения единицы системой ортопроекторов $\mathcal{P}_n$, $n\in\mathbb{Z}$.

При этом проекторы $\mathcal{P}_n$, $n\in\mathbb{Z}$, обладают свойствами:

1) $\mathcal{P}_n^*=\mathcal{P}_n$, $n\in\mathbb{Z}$;

2) $\mathcal{P}_i\mathcal{P}_j=0$ при $i\ne j$, $i, j\in\mathbb{Z}$, $\mathcal{P}_i^2=\mathcal{P}_i$;

3) ряд $\sum\limits_{n\in\mathbb{Z}}\mathcal{P}_nx$ безусловно сходится к $x\in\mathcal{H}$ и
$\|x\|^2=\sum\limits_{k\in\mathbb{Z}}\|\mathcal{P}_kx\|^2$.
\begin{dfn}
Линейный оператор $\mathcal{E}: D(\mathcal{E})\subset\mathcal{H}\to\mathcal{H}$ называется ортогональной прямой суммой
ограниченных операторов $\mathcal{E}_n\in\mathrm{End}\,\mathcal{H}_n$, $n\in\mathbb{Z}$, относительно разложе�