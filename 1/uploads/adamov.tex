%\documentclass[a4paper,11pt,twoside]{article}
%%\usepackage[cp1251]{inputenc}  %% 1
%\usepackage[T2A]{fontenc}      %% 2
%%\usepackage[russian]{babel}    %% 3
%
%\usepackage{vvmph2}
%\usepackage{rumathbr}
%\begin{document}

\udk{53.082.531, 53.082.532}
\bbk{22.343 }
\art[Оценка возможности применения метода лазерной триангуляции]{Оценка возможности применения метода лазерной триангуляции к измерениям толщин тонких пленок}{Evaluation of the possibility of applying the method\plb of the laser triangulation to measurement\plb of thin film thickness}
%\support{Ссылка на грант - если имеется} %команда может отсутствовать
\fio{Адамов}{Антон}{Андреевич}
%\sdeg{ученая степень}{academic degree in English} %если есть
\pos{аспирант кафедры лазерной физики}{Postgraduate student of Department of Laser Physics}
\emp{Учебный мастер кафедры радиофизики,\\ Волгоградский государственный университет}{Class Master of Department of Radiophysics,\\ Volgograd State University}
\email{anto-kha1@mail.ru, lf@volsu.ru}
\addr{просп. Университетский, 100, 400062 г. Волгоград, Российская Федерация }{Prosp. Universitetsky, 100, 400062 Volgograd, Russian Federation}

\fio{Храмов}{Владимир}{Николаевич}
\sdeg{Кандидат физико-математических наук}{Candidate of Physical and Mathematical Sciences} %если есть
\pos{доцент,\\
заведующий кафедрой лазерной физики}{Associate Professor,\\ Head of Department of Laser Physics}
\emp{Волгоградский государственный университет}{Volgograd State University}
\email{vladimir.khramov@volsu.ru, lf@volsu.ru}
\addr{просп. Университетский, 100, 400062 г. Волгоград, Российская Федерация }{Prosp. Universitetsky, 100, 400062 Volgograd, Russian Federation}

\maketitle

\begin{abstract}
Рассмотрена задача определения толщины тонких пленок методом лазерной триангуляции. Получено выражение, связывающее значение толщины пленки и координаты лучей на фотоприемнике. Экспериментально показана возможность применения данного метода для измерения толщин в диапазоне [0,1;1] мм. Полученные результаты могут быть использованы в офтальмологии для экспресс-диагностики во время проведения операций на роговичном слое.
\end{abstract}

\keywords{лазерная триангуляция, толщина, тонкая пленка, ПЗС-матрица, показатель преломления, погрешность измерения}{laser triangulation, thickness, thin film, CCD matrix, index of refraction, measurement error}

\section*{Введение}
Измерение толщин тонких прозрачных пленок (биологических тканей), в частности, роговицы и остаточных слоев роговичного слоя (сотых долей миллиметра) в режиме реального времени является нерешенной проблемой в современной офтальмологии при проведении ряда медицинских операций на передних тканях глаза.


Используемые в офтальмологической практике и выпускаемые промышленностью (как отечественной, так и зарубежной) измерители толщины роговицы (пахиметры), основанные на оптических и ультразвуковых методах, предназначены, во-первых, для измерения исходного, достаточно толстого слоя роговицы (от десятых долей до единиц миллиметров), во-вторых, не обеспечивают измерения во время офтальмологической операции. Современные методы лазерной физики и интерференционной оптики позволяют решить данную проблему.


Покрытия из тонких пленок также широко распространены и используются в электронной промышленности и оптике. Их применяют при производстве солнечных батарей, жидкокристаллических дисплеев и фотоприемников. При нанесении тонких пленок появляется необходимость контроля их толщины. Стоит отметить необходимость измерений тонких прозрачных и полупрозрачных пленок в биологических и медицинских исследованиях.


Лазерными измерителями толщин и расстояний занимались в Саратовском государственном университете им. Чернышевского
В.П. Рябухо, В.В. Лычагов, А.Л. Кальянов \cite{lichagov}.
Применение лазерного излучения для визуализации процессов в биологических тканях также изучалось представителями Саратовского университета Э.А. Геноной, И.В.~Федосовым, А.Н. Башкатовым, Г.Б. Альтшулером, Д.А. Зимняковым, В.В.~Тучиным \cite{genina}. Исследование биотканей методами светорассеяния проводились В.В. Тучиным \cite{tuchin}. Вопросами оптической томографии тканей занимались Д.А. Зимняков, В.В. Тучин \cite{zimnyakov}.
Исследования параметров слоистых неоднородных структур методом лазерной поляризационной нефелометрии проводила В.Ф. Изотова \cite{izotova}.
Вопросами экспресс-диагностики состояния роговицы глаза методами линейной и нелинейной нефелометрии занимались в Волгоградском государственном университете В.В. Яцышен, Р.В. Щелоков \cite{zimnyakov, shcelok}.


Нужно упомянуть, что все известные методы измерений толщин пленочных покрытий являются косвенными, но самым простым с методологической точки зрения и реализации является триангуляционный способ. Можно ожидать, что использование этого метода позволит решить данную проблему измерений в офтальмологии и других отраслях промышленности и техники.

\section{Постановка задачи}
Рассмотрим задачу об измерении толщины тонкой пленки применительно к офтальмологии.


С учетом анатомического строения и оптической схемы (рис. \ref{ris:eye}), учитывающей показатели преломлений различных составляющих глаза, в данной работе была выбрана модель роговицы, которая представляет собой тонкую прозрачную для видимого света пленку на поверхности воды. Ввиду сильной локализованности измерений (используется малый диаметр лазерного пучка) кривизной роговицы можно пренебречь и использовать плоскопараллельную пленку. В первом приближении также не учитывалась пространственная дисперсия $n = n(z)$ для роговичного слоя, где $z$ --- толщина роговицы. С помощью триангуляционного метода необходимо определить толщину прозрачной пленки на поверхности воды.

\begin{figure}[h]
\begin{center}
\includegraphics[width=8cm]{eye.jpg}
\end{center}
\caption{Оптическая схема глаза}
\label{ris:eye}
\end{figure}

\section{Методика решения}

Триангуляционный метод использует способность лазерного луча распространяться в хорошо коллимированной форме (то есть с малой расходимостью) на большие расстояния. Обычно лазерный луч освещает точку, расстояние до которой от лазерного устройства необходимо измерить; по существу, лазер используется в качестве \hbox{указателя.}

\begin{figure}[h]
\begin{center}
\includegraphics[width=12cm]{block.jpg}
\end{center}
\caption{Блок-схема лазерного триангуляционного измерителя}
\label{ris: block}
\end{figure}

Измерительное устройство, блок-схема которого изображена на рисунке \ref{ris: block}, включает приемно-излучательную систему: лазерный излучатель (светодиодный/твердотельный или газовый) и фотодетектор с оптической системой; и систему обработки информации. В качестве фотодетектора используют ПЗС-камеру (CDD-матрица) или PSD-датчик. (В~данной работе результаты исследования регистрировались на цифровой фотоаппарат: 13 (4\,:\,3) мегапикселей ($4\,128 \times 3\,096$), основным элементом которого является ПЗС-матрица. Сторонняя засветка была устранена. Эксперимент проводился при минимальном освещении.) Лазерный источник проецирует на объект световой луч. Часть светового потока, отражаясь от объекта, проходит через объектив датчика и попадает на активную поверхность фотодатчика. В зависимости от расстояния до объекта меняется и угол падения отраженного луча на поверхность датчика. Эти углы и определяют местоположения пятен на активной поверхности фотоприемного датчика. Базовые расстояния внутри оптической системы ($AO,OD_1$) известны. Путем расчета определяется расстояние до поверхностей (толщина) пленки \cite{johnson, barishnikov}.


Примем обозначения $AO=r$ --- расстояние от поверхности до объектива; $OD_1=r'$~--- от объектива до ПЗС; $CD_1=\Delta x$ --- расстояние между световыми пятнами на ПЗС; $AB=\Delta z$ --- толщина пленки.
Из рисунка \ref{ris:triang-stand} видно, что

\begin{equation}
\left\{
\begin{array}{ccc}
AB=AB_1\frac{\tg\alpha_1}{\tg\gamma};\\
\frac{\sin\alpha_1}{\sin\gamma}=n;
\end{array}
\right.
\end{equation}

\begin{equation}
M = \frac{x_1}{z'_1}=\frac{r'}{r},
\end{equation}
где $M$ --- увеличение линзы, расположенной в точке О; $z_1'$ --- расстояние между световыми пятнами;  $x_1$ --- изображение этого расстояния.

\begin{figure}[h!]
\begin{center}
\includegraphics[width=10cm]{triang-stand.jpg}
\end{center}
\caption{Функционально-оптическая схема\\ лазерной триангуляционной приемно-излучательной системы}
\label{ris:triang-stand}
\end{figure}

\begin{equation}
z_1'=x_1\frac{r}{r'};
\end{equation}

\begin{equation}
z_1'=z'\sin \alpha_1;
\end{equation}
где  $z_'$ --- расстояние между точками под углом $\alpha_1$ к оптической оси.

Из треугольника $CH_1B_1$:

\begin{equation}
\sin \alpha_1=\cos \beta_1;
\end{equation}
откуда получаем с учетом

\begin{equation}
x_1 = \Delta x \sin \beta_1;
\end{equation}

\begin{equation}
z'=\Delta x \frac{r}{r'} \ctg \alpha_1.
\end{equation}

Используя закон преломления лучей (Снеллиуса), можно получить выражение для истинной толщины пленки:

\begin{equation}
\Delta z = \Delta x \frac{r}{r'} \frac{\sqrt{n^2-\sin^2\alpha_1}}{\sin \alpha_1},
\label{eq:triangl-stand}
\end{equation}
где $n$ --- показатель преломления пленки.\\


Углы $\alpha$ и $\alpha_1$ связаны через известное из \cite{bronshtein}
соотношение для треугольников $AOB_1$ и $COD_1$:

\begin{equation}
\cos \alpha_1 = \frac{r'\sin \alpha - \Delta x}{\sqrt{(\Delta x)^2
+(r')^2-2\Delta x \cdot r' \cdot \sin \alpha}}.
\label{eq:angl}
\end{equation}

Вид формулы (\ref{eq:triangl-stand}) совпадает с формулой, полученной для
измерения положений смещения поверхности в воздухе \cite{burkov}, с точностью до угла поворота приемника излучения и учета показателя преломления среды.


Если менять значение углов $\alpha_1$ и $\beta_1$ независимо друг от друга, то формула (\ref{eq:triangl-stand}) перейдет в формулу

\begin{equation}
\Delta z = 2 \Delta x \frac{r}{r'} \frac{\sin \beta_1}{\sin 2\alpha_1}\sqrt{n^2-\sin^2\alpha_1}.
\end{equation}

Также следует учесть $\tg 2\alpha_1=M\tg\beta_1$~--- условие Шемпфлюта [10], где $M=\frac{r'}{r}$~--- увеличение оптической системы. Плоскость, проходящая через зондирующий луч, оказывается при этом оптически сопряженной с выходной плоскостью, поэтому любая точка зондирования проецируется в плоскость расположения фотоприемника без каких-либо дефокусировок.


Из выражения (\ref{eq:triangl-stand}) с учетом (\ref{eq:angl}) можно получить, что оптимальный угол
наблюдения (то есть угол, при котором расстояние между фокусируемыми
пятнами максимально: $\Delta x \to \text{max}$) равен $90^
0$, но пятна имеют некоторый физический размер $y$, а значит они будут иметь под углом $\alpha$ видимый размер $y'=y\cos \alpha$. Нетрудно видеть, что $y' \to \text{max}$ при $ \alpha \to 0^0$. Отсюда следует, что оптимальным углом наблюдения будет угол в $45^0$.

\subsection{Предел измерения для триангуляционного метода}

Оценим минимальное значение толщины пленки, которую можно измерить с помощью метода триангуляции.


Выберем угол $\alpha = 45^0$ (как наилучший угол для проведения измерений); среднее значение для показателя преломления роговицы $n=1,355$. Тогда для толщины пленки выражение
(\ref{eq:triangl-stand}) примет вид

\begin{equation}
\Delta z \approx 1,63 \Delta x \frac{r}{r'}.
\label{eq: tolshina}
\end{equation}

Оценим минимальное значение $\Delta x$ с помощью критерия разрешения Рэлея:

\begin{equation}
\Delta x_{\text{min}}=1,22\frac{\lambda}{b}F,
\label{eq: relei}
\end{equation}
где $b=1$ мм --- диаметр апертурной диафрагмы; $F=2,5$ см --- фокусное расстояние линзы. С учетом того что фронт волны для полупроводникового лазера данного типа будет заметно отличаться от плоского, для которого верна формула (\ref{eq: relei}), реальное значение будет примерно на порядок больше. Также следует принять во внимание, что наблюдение световых пятен на поверхностях пленки и воды, на которых они имеют форму кругов, происходит под углом $45^0$, то есть они будут видны как эллипсы с коэффициентом сжатия, равным $\cos 45^0 \approx 0,71$.\\


В итоге имеем следующие результаты минимального значения оптического разрешения для каждого лазера:

\begin{enumerate}
\item $\Delta x_{\text{min}}=0,07$ мм ($\lambda_1=655$ нм);
\item $\Delta x_{\text{min}}=0,06$ мм ($\lambda_2=532$ нм);
\item $\Delta x_{\text{min}}=0,04$ мм ($\lambda_3=405$ нм).
\end{enumerate}


Используя формулу (\ref{eq: tolshina}), получим минимальные значения для толщины пленки, которую можно измерить с помощью данного метода:

\begin{equation}
\Delta z_{\text{min}} \in [0,04; 0,08] \, \text{мм}.
\end{equation}

Итак, для зеленого лазера данного типа, который является наиболее эффективным при использовании ПЗС-линейки в качестве приемника излучения, принципиально невозможно измерить толщину пленки меньшую, чем 0,05 мм. Однако если использовать систему, которая сформирует фронт излучения плоским, то значение для толщины может быть уменьшено еще примерно на порядок. Но данная система является достаточно громоздкой, дорогостоящей и весьма чувствительной к внешнему воздействию повреждающих факторов. К примеру, механические вибрации могут привести к смещению оптических осей, формирующих плоский фронт линз в данной системе, что приведет к увеличению диаметра пучка на поверхностях пленки и воды.


Кроме того, если вычислить погрешность для толщины измеряемой пленки по формуле для косвенных воспроизводимых измерений \cite{kassandrova}, при $\Delta n = 0,5$, $\Delta (\Delta x)=0,005$ мм, $\Delta \alpha=\Delta \alpha_1 = 0,008$ рад, $\Delta M=0,1$:

\begin{equation}
\Delta (\Delta z)=\sqrt{\sum_{i=1}^n \left(\Delta y_i \frac{\partial G}{\partial y_i}\right)^2} \approx 0,02 \, \text{мм}.
\end{equation}

Следовательно, реально достижимое значение для наименьшей толщины, которую можно будет измерить этим методом при данных параметрах, будет лежать в окрестности 0,1 мм.

\section{Экспериментальные исследования}

Апробация метода лазерной триангуляции проводилась с помощью установки, схема которой представлена на рисунке \ref{ris:device}.

\begin{figure}[h]
\begin{center}
\includegraphics[width=11cm]{device.jpg}
\end{center}\vspace*{-3mm}
\caption{Схема экспериментальной установки}
\label{ris:device}
\end{figure}

Максимальное разрешение (соответственно определение минимальной толщины пленки) в методе лазерной триангуляции возможно при выполнении условия: диаметр перетяжки должен быть минимальным на поверхности исследуемой пленки \cite{andreev, burkov, barishnikov}. В~данной работе использовался полупроводниковый лазер, фронт волны которого весьма сильно отличается от плоского, а распределение лазерного пучка является колоколообразным, но не гауссовым. И стягивание данного лазерного пучка не принесет хорошего результата. Поэтому перед линзой, которая фокусирует пучок на поверхность пленки, располагалась формирующая (ограничивающая) диафрагма диаметром 1 мм, с помощью которой из фронта излучения вырезалась центральная часть.


Отраженные лучи от поверхностей пленки (границы воздух/пленка и плен\-ка/во\-да) фокусировались объективом фотокамеры с использованием еще одной собирающей линзы.


В эксперименте использовались пленки из пластмассы с различными значениями коэффициентов отражения и преломления, которые подбирались для опыта из значений их толщин. Пленки располагались на воде (толщина водного слоя много больше толщины пленок). Коэффициенты преломления для пленок на воде определялись с помощью измерительного микроскопа, путем определения видимой и истинной толщин пленки. Погрешность показателя преломления вычислялась с помощью формулы для косвенных воспроизводимых измерений \cite{kassandrova} и составила $\Delta n = 0,1 \div 0,2$. Результаты приведены в таблице, где $D$ --- толщина пленки.

\begin{table}[h]\small
\caption*{Показатели преломлений пленок}
\begin{center}
\begin{tabular}{|c|c|c|c|}
\hline
$D$, мм & Материал & Измеренное значение $n$ & Паспортное значение $n$ \\ \hline
0.90 & Полистирол & 1,8 & 1,55\\ \hline
0.75 & Полистирол & 1,6 & 1,55\\ \hline
0.56 & Полипропилен & 1,5 & 1,49\\ \hline
0.50 & Полистирол & 1,8 & 1,55\\ \hline
0.46 & Полистирол & 1,7 & 1,55\\ \hline
0.30 & Поливинилхлорид & 1,6 & 1,54\\ \hline
0.23 & Поливинилхлорид & 1,6 & 1,54\\ \hline
0.20 & Полипропилен & 1,5 & 1,49\\ \hline
0.10 & Полипропилен & 1,5 & 1,49\\ \hline
\end{tabular}
\end{center}
\label{tab:tab1-D}
\end{table}


Критерием применимости метода служила возможность разрешения двух световых точек на фокусируемых камерой изображениях, получаемых отражением падающего лазерного излучения от поверхностей исследуемой пленки и водяной подложки.


Все измерения проводились при использовании зеленого лазера
($\lambda=532$ нм) и фокусирующей линзы (расположенной между камерой и пленкой) с фокусным расстоянием 7 см, если не указано других параметров. Фотографии представлены с инвертированием цветовой схемы для удобства восприятия.


Наиболее отчетливая картинка двух световых пятен получилась при наибольшей толщине пленки (рис. \ref{ris:0-9}, \emph{a}). При уменьшении толщины пленки соответственно разрешение точек ухудшалось (рис. \ref{ris:0-9}, \emph{b}, \emph{c} и рис. \ref{ris:0-30}).

\begin{figure}[h]
\begin{center}
\includegraphics[width=14cm]{0-9.jpg}
\end{center}
\caption{\!\!\!\!:\ \emph{a} --- толщина пленки 0,90 мм; \emph{b} --- толщина пленки 0,75 мм; \emph{c} --- толщина пленки 0,56~мм}
\label{ris:0-9}
\end{figure}


Использование красного лазера ($\lambda=655$ нм) не принесло лучшего результата --- фотографии получались более размытыми, чем те, которые получались при использовании зеленого лазера
($\lambda=532$ нм).


Кроме того, стоит также учесть качество самих пленок --- их отражательную способность, которая размывала картинку. Это особенно ярко выражено для блестящей пленки толщиной 0,30 мм (рис. \ref{ris:0-30}, \emph{a}).

\begin{figure}[h]
\begin{center}
\includegraphics[width=14cm]{0-3.jpg}
\end{center}
\caption{\!\!\!\!:\ \emph{a} --- толщина пленки 0,30 мм; \emph{b} --- толщина пленки 0,23 мм; \emph{c} --- толщина пленки 0,50~мм}
\label{ris:0-30}
\end{figure}

Найменьшее значение для толщины, при котором еще можно разрешить 2 точки на фотоматрице, в опытах было равно 0,23 мм (рис. \ref{ris:0-30}, \emph{b}).


При соосном расположении еще одной линзы перед камерой (с фокусным расстоянием 18 см) и использовании фиолетового лазера
($\lambda=405$ нм) удалось получить более высокое разрешение точек для 0,23 мм. Однако это не позволило добиться желаемого результата в опытах с использованием пленки толщиной 0,1 мм. С уменьшением диаметра пучка на поверхности можно добиться разрешения фиксируемых световых точек и для толщин < 0,23 мм при использовании более коротковолнового излучения и соответствующего приемника. Однако это значительно усложнит установку и ее конечную стоимость. Кроме того, применение ультрафиолетового излучения и внедрение технологий с его использованием сдерживается из-за отсутствия отдаленных данных по действию коротковолнового УФ-излучения на окружающие ткани, а также по перестройке структуры роговицы после ее заживления \cite{tuchini}.


Стоит отметить, что при условии получения несфокусированного фото, световые пятна по-прежнему могут быть различены
(рис. \ref{ris:0-30}, \emph{c}), поэтому фотоприемник не обязательно устанавливать под углом Шемпфлюта. Данный вывод коррелирует с результатами работы \cite{burkov}. Это позволяет проводить измерения даже в отсутствии точной фокусировки на поверхности исследуемой пленки, что ускоряет процесс измерения.

\section*{Заключение}
В данной работе был апробирован метод лазерной триангуляции для измерения толщин тонких пленок и получены следующие результаты:

\begin{enumerate}
\item Выведена формула, связывающая толщину пленки и расстояние между световыми пучками, которые фокусируются на приемнике.
\item Сделана оценка для минимально возможной толщины пленки, которую можно измерить в данном опыте. Численное значение минимальной толщины пленки при данных параметрах установки лежит в окрестности 0,1 мм.
\item Наименьшее значение толщины, полученное в опыте, для данного метода и при данных параметрах установки составляет 0,23 мм.
\end{enumerate}

Также стоит отметить, что наиболее эффективным при измерении с использованием в качестве приемника излучения ПЗС-камеры будет лазер зеленого света. Качество измерений в данном методе будет зависеть от отражательных свойств самой поверхности. Pасфокусировка не особо будет влиять на точность измерений, поэтому фотоприемник не обязательно устанавливать п система является достаточно РіСЂРѕРјРѕР·РґРєРѕР№, дорогостоящей Рё весьма чувствительной Рє внешнему воздействию повреждающих факторов. Рљ примеру, механические вибрации РјРѕРіСѓС‚ привести Рє смещению оптических осей, формирующих плоский фронт линз РІ данной системе, что приведет Рє увеличению диаметра пучка РЅР° поверхностях пленки Рё РІРѕРґС‹.


Кроме того, если вычислить погрешность для толщины измеряемой пленки по формуле для косвенных воспроизводимых измерений \cite{kassandrova}, при $\Delta n = 0,5$, $\Delta (\Delta x)=0,005$ мм, $\Delta \alpha=\Delta \alpha_1 = 0,008$ рад, $\Delta M=0,1$:

\begin{equation}
\Delta (\Delta z)=\sqrt{\sum_{i=1}^n \left(\Delta y_i \frac{\partial G}{\partial y_i}\right)^2} \approx 0,02 \, \text{РјРј}.
\end{equation}

Следовательно, реально достижимое значение для наименьшей толщины, которую можно будет измерить этим методом при данных параметрах, будет лежать в окрестности 0,1 мм.

\section{Экспериментальные исследования}

Апробация метода лазерной триангуляции проводилась с помощью установки, схема которой представлена на рисунке \ref{ris:device}.

\begin{figure}[h]
\begin{center}
\includegraphics[width=11cm]{device.jpg}
\end{center}\vspace*{-3mm}
\caption{Схема экспериментальной установки}
\label{ris:device}
\end{figure}

Максимальное разрешение (соответственно определение минимальной толщины пленки) РІ методе лазерной триангуляции возможно РїСЂРё выполнении условия: диаметр перетяжки должен быть минимальным РЅР° поверхности исследуемой пленки \cite{andreev, burkov, barishnikov}. Р’~данной работе использовался полупроводниковый лазер, фронт волны которого весьма сильно отличается РѕС‚ плоского, Р° распределение лазерного пучка является колоколообразным, РЅРѕ РЅРµ гауссовым. Р