%\documentclass[a4paper,11pt,twoside]{article}
%\usepgfplotslibrary{patchplots}
%\usepackage{pgfplots}
%\usepackage{tikz}
%\usetikzlibrary{arrows.meta}
%\usepackage{enumitem}
%\usepackage{longtable}
%\usepackage{lscape}
%\usepackage{vvmph2}%Этот пакет автоматически подключает
%пакеты: babel с опциями english, russian,
% а также amsmath, amsthm, graphicx и pscyr -
%подключать их еще раз не нужно.
%далее вы можете подключить другие необходимые Вам пакеты,
%но среди них обязательно должен быть следующий:
%\usepackage{rumathbr}%\usepgfplotslibrary{patchplots}

%\setcounter{MaxMatrixCols}{25}
%\pgfplotsset{width=6cm,compat=1.5.1}

%\begin{document}
%\newtheorem{stm}{Утверждение}[section]
%\newtheorem{rem}{Замечание}[section]
%\newtheorem{exm}{Пример}[section]
\udk{519.6}
\bbk{22.19 + 20.1}


\art[Оптимизация и сценарно-имитационное моделирование динамики структуры]{Оптимизация и сценарно-имитационное моделирование динамики структуры малых искусственных русел пойменных территорий}{
Optimization and scenario-simulation modelling\plb of the dynamics of the structure of small artificial channels\plb in floodplain areas}
\support{Работа выполнена при финансовой поддержке РФФИ и Администрации Волгоградской области в рамках научного проекта № 16-48-340147} %команда может отсутствовать
\fio{Бакулин}{Владислав}{Сергеевич}
%\sdeg{доктор физико-математических наук}{Doctor of Science} %если есть
\pos{студент кафедры фундаментальной информатики и оптимального управления}{Student,\\ Department of Fundamental Informatics and Optimal Control}
\emp{Волгоградский государственный университет}{Volgograd State University}
\email{gitara_93@mail.ru, fiou@volsu.ru}
\addr{просп. Университетский, 100, 400062 г. Волгоград, Российская Федерация}{Prosp. Universitetsky, 100, 400062 Volgograd, Russian Federation\pagebreak}
\fio{Васильченко}{Анна}{Анатольевна}
\pos{старший преподаватель кафедры фундаментальной информатики\\ и оптимального управления}{Senior Lecturer, Department of Fundamental Informatics and Optimal Control}
\emp{Волгоградский государственный университет}{Volgograd State University}
\email{aa-vasilchenko@mail.ru, fiou@volsu.ru}
\addr{просп. Университетский, 100, 400062 г. Волгоград, Российская Федерация
}{\vspace*{-0.1mm}Prosp. Universitetsky, 100, 400062 Volgograd, Russian Federation}
\fio{Воронин}{Александр}{Александрович}
\sdeg{доктор физико-математических наук}{Doctor of Physical and Mathematical Sciences} %если есть
\pos{профессор, заведующий кафедрой фундаментальной информатики и оптимального управления}{professor, 	
Head of Department of Fundamental Informatics and Optimal Control}
\emp{Волгоградский государственный университет}{Volgograd State University}
\email{voronin.prof@gmail.com, fiou@volsu.ru}
\addr{просп. Университетский, 100, 400062 г. Волгоград, Российская Федерация}{\vspace*{-0.1mm}Prosp. Universitetsky, 100, 400062 Volgograd, Russian Federation}
\fio{Харитонов}{Михаил}{Алексеевич}
%\sdeg{ученая степень}{academic degree in English} %если есть
\pos{младший научный сотрудник кафедры фундаментальной информатики\\ и оптимального управления}{junior researcher, Department of Fundamental Informatics and Optimal Control}
\emp{Волгоградский государственный университет}{Volgograd State University}
\email{kharitonov.mihail@gmail.com, fiou@volsu.ru}
\addr{просп. Университетский, 100, 400062 г. Волгоград, Российская Федерация}{Prosp. Universitetsky, 100, 400062 Volgograd, Russian Federation}
\maketitle
\begin{abstract}
Целью  настоящей работы является исследование свойств оптимальности локальных искусственных русловых систем пойменных территорий. Предложен метод  поиска  топологической структуры и площади поперечных сечений искусственных русел, максимизирующих суммарную взвешенную площадь затапливаемой территории.  Для адекватности постановки и решения задачи управления проведено адаптивное зонирование цифровой модели территории поймы на участки с незначительным эффектом межзонного затопления при малых и средних паводках. Оптимальная пропускная способность малых русел разыскивалась с помощью многошагового рекурсивного метода динамического программирования. Представлены результаты  численной реализации алгоритма с использованием компьютерных гидродинамических имитаций паводковой динамики и геоинформационным моделированием  пространственно распределенных параметров  рельефа нескольких участков территории северной части Волго"=Ахтубинской поймы с искусственными локальными  русловыми системами.

Построен взвешенный ориентированный граф межфакторных взаимодействий, определяющих паводковую динамику  северной части Волго"=Ахтубинской поймы, при помощи которого проведена формализация  сценариев развития ситуации и динамики параметров, непосредственно определяющих оптимальную локальную русловую структуру. Приведены субъективные оценки вероятности сценариев и нечеткие оценки динамики параметров русловой структуры.
\end{abstract}

\keywords{структурная оптимизация, динамическое программирование, производственная функция, геоинформационное моделирование, гидродинамическое моделирование, системы с потоками}
{structural optimization, dynamic programming, production fun\-c\-tion, geoinformation modeling, hydrodynamic modeling, flow systems}

\section*{Введение} % Введение обычно не нумеруется, потому стоит звездочка
Волго-Ахтубинская пойма (ВАП), расположенная в нижнем течении р. Волги, имеет протяженность около 450 км, площадь ее территории превышает 20 тыс. кв. км.  На ее территории расположено несколько тысяч малых русел, через которые происходит весеннее паводковое затопление.  В северной части ВАП расположены природный парк с ценными представителями флоры и фауны, значительное число населенных пунктов, ведется сельскохозяйственный, лесной и рекреационный бизнес. Жизнь  экосистемы ВАП полностью определяется весенним паводковым режимом р. Волги, который с 1959 г. регулируется Волжской ГЭС (ВГЭС)  и поэтому существенно отличается от природного. В~последние десятилетия наблюдается деградация пойменной экосистемы ВАП вследствие прогрессирующего снижения уровня паводкового затопления~\cite{VorVas-Zemlanov,VorVas-Ivanov}.

Одной из мер противодействия природной деградации ВАП является повышение эффективности территориального распределения паводковых вод путем строительства каналов с управляемой структурой и поперечным сечением русел.  В данной работе предложен метод оптимизации  структуры локальных искусственных русловых систем пойменных территорий, а также представлены результаты его численной реализации для нескольких участков территории ВАП при использовании компьютерных гидродинамических имитаций паводковой динамики с геоинформационным моделированием  пространственно распределенных параметров  русловой системы. Проанализированы варианты динамики оптимальной структуры русел при различных сценариях изменения системных параметров. Использованная в работе цифровая модель рельефа (ЦМР) ВАП и численная гидродинамическая модель  (программно-алгоритмический комплекс <<Web-ЭКОГИС>>) описаны в~\cite{VorVas-Khrap1,VorVas-Voronin2,VorVas-Khop,VorVas-Khrap,VorVas-Khrap2}.

\section{Системный анализ и имитационное моделирование динамики паводковых вод на территории северной части ВАП}\label{VorVas-S1}




Анализ космических снимков и результатов гидродинамических расчетов~\cite{VorVas-Khop, VorVas-Khrap} показал, что площадь затопления территории определяется величиной постоянного расхода  $Q$ и длительностью $t$ первой фазы паводка, средние  значения которых за последние 30 лет равны соответственно $25\,000\,\emph{м}^3/c$ и 10 суток. Карта территории паводкового затопления северной части ВАП при $Q =25\,000\,\emph{м}^3/c$  и $t= 10$ суток  представлена на рисунке~\ref{ris:vorvas-1}.

 \begin{figure}[h!]
 \begin{center}
 \begin{minipage}{.75\linewidth}
 \includegraphics[width=1.\linewidth]{vorvas-1.pdf}
 \caption{Распределение воды на территории ВАП при $Q =25\,000\,\emph{м}^3/c$\\ и $t= 10$ суток}
 \label{ris:vorvas-1}\end{minipage}
 \end{center}
 \end{figure}

Топологическая структура русел северной части ВАП представима в виде бинарного дерева с многими десятками уровней ветвления. Однако существенное различие в величине и характере расходов воды в ветвях этого дерева обусловливают представление структуры с учетом способа ее функционирования и существа решаемой задачи управления. С учетом этого  будем считать, что первый (высший) уровень иерархической гидрологической структуры территории северной части ВАП образуют три магистральных водотока (р. Волга, р. Ахтуба и ерик <<Пахотный>>). Второй --- связанные с ними  58 русел глубиной 2--2,5 м, образующих локальные русловые системы. Третий уровень образуют русла глубиной 1,5--2 м, связанные с руслами второго уровня.  Русла четвертого уровня  меньшей глубины имеют небольшое число локальных русловых систем.

Такая гидрологическая структура создает возможность разбиения территории на зоны преимущественно независимого затопления локальными русловыми системами или даже отдельными ее руслами при малых паводках или в их начальных фазах. Мера погрешности такого затопления, растущая вместе с ростом числа зон, характеризуя потенциальную управляемость процесса паводкового затопления, ставит тем самым предел сложности системе пространственно распределенного управления.

Границы зон, минимизирующие погрешности межзонного затопления при заданных параметрах паводка, находились в ходе гидродинамического и геоинформационного моделирования.  Для построения нулевого приближения границ зон на основе иерархической цифровой модели русловой структуры использовался модифицированный алгоритм Вороного~\cite{VorVas-Karimipour}. Последующее адаптивное <<квазинепрерывное>> изменение границ  проводилось в ходе анализа временного ряда матриц цифровых карт высот водного слоя   при  $Q =25\,000\,\emph{м}^3/c$.  В случае превышения предельно допустимой объемной доли трансграничных водных потоков  производилось объединение соответствующих зон. Результат моделирования представлен в таблице~\ref{tabular:vorvas-1}.

\vspace*{5mm}
 \begin{table}[h!]\small
    \caption{Зависимость параметров зон территории ВАП от величины предельно допустимой объемной доли трансграничных потоков  при $Q =25000\, \emph{м}^3/c$, $t=5\cdot10^5 \,c$ }
    \label{tabular:vorvas-1}
\begin{center}
\begin{tabular}{|c|c|c|c|}
  \hline
  % after \\: \hline or \cline{col1-col2} \cline{col3-col4} ...
 \spc{Предельно допустимая\\ объемная доля
 межзонных\\ расходов паводковых вод (\%)}& \spc{Число зон\\ ($n$)} & \spc{Средняя\\ площадь зоны\\ ($\emph{км}^2$)} & \spc{Средний расход\\ воды через главное\\ русло зоны ($\emph{м}^3/c$)} \\  \hline
  20 & 54 & 16,1 & 55,6 \\  \hline
  15 & 41 & 21,1 & 73,2 \\  \hline
  10 & 32 & 27,1  & 93,8 \\  \hline
  5 & 28 & 30,1 & 107,1 \\  \hline
  2 & 17 & 51,0 & 176,5 \\  \hline
\end{tabular}
\end{center}
\end{table}


 Зависимость числа зон от времени затопления при предельно допустимой объемной доле трансграничных водных потоков  5\,\% представлена в таблице~\ref{tabular:vorvas-2}.


\vspace*{5mm}
\begin{table}[h!]\small
    \caption{Зависимость числа зон от времени затопления при предельно допустимой объемной доле трансграничных водных потоков  5\,\%}
    \label{tabular:vorvas-2}
\begin{center}
\begin{tabular}{|p{1.1in}|c|c|c|c|c|c|}
\hline
$t\times 10^5$ с  & 2 & 4& 6& 8&10&12\\\hline
Число зон               &58 & 49& 27& 24& 19& 11\\\hline
\end{tabular}
\end{center}
\end{table}

%\begin{figure}
%\begin{center}
%\begin{tikzpicture}
%\begin{axis}%[xlabel=$t,c$, ylabel=$\emph{число зон}$]
%\addplot+[const plot] coordinates{(200000,58) (400000, 49)  (600000, 27)  (800000,24)(1000000, 19)(1200000, 11)};
%\end{axis}
%\end{tikzpicture}
%\caption{Зависимость числа зон от времени затопления при предельно допустимой объемной доле трансграничных водных потоков  5\%.}
%  \label{ris:n(t)}
%\end{center}
%\end{figure}


%\begin{figure}[h!]
%\begin{center}
%\begin{tikzpicture}
%\begin{axis}[width=5.2cm,xlabel=$t$,ylabel=$\lambda$,title=a)]
%\addplot +[const plot]  coordinates{(200000,58) (400000, 49)  (600000, 27)  (800000,24)(1000000, 19)(1200000, 11)};
%\end{axis}
%\end{tikzpicture}
%\caption{Зависимость числа зон от времени затопления при предельно допустимой объемной доле трансграничных водных потоков  5\%.}
%  \label{ris:n(t)}
%\end{center}
%\end{figure}

\vspace*{-4mm}
Как показало  компьютерное гидродинамическое имитационное моделирование динамики паводковых вод на территории  северной части ВАП, непосредственное затопление большей части зон при малых паводках осуществляется из русел третьего иерархического уровня при несущественных трансграничных эффектах. С другой стороны, характер затопления территории при больших паводках характеризуются низкой чувствительностью к структурным параметрам русел. Поэтому управление пропускной способностью русел для максимизации общей площади затапливаемой территории актуально и потенциально осуществимо в большей части зон при значениях параметров первой фазы паводка $Q$, $t$  и меньших, чем их средние величине. При этом  общая площадь затопленной территории является суммой (взвешенной суммой) площадей независимых зонных затоплений, расчитываемых по результатам численного гидродинамического моделирования.

Расчет расходов паводковых вод на элементах локальных русловых структур показал примерно равное их распределение между боковыми руслами третьего уровня при их числе от 2 до 4.



\section{Задача оптимизации структуры искусственных малых русел\\ пойменной территории с управляемым поперечным сечением}\label{VorVas-S2}

Представленные в п.~\ref{VorVas-S1} результаты имитационного моделирования динамики паводковых вод в ВАП позволяют сделать вывод о потенциальной возможности управления их пространственным распределением в некоторой области параметров паводка при  адаптивном зонировании территории с учетом локальных особенностей рельефа и русловой структуры. Однако структура локальных русловых систем ВАП, сформированная в условиях природных паводков, может быть неоптимальной в изменившихся за последние десятилетия паводковых условиях. Поэтому представляет интерес поиск зависимости оптимальной структуры локальных русловых систем от  параметров паводка и рельефа территории. Для исследования этой проблемы представим следующую формальную задачу оптимизации структуры искусственных малых русел пойменной территории с управляемым поперечным сечением.

Модель пойменной территории (квадрат со стороной $L$) описывается цифровой моделью рельефа, представленной сеточной функцией $b(i,j)_1^N$ высот. На территории задана серия $(m,n)$-разбиений:  $n$-кратных рекурсивных дроблений на $m$  равных частей, дающих в результате $m^n$ одинаковых участков (зон). Каждому  $(m,n)$-разбиению территории соответствует  структура (дерево) искусственных русел, схематично представленная для нескольких случаев на рисунке \ref{ris:vorvas-5}.

 \begin{figure}[h!]
 \begin{center}
 \begin{minipage}{.85\linewidth}
 \includegraphics[width=1.\linewidth]{vorvas-5.pdf}
 \caption{\!\!\!\!:\ \emph{а} --- (3,1)-дерево; \emph{б} --- (3,2)-дерево; \emph{в} --- (3,3)-дерево}
 \label{ris:vorvas-5}\end{minipage}
 \end{center}
 \end{figure}


Объем паводковых вод  $V_0$, поступивших на территорию за время  $t$, определяется формулой $V_0=Q_0 t$,  где  $Q_0$    --- расход воды через входное русло территории.
Обозначим  $\bar{V}$  объем вод, локализованных в руслах.  Для объемов вод в зонах справедливы балансовые соотношения

 \begin{equation*}
 V_0=V_{i_{0}}-\bar{V}, V_{i_{q-k}}=\sum\limits_{j=1}^m V_{i_{q-k}j}, q=\overline{n,1},k=\overline{1,q}.
\end{equation*}

Здесь и далее комплексный индекс $i_1 i_2 \ldots i_n$ ($i_k=\overline{1,m}$, $k=\overline{1,n}$)  для краткости записывается как  $i_n$. Площадь затопленной территории каждой из зон описывается вогнутыми функциями $S^{zat}_{i_n}=\varphi_{i_n} (V_{i_n})$, полученными путем кусочно-линейных аппроксимаций временных  рядов $S^{zat}_{i_n} (t_k)$,
$V^{zat}_{i_n} (t_k)$, полученных агрегированием соответствующих участков массива карт затоплений $K(t_k)$, $t_k=10^3 k$, $k=\overline{1,500}$, полученных в ходе численного гидродинамического моделирования (см. п.~\ref{VorVas-S1}). С учетом результатов п.~\ref{VorVas-S1} расходы воды в русловой системе можно задать формулами  $Q_{i_{n-k},j}=\frac{Q_{i_{n-k}}}{m}$  ($j=\overline{1,m}$, $k=\overline{1,n}$). Управлениями пропускной способностью русла с площадью сечения $S^{sech}_0$ будем считать доли открытых русел $u\in [0;1]$, так что $S^{sech}=u S^{sech}_0$ . Тогда

  \begin{equation*}
    Q_{i_{n-k},j}=\frac{Q_{in}u_j}{U}=\frac{V_{in-k}}{T}\cdot\frac{u_j}{U},\ U=\sum\limits_{i=1}^m u_i.
  \end{equation*}

В каждой зоне задан коэффициент относительной экологической ценности $\alpha_{ij}=\frac{a_{ij}}{\sum\limits_{i,j=\overline{1,N}} a_{ij}}$, ($a_{ij}=\overline{0,a_{max}}$).

Задача управления водными потоками в русловой  $(m,n)$-структуре имеет вид
\begin{multline}\label{eq:vorvas-maxn}
\varphi_{i_{n-1}}(V_{i_{n-1}})=\sum\limits_{j=1}^m \alpha_{i_{n-1}j}\varphi_{i_{n-1}j}(V_{i_{n-1}j})\to \\
\to \max\limits_{\{V_{i_{n-1}j}\}_{j=1}^m} \sum\limits_{j=1}^m V_{i_{n-1},j}=V_{i_{n-1}}, V_{i_{n-1}j} \in\left[0, \frac{2V_{i_0}}{m^{n}}\right], \forall i_{n-1}.
\end{multline}
\begin{multline}\label{eq:vorvas-maxk}
\varphi_{i_{n-k}}(V_{i_{n-k}})=\sum\limits_{j=1}^m \alpha_{i_{n-k}j}\varphi_{i_{n-k}j}(V_{i_{n-k}j})\to  \\
\to \max\limits_{V_{i_{n-k}j}}  \sum\limits_{j=1}^m V_{i_{n-k},j}=V_{i_{n-k}}, V_{i_{n-k}j} \in\left[0, \frac{2V_{i_0}}{m^{n-k+1}}\right], \forall i_{n-k},k=\overline{2,n}.
\end{multline}


Решение  задачи~(\ref{eq:vorvas-maxn})--(\ref{eq:vorvas-maxk}), которое далее находится рекурсивным (по глубине  $(m,n)$-дерева) численным многошаговым (дискретизацией величин объема паводковых вод) методом динамического программирования (рекурсией по ширине  $(m,n)$-дерева), обозначим функциями

 $\{V^*_{i_{q-k}j} (V_{i_{q-k-1}})\}$, $\{\varphi^*_{i_{q-k}j} (V_{i_{q-k-1}})\}$, $\varphi^*_{i_{q-k}}(V_{i_{q-k}})=\max\limits_{(V_{i_{q-k}j})} \varphi^*_{i_{q-k}j} (V_{i_{q-k}j})$.
 Оптимальное управление $u^*_{i_{q-k}j} (V_{i_{q-k-1}})$ определяется следующими формулами:

 \begin{multline}\label{eq:vorvas-j}
j^*=\arg\max  V^*_{i_{q-k}j},  V^*_{i_{n-1},j}=V_{i_{n-1}}\cdot \frac{u^*_{n-1,j}}{\sum\limits_{j=1}^m, u^*_{n-1,j}}, \\
u^*_{i_{n-1},j^*}=1, u^*_{n-1,j}=\frac{V^*_{i_{n-1},j}}{V^*_{i_{n-1},j^*}}, j=\overline{1,m}.
 \end{multline}

Оптимальные значения   $m^*$ и  $n^*$ находятся методом полного перебора.


Численное решение этой задачи проводилось в рамках серии численных гидродинамических имитаций паводкового затопления пяти  ВАП  с искусственными   $(m,n)$-системами русел при $m=2,3,4$ и $m=1,2,3$ и  следующими выбранными на основании результатов п.~\ref{VorVas-S1} характеристиками параметров паводка и  территории: размера  (квадрат со стороной $L=5\,\emph{км}$), $Q_0=100\,\emph{м}^3/c$, $t=(0-100)\times 10^4 c$   при уклоне русел $0,03^\circ$,  и следующих значениях  диаметров $(d(n))$  и глубин $(h(n))$ русел:
$d(1)=50\,\emph{м}$, $h(1)=3\,\emph{м}$, $d(2)=30\,\emph{м}$, $h(2)=2\,\emph{м}$,  $d(3)=10\,\emph{м}$, $h(3)=1,5\,\emph{м}$.
Коэффициенты паводковой неоднородности каждой из $m^n$ зон выбирались случайно в диапазоне  $\overline{1,a_{max}}$.


%
%\begin{figure}
%\begin{center}
%\begin{tikzpicture}
%%\pgfplotsset{
%%every axis legend/.append style={
%%at={(0.5,1.03)},
%%anchor=south
%%},%$A: (m^*, n^*)=(2;1)$, $B: (m^*, n^*)=(2;2)$, $C: (m^*, n^*)=(2;3)$, $D: (m^*, n^*)=(3;2)$
%%}
%\begin{axis}[legend pos=outer north east,
%xlabel=$V_0\times 10^7 (\emph{м}^3)$,ylabel=$\varphi_{i_0}\times 10^6 (\emph{м}^2)$,legend entries={$(2;1)$, $(2;2)$, $(2;3)$, $(3;2)$},
%]
%%\addplot[ blue, mark=none, domain=0:6.2,samples=400,line width=1.0pt ]  {3*x^(0.37)};
%%\addplot [ red, mark=none, domain=0:6.2,samples=400,line width=1.0pt, loosely dashdotdotted ]  {2.65*x^(0.49)};
%\addplot[ blue, smooth, mark=none, line width=1.0pt ] coordinates {(0.00, 0.0000)(0.05, 0.9902)(0.10, 1.2797)(0.15, 1.4869)(0.20, 1.6539)(0.25, 1.7962)(0.30, 1.9216)(0.35, 2.0344)(0.40, 2.1374)(0.45, 2.2326)(0.50, 2.3213)(0.55, 2.4047)(0.60, 2.4833)(0.65, 2.5580)(0.70, 2.6291)(0.75, 2.6971)(0.80, 2.7623)(0.85, 2.8249)(0.90, 2.8853)(0.95, 2.9436)(1.00, 3.0000)(1.05, 3.0546)(1.10, 3.1077)(1.15, 3.1592)(1.20, 3.2094)(1.25, 3.2582)(1.30, 3.3058)(1.35, 3.3523)(1.40, 3.3977)(1.45, 3.4421)(1.50, 3.4856)(1.55, 3.5281)(1.60, 3.5698)(1.65, 3.6107)(1.70, 3.6508)(1.75, 3.6902)(1.80, 3.7288)(1.85, 3.7668)(1.90, 3.8042)(1.95, 3.8409)(2.00, 3.8771)(2.05, 3.9126)(2.10, 3.9477)(2.15, 3.9822)(2.20, 4.0162)(2.25, 4.0498)(2.30, 4.0828)(2.35, 4.1154)(2.40, 4.1476)(2.45, 4.1794)(2.50, 4.2107)(2.55, 4.2417)(2.60, 4.2723)(2.65, 4.3025)(2.70, 4.3324)(2.75, 4.3619)(2.80, 4.3911)(2.85, 4.4199)(2.90, 4.4484)(2.95, 4.4767)(3.00, 4.5046)(3.05, 4.5322)(3.10, 4.5596)(3.15, 4.5867)(3.20, 4.6135)(3.25, 4.6400)(3.30, 4.6663)(3.35, 4.6923)(3.40, 4.7181)(3.45, 4.7437)(3.50, 4.7690)(3.55, 4.7941)(3.60, 4.8190)(3.65, 4.8436)(3.70, 4.8681)(3.75, 4.8923)(3.80, 4.9163)(3.85, 4.9402)(3.90, 4.9638)(3.95, 4.9873)(4.00, 5.0105)(4.05, 5.0336)(4.10, 5.0565)(4.15, 5.0792)(4.20, 5.1018)(4.25, 5.1242)(4.30, 5.1464)(4.35, 5.1685)(4.40, 5.1904)(4.45, 5.2121)(4.50, 5.2337)(4.55, 5.2552)(4.60, 5.2764)(4.65, 5.2976)(4.70, 5.3186)(4.75, 5.3395)(4.80, 5.3602)(4.85, 5.3808)(4.90, 5.4012)(4.95, 5.4216)(5.00, 5.4418)(5.05, 5.4618)(5.10, 5.4818)(5.15, 5.5016)(5.20, 5.5213)(5.25, 5.5409)(5.30, 5.5604)(5.35, 5.5797)(5.40, 5.5990)(5.45, 5.6181)(5.50, 5.6371)(5.55, 5.6560)(5.60, 5.6748)(5.65, 5.6935)(5.70, 5.7121)(5.75, 5.7306)(5.80, 5.7490)(5.85, 5.7673)(5.90, 5.7854)(5.95, 5.8035)(6.00, 5.8215)(6.05, 5.8394)(6.10, 5.8572)(6.15, 5.8750)(6.20, 5.8926)};
%
%\addplot [red, smooth, mark=none, line width=1.0pt ] coordinates {(0.00, 0.0000)(0.05, 0.6106)(0.10, 0.8575)(0.15, 1.0460)(0.20, 1.2043)(0.25, 1.3435)(0.30, 1.4690)(0.35, 1.5843)(0.40, 1.6914)(0.45, 1.7919)(0.50, 1.8869)(0.55, 1.9771)(0.60, 2.0632)(0.65, 2.1457)(0.70, 2.2251)(0.75, 2.3016)(0.80, 2.3755)(0.85, 2.4472)(0.90, 2.5167)(0.95, 2.5842)(1.00, 2.6500)(1.05, 2.7141)(1.10, 2.7767)(1.15, 2.83