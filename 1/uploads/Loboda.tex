%
%\documentclass[a4paper,11pt,twoside]{article}
%\usepackage{vvmph2}
%\usepackage{rumathbr}
%
%
%\begin{document}

\counterwithout{equation}{section}
\plang{0}\selectlanguage{russian}

\udk{517.55 + 514.74 + 004.94}
\bbk{22.161.5 + 22.151.5}
\art[Об аффинно-однородных вещественных гиперповерхностях]{Об аффинно-однородных вещественных гиперповерхностях общего положения в $\Bbb C^3$}{On Affine Homogeneous Real Hypersurfaces\plb of General position in $\Bbb C^3$}
\support{Работа частично поддержана грантом РФФИ № 14-01-00709} %команда может отсутствовать
\fio{Лобода}{Александр}{Васильевич}
\sdeg{доктор физико-математических наук}{doctor of Physical and Mathematical
Sciences} %если есть
\pos{профессор}{professor}
\emp{Воронежский государственный технический университет}
{Voronezh State Technical University }
\email{lobvgasu@yandex.ru}
\addr{ул. 20-летия Октября, 84, 394006 г. Воронеж, Российская Федерация}{20-letiya Oktyabrya St., 84, 394006 Voronezh, Russian Federation}

\fio{Шиповская}{Александра}{Владимировна}
\pos{аспирант}{Postgraduate Student}
\emp{Воронежский государственный технический университет}
{Voronezh State Technical University }
\email{al.shipovskaia@gmail.com}
\addr{ул. 20-летия Октября, 84, 394006 г. Воронеж, Российская Федерация}{20-letiya Oktyabrya St., 84, 394006 Voronezh, Russian Federation}

\maketitle

\begin{abstract}
  В статье развиваются различные подходы к задаче
описания аффинно-однородных вещественных гиперповерхностей 3-мерного
комплексного пространства. Описывается схема изучения задачи об однородности,
использующая канонические уравнения изучаемых многообразий и
матричные алгебры Ли.

   В рамках этой схемы,
позволившей ранее получить полные классификационные результаты для нескольких
типов однородных поверхностей, построены
примеры матричных алгебр Ли, отвечающих строго псевдовыпуклым
однородным поверхностям общего положения. Приведены также примеры
однородных многообразий этого типа.

  Доказана теорема об опорном наборе коэффициентов канонического уравнения,
определяющем любую однородную поверхность из изучаемого класса. За счет
компьютерного исследования большой системы полиномиальных уравнений,
являющегося частью
обсуждаемой схемы, получен вывод о запретах на однородность, связанный
с ограничениями на некоторые коэффициенты опорного набора.

\end{abstract}

\keywords{аффинное преобразование, вещественная гиперповерхность,
каноническое уравнение поверхности, однородное многообразие, алгебра Ли,
система полиномиальных уравнений, символьные вычисления}{affine transformation, real hypersurface, canonical equation of  surface,
homogeneous manifold, Lie algebra, system of polynomial equations,
symbolic calculations}

\section*{Введение}

  Важной задачей многомерного комплексного анализа является исследование
вещественных подмногообразий и, в частности, гиперповерхностей комплексных
пространств, однородных относительно
голоморфных преобразований. Однако после работы Э. Картана \cite{CartanHom}, получившего
полное описание
однородных вещественных гиперповерхностей двумерных комплексных пространств,
повышение размерности в этой задаче столкнулось с существенными трудностями.

  При этом описания отдельных семейств и примеров однородных гиперповерхностей
в 3-мерных
комплексных пространствах приводят (см.: \cite{Loboda03, FelsKaup, Beloshapka})
к аффинно-однородным многообразиям. В этой связи
можно считать естественной аффинную постановку рассматриваемой задачи.
При ее решении весьма эффективными оказываются символьные вычисления, использующие
понятие канонического
уравнения для изучаемых многообразий (см., например, \cite{EastwoodEzhov, LobodaKhod}).

Напомним, что уравнение
строго псевдовыпуклой (СПВ)
вещественно"=аналитической гиперповерхности $ M $ комплексного пространства
$ \Bbb C^3 $ может быть (локально) приведено аффинными преобразованиями к
виду (см. \cite{LobodaKhod}):
\begin{equation}
   v = |z_1|^2 + |z_2|^2 + \varepsilon_1 (z_1^2 + \bar z_1^2) +
   \varepsilon_2 (z_2^2 + \bar z_2^2) +\sum_{k+l+2m \ge 3} F_{klm} (z,\bar z) u^m.\label{kanonich1}
\end{equation}


  Здесь пара
неотрицательных чисел
$ (\varepsilon_1, \varepsilon_2) $ является аффинным инвариантом поверхности;
$u = Re\,w, \ v = Im\,w $, $ z=(z_1,z_2)$, а через $ F_{klm} $ обозначается однородный многочлен степени
$ k $ по переменной $ z=(z_1,z_2) $, степени $ l $ по $ \bar z $ и степени $ m $ по $ u $.

  С использованием канонических уравнений (\ref{kanonich1}) удается получать, например,
оценки размерностей аффинных групп Ли (и, соответственно,
связанных с ними алгебр Ли), транзитивно действующих на таких многообразиях
\cite{Loboda01}.
  Эти оценки, как и другие свойства
одно\-родных поверхностей, существенно
зависят от значений отдельных элементов пары
$ (\varepsilon_1, \varepsilon_2) $. В связи с обсуждаемой задачей об
однородности естественно выделить несколько
семейств (типов) СПВ-поверхностей, определяемых именно такими парами:

1) \ $ \varepsilon_{1} = 0, \varepsilon_{2} = 0;
\quad
2) \ \varepsilon_{1} = 0, \varepsilon_{2} = 1/2;
\quad
3) \ \varepsilon_{1} = 0 < \varepsilon_{2} \ne 1/2 $;

4) \ $ \varepsilon_{1} = 1/2, \varepsilon_{2} = 1/2;
\quad
5) \ \varepsilon_{1} = 1/2, \ 0 < \varepsilon_{2} \ne 1/2;
\quad
6) \ 0 < \varepsilon_{1} = \varepsilon_{2} \ne 1/2 $;

7)\ $ (\varepsilon_{1}, \varepsilon_{2}) $, \ где
\
$ ( 2 \varepsilon_{1} - 1 )\cdot( 2 \varepsilon_{2} - 1 )\cdot
   (\varepsilon_{1} - \varepsilon_{2} ) \ne 0
$.


  К настоящему времени получены и опубликованы
полные описания аффинно-одно\-родных поверхностей,
отвечающих первым четырем случаям из приведенного списка
\cite{AtanovLob,
LobNguen,
LobShip,
NguenTTZ,
AtanLobShip};
описание пятого типа готовится к печати.
   Авторы настоящей статьи признательны профессору А.А. Григорьяну, интерес которого
к этим исследованиям и их поддержка (см.: \cite{LobBiele1, LobBiele2}) во многом стимулировали
получение как упомянутых, так и излагаемых ниже результатов.


  Основной целью данной статьи является изучение аффинно-однородных
гиперповерхностей так называемого {\it общего положения}, удовлетворяющих условию п. 7), или,
в более точной форме, неравенству
\begin{equation}
   \varepsilon_1 \varepsilon_2 \left(\varepsilon_1 - \frac12 \right)
 \left(\varepsilon_2 - \frac12 \right) (\varepsilon_1 - \varepsilon_2) \ne 0.\label{nerav.2.0}
\end{equation}

  Отметим, что описания однородных поверхностей четырех из перечисленных
выше семи типов получены в рамках
общей схемы исследования аффинной однородности.
 Сама такая схема кратко обсуждается ниже, в \S\,1, а в \S\,2 приводится
большое количество примеров матричных алгебр Ли, полученных с помощью
этой схемы и отвечающих однородным поверхностям общего положения.

   В то же время уточним, что в случае общего положения эта схема
пока позволила получить
лишь большое количество примеров однородных поверхностей.
Для полного решения задачи классификации помимо ее использования
необходимы и другие подходы, использующие, например,
оценки размерности пространств модулей для семейств и подсемейств
изучаемых многообразий.
  В статье получена такая оценка при условии ненулевого многочлена веса 3
из канонического уравнения (теорема 1).

С использованием компьютерных алгоритмов и символьных вычислений доказано
(теорема 2), что свойство аффинной однородности поверхности общего положения
не совместимо с ее принадлежностью к одному из
четырех естественных подсемейств, определяемых коэффициентами уравнения (\ref{kanonich1}).



\section{Схема описания аффинно-однородных гиперповерхностей в $ \Bbb C^3 $ }

  Аффинную однородность гиперповерхности $ M \in \Bbb C^3 $ (вблизи некоторой ее точки)
везде в статье мы понимаем как существование
алгебры Ли $ {\cal{G}}(M) $ аффинных векторных полей, касательных к $ M $,
значения которых в обсуждаемой точке поверхности накрывают всю
касательную плоскость к этой поверхности.



  Все изучаемые однородные СПВ-гиперповерхности будут ниже задаваться
своими аффинными каноническими уравнениями вида (\ref{kanonich1}), а
точку, вблизи которой обсуждается однородность, мы считаем
началом координат пространства $ \Bbb C^3 $.
Алгебры Ли касательных
векторных полей к таким поверхностям мы будем представлять в матричной форме.
Принцип сопоставления полей и соответствующих им матриц обозначен ниже
\begin{equation}
Z=\left(  \begin{array}{cccc}
   \left( A_{1}z_{1}+A_{2}z_{2} + A_{3}w + p_1 \right)
          \frac{\partial }{\partial z_{1}} +\\
    \\
  + \left( B_{1}z_{1}+B_{2}z_{2} + B_{3}w + p_2 \right)
          \frac{\partial }{\partial z_{2}}+\\
   \\
  + \left( az_{1} + bz_{2} + cw + q \right)
        \frac{\partial }{\partial w}
  \end{array}
\right)
\sim
Z =
\left(
\begin{array}{cccc}
A_1 & A_2 & A_3 & p_1\\
B_1 & B_2 & B_3 & p_2\\
a & b & c & q\\
0 & 0 & 0 & 0
\end{array}
\right). \label{matrix_eq_1.1}
\end{equation}

  Нулевая четвертая строка добавляется к каждой
$ 3\times 4 $-матрице коэффициентов аффинного векторного поля для удобства оперирования с матрицами.
  Известно, что при таком изоморфизме алгебр Ли коммутатор (скобка)
$ [Z_1, Z_2] $
  векторных полей переходит в коммутатор (скобку)
$ Z_1\cdot Z_2  - Z_2\cdot Z_1 $
соответствующих $ 4\,\times\, 4 $-матриц (см., например,~\cite{EastwoodEzhov}) .


   Несложно показать на основе рассуждений, изложенных в
\cite{Loboda01}, что для однородных поверхностей общего положения
выполняется оценка
$
      5 \le dim_{\Bbb R}\,{\cal{G}}(M) \le 6.
$
  При этом размерность 6 достигается лишь для поверхностей, аффинно
эквивалентных квадрикам
\begin{equation}
    v = |z_1|^2 + |z_2|^2 + \varepsilon_1 (z_1^2 + \bar z_1^2) +
                             \varepsilon_2 (z_2^2 + \bar z_2^2). \label{quadrika_1}
\end{equation}
Поэтому мы рассматриваем ниже только поверхности с 5-мерными группами и
алгебрами Ли. Само условие 5-мерности означает, что из всех элементов
матриц (или, что то же самое, аффинных векторных полей)  вида (\ref{matrix_eq_1.1}),
отвечающих однородным поверхностям,
только сдвиговые параметры $ (p_1, p_2, q) $ являются свободными.
Мы будем называть их {\bf основными} параметрами в отличие от остальных
элементов (\ref{matrix_eq_1.1}).

  Согласно \cite{Loboda01}, у 5-мерной алгебры Ли аффинных векторных полей,
касательных к однородной поверхности $ M $ общего положения, имеется базис
следующего вида:
$$
E_1=
\left[ \begin {array}{cccc}
A1_1 & A2_1 & A3_1 & 1 \\
B1_1 & B2_1 & B3_1 & 0 \\
2i(1+2\varepsilon_1) & 0 & 2m_1 & 0 \\
0 & 0 & 0 & 0
\end{array}\right]
,\
E_2=
\left[ \begin {array}{cccc}
A1_2 & A2_2 & A3_2 & i \\
B1_2 & B2_2 & B3_2 & 0 \\
2 (1- 2\varepsilon_1) & 0 & 2 m_2 & 0 \\
0 & 0 & 0 & 0
\end{array}\right],
$$
\begin{equation}\label{bazis_3.1}
E_3=
\left[ \begin {array}{cccc}
A1_3 & A2_3 & A3_3 & 0 \\
B1_3 & B2_3 & B3_3 & 1 \\
0 & 2i(1+2\varepsilon_2) & 2m_3 & 0 \\
0 & 0 & 0 & 0
\end{array}\right]
,\
E_4=
\left[ \begin {array}{cccc}
A1_4 & A2_4 & A3_4 & 0 \\
B1_4 & B2_4 & B3_4 & i \\
0 & 2(1-2\varepsilon_2) & 2m_4 & 0 \\
0 & 0 & 0 & 0
\end{array}\right],
\end{equation}
$$
E_5=
\left[ \begin {array}{cccc}
A1_5 & A2_5 & A3_5 & 0 \\
B1_5 & B2_5 & B3_5 & 0 \\
0 & 0 & 2m_5 & 1 \\
0 & 0 & 0 & 0
\end{array}\right].
$$
    Здесь параметры $ \varepsilon_1, \varepsilon_2, m_1, m_2, m_3, m_4 $ являются вещественными,
а остальные (обозначенные буквами) элементы выписанных матриц являются,
вообще говоря, комплексными числами. При этом $ \varepsilon_1, \varepsilon_2 $
удовлетворяют неравенству (\ref{nerav.2.0}).

  Наличие специальной структуры у базисных матриц (\ref{bazis_3.1}) позволяет получать
содержательные выводы об алгебрах Ли с такими базисами за счет рассмотрения
коммутаторов (скобок)
$$
    [ E_k, E_j ] = E_k E_j - E_j E_k
$$
и использования свойства замкнутости алгебры Ли относительно
операции коммутирования.

  Первым и основным этапом упомянутой схемы является построение всех
возможных алгебр
с базисами вида (\ref{bazis_3.1}). На втором этапе для каждой такой алгебры нужно найти
ее (единственное) интегральное многообразие, проходящее через начало координат.
Технически для этого требуется решить систему из пяти
уравнений в частных производных, отвечающую
основному соотношению
\begin{equation}
Re\,(E_k(\Phi)_{|M}) \equiv 0 \label{osnovnoe_tozhdestvo}
\end{equation}
для пяти базисных векторных полей. В формуле (6) через $ \Phi $ обозначена определяющая
функция поверхности $M $, которую мы записываем, в соответствии с (1), в виде
$$
  \Phi = -v + |z_1|^2 + |z_2|^2 + \varepsilon_1 (z_1^2 + \bar z_1^2) +
   \varepsilon_2 (z_2^2 + \bar z_2^2) +\sum_{k+l+2m \ge 3} F_{klm} (z,\bar z) u^m.
$$

\begin{remn} Отметим, что две однородные поверхности, получаемые интегрированием
<<различных>> алгебр Ли, могут оказаться аффинно эквивалентными.
  В изученных случаях 1)--4) обозначенные вопросы не играли заметной
роли, но для поверхностей общего положения они становятся существенными, как
показано в \S\,3.
\end{remn}

\begin{remn}
   Схема описания аффинно-однородных поверхностей с использованием соответствующих
им матричных алгебр Ли была эффективно реализована в работе
\cite{DoubrovKomRab} для построения достаточно объемной (полной) классификации
однородных поверхностей в 3-мерном вещественном пространстве. Изучаемый нами случай
является гораздо более сложным как по количеству описываемых объектов, так и
по объему необходимых вычислений.
\end{remn}


  Остановимся подробнее на первом этапе обсуждаемой схемы применительно к
алгебрам с базисами вида (\ref{bazis_3.1}) (см. также \cite{Ship15, LobShip}).
 Рассмотрение
десяти скобок $ [E_k, E_j] $ для всех пар базисных матриц (\ref{bazis_3.1})
приводит к системе десяти матричных уравнений. Например, одно из таких
уравнений имеет вид
\begin{equation}
  [E_1,E_2] =r_1 E_1 + r_2 E_2 + r_3 E_3 + r_4 E_4 - 4 E_5, \label{eq_5.1}
\end{equation}
где $ r_1, ... , r_4 $ --- некоторые вещественные коэффициенты.

  Расписывая поэлементно матричные уравнения типа (\ref{eq_5.1}),
получаем большую систему скалярных уравнений, являющихся, вообще говоря,
квадратичными относительно матричных элементов исходных матриц  (\ref{bazis_3.1}).

  При этом вся обсуждаемая алгебра определяется
лишь первой четверкой базисных матриц  (\ref{bazis_3.1}), а наиболее
важную роль при таком подходе играют левые верхние $2\times2$-блоки
этих матриц. В самом деле, выделяя вещественные и
мнимые части интересующих нас матричных элементов
($ t_1= Re\,(E_1)_{1,1}, \ t_2= Im\,(E_1)_{1,1} $ и т. д.), представим такие
блоки в развернутой форме
\begin{equation}
e_1=
\left[ \begin {array}{cc}
t_1+it_2 & t_3+it_4 \\
t_5+it_6 & t_7+it_8
\end{array}\right],
\
e_2=
\left[ \begin {array}{cc}
t_9+it_{10} & t_{11}+it_{12} \\
t_{13}+it_{14} & t_{15}+it_{16}
\end{array}\right],
$$
$$
e_3=
\left[ \begin {array}{cc}
t_{17}+it_{18} & t_{19}+it_{20} \\
t_{21}+it_{22} & t_{23}+it_{24}
\end{array}\right],
\
e_4=
\left[ \begin {array}{cc}
t_{25}+it_{26} & t_{27}+it_{28} \\
t_{29}+it_{30} & t_{31}+it_{32}
\end{array}\right].
\end{equation}

  Тогда, например, коэффициент $ r_1 $ из формулы  (\ref{eq_5.1}) определяется равенством
$$
 r_1 = -t_2 + t_9,
$$
и аналогичные формулы справедливы для остальных коэффициентов из  (\ref{eq_5.1}) и для аналогичных
разложений остальных девяти скобок.

   Отметим еще, что четвертые столбцы всех матриц (\ref{bazis_3.1}) устроены очень просто.
В~силу этого в
упомянутой системе скалярных уравнений ее отдельные уравнения также
устроены проще остальных.
Например, в ней имеется подсистема уравнений,
линейных относительно набора из 36 вещественных
переменных, включающего помимо $ t_k \ (k = 1,..., 32) $ еще и (3,3)-элементы
$ m_1, m_2, m_3, m_4 $ первых четырех базисных матриц.

  Решение такой линейной подсистемы позволяет (например, в случае
поверхностей общего положения) уменьшить набор из 36 неизвестных до 16
переменных
\begin{equation}
   t_1, t_2, t_3, t_4, t_5, t_7, t_8, t_{11}, t_{19},t_{23}, t_{24}, t_{27}, m_1,m_2, m_3, m_4. \label{nabor_7.1}
\end{equation}


  Далее выписывается система уравнений, квадратичных (по существу)
относительно набора (\ref{nabor_7.1}). Эта система оказывается переопределенной, так как
содержит (в случае поверхностей общего положения) 50 уравнений относительно
16 неизвестных.

  В изученных случаях 2)--5) из введения построение всех решений
аналогичных систем квадратичных уравнений фактически приводило к списку всех
алгебр Ли требуемого вида. А само получение решений оказывалось задачей,
хотя и технически сложной, но реализуемой с помощью компьютерных алгоритмов.

  В случае однородных поверхностей общего положения ситуация с
решением системы квадратичных уравнений сильно усложняется.
 Итоговые 50 уравнений оказываются весьма объемными. С учетом вхождения в них
16 неизвестных, многие из этих уравнений содержат порядка 100 квадратичных слагаемых
(при максимально возможном количестве 135). Усугубляется ситуация тем, что
коэффициентами уравнений являются многочлены от пары $ (\varepsilon_1, \varepsilon_2) $,
и степени таких многочленов достигают 6.

   В итоге решить такую систему уравнений в полном объеме пока не удалось. В~то же
время анализ квадратичных уравнений полученной системы показывает, что все 16~входящих
в них переменных целесообразно разбить на две группы (по 8 переменных в каждой), например,
\begin{equation}
  \{ t_1, t_4, t_7, t_{11}, t_{19}, t_{24}, m_1, m_4
\} \
\mbox{ и } \
\{ t_2, t_3, t_5, t_8, t_{23}, t_{27}, m_2, m_3 \}. \label{nabor_8.1}
\end{equation}


  При этом ровно 25 уравнений из обсуждаемых 50 оказываются <<антидиагональными>>, то есть
билинейно зависящими от двух выделенных
групп переменных. Каждое уравнение из другой, <<диагональной>>, части системы распадается
в сумму двух
<<раздельных>> квадратичных форм: каждое из его слагаемых содержит квадратичные произведения
переменных только из одной или только из второй группы. <<Смешанных>> произведений диагональные
формы не содержат.

   Компьютерные исследования антидиагональной части полученной системы показали, что из 25
билинейных уравнений линейно независимы только 15. Тем самым, в случае однородных
поверхностей общего положения в обсуждаемой задаче можно выделить подзадачу исследования
(нахождения всех решений) системы 15 уравнений, билинейно зависящих от двух групп переменных
(содержащих по 8 неизвестных в каждой группе). Можно отметить, что задачи такого рода
обсуждаются в современной математической литературе (см., например, \cite{BilinSyst}). Однако авторам
не удалось найти действенный алгоритм для получения полного решения изучаемой задачи.

   Несколько семейств матричных алгебр Ли, отвечающих
аффинно-однородным поверхностям общего положения и полученных из частных решений системы
15 билинейных уравнений, приведены в следующем параграфе. Способ получения этих частных
решений связан с обращением в нуль восьмерки переменных, составляющих одну из двух групп
(\ref{nabor_8.1}). Ясно, что при этом все билинейные (антидиагональные) уравнения обратятся в нуль,
и останется изучить диагональную часть квадратичной системы. В~такой ситуации каждое
из оставшихся 25 уравнений зависит лишь от 8 переменных противоположной группы.

   Применение компьютерных алгоритмов к такой усеченной задаче (и в частности, отсечение
решений системы, которые не дают алгебр Ли) оказывается вполне
эффективным: в разделе 2  настоящей статьи приведен {\bf полный} список
матричных алгебр Ли, отвечающих обнулению
(по отдельности) каждой из двух восьмерок (\ref{nabor_8.1}). Комментарии об однородных поверхностях,
соответствующих построенным алгебрам, приведены в разделе 3.



\section{Примеры алгебр Ли, отвечающих поверхностям общего положения}

  В этом параграфе приведены базисы нескольких семейств 5-мерных
матричных алгебр Ли. Эти примеры построены вторым автором статьи по
описанной выше схеме.

  Решения, соответствующие обращению в нуль переменных из
первой группы (\ref{nabor_7.1}), имеют в нумерации алгебр дополнительный номер 1,
обнулению второй группы (\ref{nabor_7.1}) соответствует дополнительный номер 2.

  Для всех алгебр используются общие обозначения
$$
    \mu_k = 1 + 2 \varepsilon_k, \quad
    \nu_k = 1 - 2 \varepsilon_k, \quad k = 1,2.
$$

{\bf Семейство алгебр 2-1} \ ($ t_{24} \in \Bbb R $).
\begin{equation}
 E_1 =
\left[ \begin {array}{cccc} 0&0&0&1\\
0&0&0 &0\\
2\,i \mu_1 &0&0&0 \\
0&0&0&0\end {array} \right]
,\
E_2 =
\left[ \begin {array}{cccc}
0&0&0&i\\
0&0&0 &0\\
2\,\nu_1 &0&0&0 \\
0&0&0&0\end {array} \right]
,\
E_3 =
\left[ \begin {array}{cccc} 0&0&0&0\\
0&it_{{24}}&0 &1\\
0&2\,i \mu_2 &0&0 \\
0&0&0&0\end {array} \right]
,\
$$
$$
E_4 =
 \left[ \begin {array}{cccc} - t_{{24}}/{\mu_2} &0&0&0\\
0&-t_{{24}}&0&i\\
0&2\nu_2 &-2\,t_{{24}}/{\mu_2}&0 \\
0&0&0&0\end {array} \right]
,\
E_5 =
\left[ \begin {array}{cccc} 0&0&0&0\\
0&0&0 &0\\
0&0&0&1 \\
0&0&0&0\end {array} \right].
\end{equation}


{\bf Семейство алгебр 3-1} \ ($ t_{4} \in \Bbb R $).
\begin{equation}
E_1 =
 \left[ \begin еет вид
\begin{equation}
  [E_1,E_2] =r_1 E_1 + r_2 E_2 + r_3 E_3 + r_4 E_4 - 4 E_5, \label{eq_5.1}
\end{equation}
где $ r_1, ... , r_4 $ --- некоторые вещественные коэффициенты.

  Расписывая поэлементно матричные уравнения типа (\ref{eq_5.1}),
получаем большую систему скалярных уравнений, являющихся, вообще говоря,
квадратичными относительно матричных элементов исходных матриц  (\ref{bazis_3.1}).

  При этом вся обсуждаемая алгебра определяется
лишь первой четверкой базисных матриц  (\ref{bazis_3.1}), а наиболее
важную роль при таком подходе играют левые верхние $2\times2$-блоки
этих матриц. В самом деле, выделяя вещественные и
мнимые части интересующих нас матричных элементов
($ t_1= Re\,(E_1)_{1,1}, \ t_2= Im\,(E_1)_{1,1} $ и т. д.), представим такие
блоки в развернутой форме
\begin{equation}
e_1=
\left[ \begin {array}{cc}
t_1+it_2 & t_3+it_4 \\
t_5+it_6 & t_7+it_8
\end{array}\right],
\
e_2=
\left[ \begin {array}{cc}
t_9+it_{10} & t_{11}+it_{12} \\
t_{13}+it_{14} & t_{15}+it_{16}
\end{array}\right],
$$
$$
e_3=
\left[ \begin {array}{cc}
t_{17}+it_{18} & t_{19}+it_{20} \\
t_{21}+it_{22} & t_{23}+it_{24}
\end{array}\right],
\
e_4=
\left[ \begin {array}{cc}
t_{25}+it_{26} & t_{27}+it_{28} \\
t_{29}+it_{30} & t_{31}+it_{32}
\end{array}\right].
\end{equation}

  Тогда, например, коэффициент $ r_1 $ из формулы  (\ref{eq_5.1}) определяется равенством
$$
 r_1 = -t_2 + t_9,
$$
и аналогичные формулы справедливы для остальных коэффициентов из  (\ref{eq_5.1}) и для аналогичных
разложений остальных девяти скобок.

   Отметим еще, что четвертые столбцы всех матриц (\ref{bazis_3.1}) устроены очень просто.
В~силу этого в
упомянутой системе скалярных уравнений ее отдельные уравнения также
устроены проще остальных.
Например, в ней имеется подсистема уравнений,
линейных относительно набора из 36 вещественных
переменных, включающего помимо $ t_k \ (k = 1,..., 32) $ еще и (3,3)-элементы
$ m_1, m_2, m_3, m_4 $ первых четырех базисных матриц.

  Решение такой линейной подсистемы позволяет (например, в случае
поверхностей общего положения) уменьшить набор из 36 неизвестных до 16
переменных
\begin{equation}
   t_1, t_2, t_3, t_4, t_5, t_7, t_8, t_{11}, t_{19},t_{23}, t_{24}, t_{27}, m_1,m_2, m_3, m_4. \label{nabor_7.1}
\end{equation}


  Далее выписывается система уравнений, квадратичных (по существу)
относительно набора (\ref{nabor_7.1}). Эта система оказывается переопределенной, так как
содержит (в случае поверхностей общего положения) 50 уравнений относительно
16 неизвестных.

  В изученных случаях 2)--5) из введения построение всех решений
аналогичных систем квадратичных уравнений фактически приводило к списку всех
алгебр Ли требуемого вида. А само получение решений оказывалось задачей,
хотя и технически сложной, но реализуемой с помощью компьютерных алгоритмов.

  В случае однородных поверхностей общего положения ситуация с
решением системы квадратичных уравнений сильно усложняется.
 Р