%
%\documentclass[a4paper,11pt,twoside]{article}
%\usepackage{vvmph2}
%\usepackage{rumathbr}
%
%
%\begin{document}

\counterwithout{equation}{section}
\plang{0}\selectlanguage{russian}

\udk{517.55 + 514.74 + 004.94}
\bbk{22.161.5 + 22.151.5}
\art[Об аффинно-однородных вещественных гиперповерхностях]{Об аффинно-однородных вещественных гиперповерхностях общего положения в $\Bbb C^3$}{On Affine Homogeneous Real Hypersurfaces\plb of General position in $\Bbb C^3$}
\support{Работа частично поддержана грантом РФФИ № 14-01-00709} %команда может отсутствовать
\fio{Лобода}{Александр}{Васильевич}
\sdeg{доктор физико-математических наук}{doctor of Physical and Mathematical
Sciences} %если есть
\pos{профессор}{professor}
\emp{Воронежский государственный технический университет}
{Voronezh State Technical University }
\email{lobvgasu@yandex.ru}
\addr{ул. 20-летия Октября, 84, 394006 г. Воронеж, Российская Федерация}{20-letiya Oktyabrya St., 84, 394006 Voronezh, Russian Federation}

\fio{Шиповская}{Александра}{Владимировна}
\pos{аспирант}{Postgraduate Student}
\emp{Воронежский государственный технический университет}
{Voronezh State Technical University }
\email{al.shipovskaia@gmail.com}
\addr{ул. 20-летия Октября, 84, 394006 г. Воронеж, Российская Федерация}{20-letiya Oktyabrya St., 84, 394006 Voronezh, Russian Federation}

\maketitle

\begin{abstract}
  В статье развиваются различные подходы к задаче
описания аффинно-однородных вещественных гиперповерхностей 3-мерного
комплексного пространства. Описывается схема изучения задачи об однородности,
использующая канонические уравнения изучаемых многообразий и
матричные алгебры Ли.

   В рамках этой схемы,
позволившей ранее получить полные классификационные результаты для нескольких
типов однородных поверхностей, построены
примеры матричных алгебр Ли, отвечающих строго псевдовыпуклым
однородным поверхностям общего положения. Приведены также примеры
однородных многообразий этого типа.

  Доказана теорема об опорном наборе коэффициентов канонического уравнения,
определяющем любую однородную поверхность из изучаемого класса. За счет
компьютерного исследования большой системы полиномиальных уравнений,
являющегося частью
обсуждаемой схемы, получен вывод о запретах на однородность, связанный
с ограничениями на некоторые коэффициенты опорного набора.

\end{abstract}

\keywords{аффинное преобразование, вещественная гиперповерхность,
каноническое уравнение поверхности, однородное многообразие, алгебра Ли,
система полиномиальных уравнений, символьные вычисления}{affine transformation, real hypersurface, canonical equation of  surface,
homogeneous manifold, Lie algebra, system of polynomial equations,
symbolic calculations}
