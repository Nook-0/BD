%\documentclass[12pt]{amsart}
%\usepackage{amsmath,amsthm}
%\usepackage[a4paper,left=2.5cm,right=2.5cm,top=3.4cm,bottom=3.0cm]{geometry}
%
%\documentclass[12pt,twoside]{article}
%\usepackage{vvmph2}
%\usepackage{rumathbr}

%\newtheoremstyle{theorem}%name
%  {10pt}		  % space above
%  {10pt}  % space below
%  {\sl}  % bofy font
%  {\parindent}     % ident - empty=no indent,  \parindent= paragraph indent
%  {\bf}  % thm head font
%  {. }    % punctuation after thm head
%  { }    % space after thm head: `` ``=normal \newline=linebreak
%  {}     % thm head specification
%\theoremstyle{theorem}
%\newtheorem{thm}{Theorem}[section]
%\newtheorem{crl}[thm]{Corollary}
%\newtheorem{prop}[thm]{Proposition}
%\newtheorem{lem}[thm]{Lemma}

%\newtheoremstyle{defi}%name
%  {10pt}		  % space above
%  {10pt}  % space below
%  {\rm}  % bofy font
%  {\parindent}     % ident - empty=no indent,  \parindent= paragraph indent
%  {\bf}  % thm head font
%  {. }    % punctuation after thm head
%  { }    % space after thm head: `` ``=normal \newline=linebreak
%  {}     % thm head specification
%\theoremstyle{defi}
%\newtheorem{dfn}[thm]{Definition}

%\newtheoremstyle{remk}%name
%  {10pt}		  % space above
%  {10pt}  % space below
%  {\rm}  % bofy font
%  {\parindent}     % ident - empty=no indent,  \parindent= paragraph indent
%  {\sl}  % thm head font
%  {. }    % punctuation after thm head
%  { }    % space after thm head: `` ``=normal \newline=linebreak
%  {}     % thm head specification
%\theoremstyle{remk}
%\newtheorem{rmk}[thm]{Remark}
\newtheorem*{ntn}{Notation}

%\def\proofname{\indent {\sl Proof.}}

%%%% Local Definitions start here
%\usepackage{amssymb,mathrsfs}
%\usepackage{showkeys}
%\usepackage{bm}
%\usepackage[dvipdfmx]{hyperref}
%\allowdisplaybreaks[4]

\numberwithin{equation}{section}
%\newcommand{\meas}{m}
%\newcommand{\ind}[1]{\mathbf{1}_{#1}}%\textrm{\rmfamily\bfseries\upshape 1}_{#1}}
%%%% End of Local Definitions

%\subjclass[2010]{31C05, 31C25, 60J45}

%%% Theorem Like Envirouments



%\begin{document}

\plang{1}\selectlanguage{english}

\udk{517} \bbk{22.161}


\art{Equivalence of recurrence and Liouville property for symmetric Dirichlet forms}{Эквивалентность рекуррентности и лиувиллева свойства\plb для симметричных форм Дирихле}
\support{JSPS Research Fellow PD (20$\cdot$6088): The author was supported by the Japan Society for the Promotion of Science\vspace*{-1.5cm}}

\fio[Наотака Кадзино]{Kajino}{Naotaka}{}
\pos{Associate Professor, Department of Mathematics, Graduate School of Science}{Доцент, факультет математики, Высшая естественнонаучная школа} \emp{Kobe University}{Kobe University} \email{nkajino@math.kobe-u.ac.jp}
\addr{Rokkodai-cho 1-1, Nada-ku, Kobe 657-8501, Japan}{Rokkodai-cho 1-1, Nada-ku, Kobe 657-8501, Japan}


\maketitle

\begin{abstract}
Given a symmetric Dirichlet form
$(\mathcal{E},\mathcal{F})$ on a (non-trivial) $\sigma$-finite measure space
$(E,\mathcal{B},\meas)$ with associated Markovian semigroup
$\{T_{t}\}_{t\in(0,\infty)}$, we prove that $(\mathcal{E},\mathcal{F})$ is
both irreducible and recurrent if and only if there is no non-constant
$\mathcal{B}$-measurable function
$u:E\to[0,\infty]$ that is \emph{$\mathcal{E}$-excessive},
i.e., such that $T_{t}u\leq u$ $\meas$-a.e.\ for any $t\in(0,\infty)$.
We also prove that these conditions are equivalent to the
equality $\{u\in\mathcal{F}_{e}\mid \mathcal{E}(u,u)=0\}=\mathbb{R}\ind{}$,
where $\mathcal{F}_{e}$ denotes the extended Dirichlet space associated with
$(\mathcal{E},\mathcal{F})$. The proof is based on simple analytic arguments
and requires no additional assumption on the state space or on the form.
In the course of the proof we also present a characterization of the
$\mathcal{E}$-excessiveness in terms of $\mathcal{F}_{e}$ and $\mathcal{E}$,
which is valid for any symmetric positivity preserving form.
\end{abstract}

\keywords{symmetric Dirichlet forms, symmetric positivity preserving forms, extended Dirichlet space, excessive functions, recurrence, Liouville property}{симметричные формы Дирихле; симметричные формы, сохраняющие положительность; расширенное пространство Дирихле, эксцессивные функции, рекуррентность, лиувиллево свойство}


%
\section{Introduction and the statement of the main theorem}\label{sec:intro}
%
Since the classical theorem of Liouville saying that there is no non-constant
bounded holomorphic function on $\mathbb{C}$, non-existence of non-constant bounded
(super-)harmonic functions on the whole space, so-called \emph{Liouville property},
has been one of the main concerns of harmonic analysis on various spaces.
One of the most well-known facts about Liouville property is that the non-existence
of non-constant bounded superharmonic functions on the whole space is equivalent to
the recurrence of the corresponding stochastic process. Such an equivalence is
known to hold for standard processes on locally compact separable metrizable spaces
by Blumenthal and Getoor \cite[Chapter II, (4.22)]{BG} and also for more general
right processes by Getoor \cite[Proposition (2.4)]{Get:LMN80}.
Getoor \cite[Proposition 2.14]{Get:Exc} provides the same kind of equivalence
in terms of excessive measures.
The purpose of this paper is to give a completely elementary proof of this
equivalence in the framework of an \emph{arbitrary} symmetric Dirichlet form
on a (non-trivial) $\sigma$-finite measure space. Our proof is purely
functional-analytic and free of topological notions on the state space,
although we need to assume the symmetry of the Dirichlet form.

In the rest of this section, we describe our setting and state the main theorem.
We fix a $\sigma$-finite measure space $(E,\mathcal{B},\meas)$ throughout this paper,
and below all $\mathcal{B}$-measurable functions are assumed to be $[-\infty,\infty]$-valued.
Let $(\mathcal{E},\mathcal{F})$ be a symmetric Dirichlet form on $L^{2}(E,\meas)$ and
let $\{T_{t}\}_{t\in(0,\infty)}$ be its
associated Markovian semigroup on $L^{2}(E,\meas)$. Let
$L_{+}(E,\meas):=\{f\mid f:E\to[0,\infty],\,f\textrm{ is }\mathcal{B}\textrm{-measurable}\}$
and
$L^{0}(E,\meas):=\{f\mid f:E\to\mathbb{R},\,f\textrm{ is }\mathcal{B}\textrm{-measurable}\}$,
where we of course identify any two $\mathcal{B}$-measurable functions which are
equal $\meas$-a.e. Let $\ind{}$ denote the constant function $\ind{}:E\to\{1\}$,
and we regard $\mathbb{R}\ind{}:=\{c\ind{}\mid c\in\mathbb{R}\}$ as
a linear subspace of $L^{0}(E,\meas)$. Also let
$L^{p}_{+}(E,\meas):=L^{p}(E,\meas)\cap L_{+}(E,\meas)$
for $p\in[1,\infty]\cup\{0\}$.
Note that $T_{t}$ is canonically extended to an operator on $L_{+}(E,\meas)$ and
also to a linear operator from
$\mathcal{D}[T_{t}]:=\{f\in L^{0}(E,\meas)\mid T_{t}|f|<\infty\ \meas\textrm{-a.e.}\}$
to $L^{0}(E,\meas)$; see Proposition \ref{prop:T-extension-pos} below.
%
\begin{dfn}\label{dfn:excessive}
$u\in L_{+}(E,\meas)$ is called \emph{$\mathcal{E}$-excessive} if and only if
$T_{t}u\leq u$ $\meas$-a.e.\ for any $t\in(0,\infty)$. Similarly,
$u\in\bigcap_{t\in(0,\infty)}\mathcal{D}[T_{t}]$ is called
\emph{$\mathcal{E}$-excessive in the wide sense} if and only if
$T_{t}u\leq u$ $\meas$-a.e.\ for any $t\in(0,\infty)$.
\end{dfn}
%
\begin{remn}
As stated in \cite{BG,CF,FOT,FT,Shigekawa:convex}, when we call a function $u$
\emph{excessive}, it is usual to assume that \emph{$u$ is non-negative},
which is why we have added \emph{``in the wide sense''}
in the latter part of Definition \ref{dfn:excessive}.
\end{remn}
%
$\mathcal{E}$-excessive functions will play the role of superharmonic functions on
the whole state space, and the main theorem of this paper (Theorem \ref{thm:Liouville})
asserts that $(\mathcal{E},\mathcal{F})$ is irreducible and recurrent if and only if
there is no non-constant $\mathcal{E}$-excessive function.

Yet another possible way of formulation of harmonicity of functions (on the whole
space $E$) is to use the extended Dirichlet space $\mathcal{F}_{e}$ associated
with $(\mathcal{E},\mathcal{F})$; $u\in\mathcal{F}_{e}$ could be called
\emph{``superharmonic''} if $\mathcal{E}(u,v)\geq 0$ for any
$v\in\mathcal{F}_{e}\cap L_{+}(E,\meas)$, and
\emph{``harmonic''} if $\mathcal{E}(u,v)=0$ for any $v\in\mathcal{F}_{e}$,
or equivalently, if $\mathcal{E}(u,u)=0$.
In fact, as a key lemma for the proof of the main theorem,
in Proposition \ref{prop:Fe-Tt} below we prove that $u\in\mathcal{F}_{e}$ is ``superharmonic''
in this sense if and only if $u$ is $\mathcal{E}$-excessive in the wide sense.
Under this formulation of harmonicity,
if $(\mathcal{E},\mathcal{F})$ is \emph{recurrent}, i.e.,
$\ind{}\in\mathcal{F}_{e}$ and $\mathcal{E}(\ind{},\ind{})=0$,
then the non-existence of non-constant harmonic
functions amounts to the equality
\begin{equation}\label{eq:irr_ker}
\{u\in\mathcal{F}_{e}\mid\mathcal{E}(u,u)=0\}=\mathbb{R}\ind{}.
\end{equation}

\={O}shima \cite[Theorem 3.1]{Oshima:LMN82} proved \eqref{eq:irr_ker}
(and the completeness of $(\mathcal{F}_{e}/\mathbb{R}\ind{},\mathcal{E})$ as well)
for the Dirichlet form associated with a symmetric Hunt process which is
\emph{recurrent in the sense of Harris}; note that the recurrence in the sense of
Harris is stronger than the usual recurrence of the associated Dirichlet form.
Fukushima and Takeda \cite[Theorem 4.2.4]{FT} (see also \cite[Theorem 2.1.11]{CF})
showed \eqref{eq:irr_ker} for irreducible recurrent symmetric Dirichlet forms
$(\mathcal{E},\mathcal{F})$ under the (only) additional assumption that $\meas(E)<\infty$.
In the recent book \cite{CF}, Chen and Fukushima has extended this result
to the case of $\meas(E)=\infty$ when $(\mathcal{E},\mathcal{F})$ is regular,
by using the theory of random time changes of Dirichlet spaces.
As part of our main theorem, we generalize \eqref{eq:irr_ker} to \emph{any}
irreducible recurrent symmetric Dirichlet form. In fact, this generalization
could be obtained (at least when $L^{2}(E,\meas)$ is separable) also by applying
the theory of regular representations of Dirichlet spaces
(see \cite[Section A.4]{FOT}) to reduce the proof to the case where
$(\mathcal{E},\mathcal{F})$ is regular.
The advantage of our proof is that it is based on totally
elementary analytic arguments and is free from any use of time changes or
regular representations of Dirichlet spaces.

Here is the statement of our main theorem. See \cite[Section 1.1]{CF} or
\cite[Section 1]{F:Fe} for basics on $\mathcal{F}_{e}$, and
\cite[Sections 1.5 and 1.6]{FOT} or \cite[Section 2.1]{CF} for details about
irreducibility and recurrence of $(\mathcal{E},\mathcal{F})$.
We remark that $\mathcal{F}_{e}\subset\bigcap_{t\in(0,\infty)}\mathcal{D}[T_{t}]$
by Lemma \ref{lem:Fe-Tt}-(1) below.
We say that $(E,\mathcal{B},\meas)$ is \emph{non-trivial} if and only if
both $\meas(A)>0$ and $\meas(E\setminus A)>0$ hold for some $A\in\mathcal{B}$,
which is equivalent to the condition that $L^{2}(E,\meas)\not\subset\mathbb{R}\ind{}$
since $(E,\mathcal{B},\meas)$ is assumed to be $\sigma$-finite.
%
\begin{thm}\label{thm:Liouville}%
Consider the following six conditions.\\
$\textrm{\bfseries\upshape 1)}$
    $(\mathcal{E},\mathcal{F})$ is both irreducible and recurrent.\\
$\textrm{\bfseries\upshape 2)}$
    $\{u\in\mathcal{F}_{e}\mid\mathcal{E}(u,u)=0\}=\mathbb{R}\ind{}$.\\
$\textrm{\bfseries\upshape 3)}$
    $\{u\in\mathcal{F}_{e}\cap L^{\infty}_{+}(E,\meas)\mid\mathcal{E}(u,u)=0\}
    =\{c\ind{}\mid c\in[0,\infty)\}$.\\
$\textrm{\bfseries\upshape 4)}$
    If $u\in\mathcal{F}_{e}$ is $\mathcal{E}$-excessive in the wide sense
    then $u\in\mathbb{R}\ind{}$.\\
$\textrm{\bfseries\upshape 5)}$
    If $u\in L^{0}_{+}(E,\meas)$ is $\mathcal{E}$-excessive
    then $u\in\mathbb{R}\ind{}$.\\
$\textrm{\bfseries\upshape 6)}$
    If $u\in\mathcal{F}_{e}\cap L^{\infty}_{+}(E,\meas)$ is
    $\mathcal{E}$-excessive then $u\in\mathbb{R}\ind{}$.\\
The three conditions
$\textrm{\bfseries\upshape 1)},\textrm{\bfseries\upshape 2)},\textrm{\bfseries\upshape 3)}$
are equivalent to each other and imply
$\textrm{\bfseries\upshape 4)},\textrm{\bfseries\upshape 5)},\textrm{\bfseries\upshape 6)}$.
\!\!If\plb $(E,\mathcal{B},\meas)$ is non-trivial, then the six conditions are all equivalent.
\end{thm}
%

The organization of this paper is as follows. In Section \ref{sec:preliminaries},
we prepare basic results about the extended space $\mathcal{F}_{e}$ and
$\mathcal{E}$-excessive functions, which are valid as long as
$(\mathcal{E},\mathcal{F})$ is a symmetric positivity preserving form.
The key results there are Propositions \ref{prop:Fe-Tt} and \ref{prop:excessive-char},
which are essentially known but seem new in the present general framework.
Furthermore Proposition \ref{prop:excessive-char} provides a characterization of
the notion of $\mathcal{E}$-excessive functions in terms of $\mathcal{F}_{e}$ and $\mathcal{E}$.
Making use of these two propositions, we show Theorem \ref{thm:Liouville}
in Section \ref{sec:proof}.
%
\section{Preliminaries: the extended (Dirichlet) space and excessive functions}\label{sec:preliminaries}
As noted in the previous section, we fix a $\sigma$-finite measure space
$(E,\mathcal{B},\meas)$ throughout this paper, and all
$\mathcal{B}$-measurable functions are assumed to be $[-\infty,\infty]$-valued.
Note that by the $\sigma$-finiteness of $(E,\mathcal{B},\meas)$ we can take
$\eta\in L^{1}(E,\meas)\cap L^{\infty}(E,\meas)$ such that $\eta>0$ $\meas$-a.e.
%
\begin{ntn}
\textup{(0)} We follow the convention that $\mathbb{N}=\{1,2,3,\dots\}$,
i.e., $0\not\in\mathbb{N}$.

\noindent
\textup{(1)} For $a,b\in[-\infty,\infty]$, we write $a\vee b:=\max\{a,b\}$,
$a\wedge b:=\min\{a,b\}$, $a^{+}:=a\vee 0$ and
$a^{-}:=-(a\wedge 0)$. For $\{a_{n}\}_{n\in\mathbb{N}}\subset[-\infty,\infty]$
and $a\in[-\infty,\infty]$,
we write $a_{n}\uparrow a$ (resp.\ $a_{n}\downarrow a$) if and only if
$\{a_{n}\}_{n\in\mathbb{N}}$ is non-decreasing (resp.\ non-increasing) and
$\lim_{n\to\infty}a_{n}=a$. We use the same notation
also for ($\meas$-equivalence classes of) $[-\infty,\infty]$-valued functions.

\noindent
\textup{(2)} As introduced before Definition \ref{dfn:excessive},
identifying any two $\mathcal{B}$-measurable functions that are equal $\meas$-a.e., we set
$L_{+}(E,\meas):=\{f\mid f:E\to[0,\infty],\,f\textrm{ is }\mathcal{B}\textrm{-measurable}\}$,
$L^{0}(E,\meas):=\{f\mid f:E\to\mathbb{R},\,f\textrm{ is }\mathcal{B}\textrm{-measurable}\}$
and $L^{p}_{+}(E,\meas):=L^{p}(E,\meas)\cap L_{+}(E,\meas)$, $p\in[1,\infty]\cup\{0\}$.
We regard $\mathbb{R}\ind{}:=\{c\ind{}\mid c\in\mathbb{R}\}$ as
a linear subspace of $L^{0}(E,\meas)$.
Let $\|\cdot\|_{p}$ denote the norm of $L^{p}(E,\meas)$ for $p\in[1,\infty]$.
Finally, let $\langle f,g\rangle:=\int_{E}fg\,d\meas$ for
$f,g\in L_{+}(E,\meas)$ and also for
$f,g\in L^{0}(E,\meas)$ with $fg\in L^{1}(E,\meas)$.
\end{ntn}
%

Recall the following definitions regarding bounded linear operators on $L^{2}(E,\meas)$.
%
\begin{dfn}\label{dfn:pp-Markov}
Let $T:L^{2}(E,\meas)\to L^{2}(E,\meas)$ be a bounded linear operator on
$L^{2}(E,\meas)$.

\noindent
\textup{(1)} $T$ is called \emph{positivity preserving} if and only if
$Tf\geq 0$ $\meas$-a.e.\ for any $f\in L^{2}_{+}(E,\meas)$.

\noindent
\textup{(2)} $T$ is called \emph{Markovian} if and only if
$0\leq Tf\leq 1$ $\meas$-a.e.\ for any $f\in L^{2}(E,\meas)$ with
$0\leq f\leq 1$ $\meas$-a.e.
\end{dfn}
%
Clearly, if $T$ is positivity preserving then so is its adjoint $T^{*}$. Note that
if $T$ is Markovian, then it is positivity preserving, $\|Tf\|_{\infty}\leq\|f\|_{\infty}$
for any $L^{2}(E,\meas)\cap L^{\infty}(E,\meas)$ and
$\|T^{*}f\|_{1}\leq\|f\|_{1}$ for any $f\in L^{1}(E,\meas)\cap L^{2}(E,\meas)$.
Moreover, using the $\sigma$-finiteness of $(E,\mathcal{B},\meas)$,
we easily have the following proposition.
%
\begin{prop}\label{prop:T-extension-pos}
Let $T:L^{2}(E,\meas)\to L^{2}(E,\meas)$ be a positivity preserving bounded
linear operator on $L^{2}(E,\meas)$.

\noindent
\textup{(1)} $T|_{L^{2}_{+}(E,\meas)}$ uniquely extends to a map
$T:L_{+}(E,\meas)\to L_{+}(E,\meas)$ such that
$Tf_{n}\uparrow Tf$ $\meas$-a.e.\ for any $f\in L_{+}(E,\meas)$ and
any $\{f_{n}\}_{n\in\mathbb{N}}\subset L_{+}(E,\meas)$ with
$f_{n}\uparrow f$ $\meas$-a.e.
Moreover, let $f,g\in L_{+}(E,\meas)$ and $a\in[0,\infty]$. Then
$T(f+g)=Tf+Tg$, $T(af)=aTf$,
$\langle Tf,g\rangle=\langle f,T^{*}g\rangle$,
and if $f\leq g$ $\meas$-a.e.\ then $Tf\leq Tg$ $\meas$-a.e.

\noindent
\textup{(2)} Let $\mathcal{D}[T]:=\{f\in L^{0}(E,\meas)\mid T|f|<\infty\ \meas\textrm{-a.e.}\}$.
Then $T:L^{2}(E,\meas)\to L^{2}(E,\meas)$ is extended to a linear operator
$T:\mathcal{D}[T]\to L^{0}(E,\meas)$ given by
$Tf:=T(f^{+})-T(f^{-})$, $f\in\mathcal{D}[T]$, so that it has the following properties:\\
\textup{(i)} If $f,g\in\mathcal{D}[T]$ and $f\leq g$ $\meas$-a.e.\ then
$Tf\leq Tg$ $\meas$-a.e.\\
\textup{(ii)} If $\{f_{n}\}_{n\in\mathbb{N}}\subset\mathcal{D}[T]$ and
$f,g\in\mathcal{D}[T]$ satisfy $\lim_{n\to\infty}f_{n}=f$ $\meas$-a.e.\ and
$|f_{n}|\leq |g|$ $\meas$-a.e.\ for any $n\in\mathbb{N}$, then
$\lim_{n\to\infty}Tf_{n}=Tf$ $\meas$-a.e.
\end{prop}
%
%\begin{proof}
%(1) By the $\sigma$-finiteness we can choose
%$\{\eta_{n}\}_{n\in\mathbb{N}}\subset L^{1}(E,\meas)$ so that
%$0\leq\eta_{n}\leq 1$ $\meas$-a.e., $n\in\mathbb{N}$ and $\eta_{n}\uparrow 1$
%$\meas$-a.e. Define $Tf$ for $f\in L_{+}(E,\meas)$ so that
%$T(\eta_{n}(f\wedge n))\uparrow Tf$ $\meas$-a.e.
%Then the standard arguments show that this operator
%$T:L_{+}(E,\meas)\to L_{+}(E,\meas)$ have the desired properties.
%The uniqueness of such $T$ is clear.
%
%\noindent
%(2) This is easily shown by the standard arguments based on (1).
%\end{proof}
%
%\begin{prop}\label{prop:T-extension}
%Let $T$ be a Markovian symmetric bounded linear operator on $L^{2}(E,\meas)$.
%Then $L^{1}\vee L^{\infty}(E,\meas)\subset\mathcal{D}[T]$ and
%$\|Tf\|_{p}\leq \|f\|_{p}$ for any $f\in L^{p}(E,\meas)$, $p=1,\infty$. Moreover
%for $f\in L^{1}\vee L^{\infty}(E,\meas)$, $Tf$ is the unique element of $L^{1}\vee L^{\infty}(E,\meas)$
%that satisfies $\langle Tf,g\rangle=\langle f,Tg\rangle$
%for any $g\in L^{1}(E,\meas)\cap L^{\infty}(E,\meas)$.
%In particular, $T$ defines a linear operator
%$T:L^{1}\vee L^{\infty}(E,\meas)\to L^{1}\vee L^{\infty}(E,\meas)$.
%\end{prop}
%
%\begin{proof}
%This is immediate from $\|Tf\|_{p}\leq\|f\|_{p}$ for
%$f\in L^{2}(E,\meas)\cap L^{p}(E,\meas)$, $p=1,\infty$.
%\end{proof}
%
Throughout the rest of this paper, we fix a closed symmetric form
$(\mathcal{E},\mathcal{F})$ on $L^{2}(E,\meas)$ together with
its associated symmetric strongly continuous contraction semigroup
$\{T_{t}\}_{t\in(0,\infty)}$ and resolvent $\{G_{\alpha}\}_{\alpha\in(0,\infty)}$
on $L^{2}(E,\meas)$; see \cite[Chapter 1.3]{FOT} for basics on closed
symmetric forms on Hilbert spaces and their associated semigroups and resolvents.
%(\emph{i.e.}, $\mathcal{E}$ is a non-negative definite symmetric bilinear form
%$\mathcal{E}:\mathcal{F}\times\mathcal{F}\to\mathbb{R}$
%defined on a dense linear subspace $\mathcal{F}$ of $L^{2}(E,\meas)$ such that
%$(\mathcal{F},\mathcal{E}+\langle\cdot,\cdot\rangle)$ is a Hilbert space).

Let us further recall the following definition.
%
\begin{dfn}\label{dfn:pform}
\textup{(1)} $(\mathcal{E},\mathcal{F})$ is called a \emph{positivity preserving form}
if and only if $u^{+}\in\mathcal{F}$ and $\mathcal{E}(u^{+},u^{+})\leq\mathcal{E}(u,u)$
for any $u\in\mathcal{F}$, or equivalently,
$T_{t}$ is positivity preserving for any $t\in(0,\infty)$.

\noindent
\textup{(2)} $(\mathcal{E},\mathcal{F})$ is called a \emph{Dirichlet form} if and only if
$u^{+}\wedge 1\in\mathcal{F}$ and $\mathcal{E}(u^{+}\wedge 1,u^{+}\wedge 1)\leq\mathcal{E}(u,u)$
for any $u\in\mathcal{F}$, or equivalently,
$T_{t}$ is Markovian for any $t\in(0,\infty)$.
\end{dfn}
%
See, e.g., \cite[Section 2]{Ouh:PA96} for the equivalences stated in Definition \ref{dfn:pform}.

In the rest of this section, we assume that $(\mathcal{E},\mathcal{F})$ is a
positivity preserving form. The following definition is standard
(see \cite[Definition 3]{Sch:extend}, \cite[Definition 1.1.4]{CF} or \cite[Definition 1.4]{F:Fe}).
%
\begin{dfn}\label{dfn:Fe}
We define the \emph{extended space $\mathcal{F}_{e}$ associated with
$(\mathcal{E},\mathcal{F})$} by
\begin{equation}\label{eq:Fe}
\mathcal{F}_{e}:=\biggl\{u\in L^{0}(E,\meas)\ \biggm|
  \begin{minipage}{230pt}
    $\lim_{n\to\infty}u_{n}=u$ $\meas$-a.e.\ for some
    $\{u_{n}\}_{n\in\mathbb{N}}\subset\mathcal{F}$ with
    $\lim_{k\wedge\ell\to\infty}\mathcal{E}(u_{k}-u_{\ell},u_{k}-u_{\ell})=0$
  \end{minipage}\biggr\}.
\end{equation}
For $u\in\mathcal{F}_{e}$, such
$\{u_{n}\}_{n\in\mathbb{N}}\subset\mathcal{F}$ as in \eqref{eq:Fe} is
called an \emph{approximating sequence for $u$}.
When $(\mathcal{E},\mathcal{F})$ is a Dirichlet form, $\mathcal{F}_{e}$ is called
the \emph{extended Dirichlet space associated with $(\mathcal{E},\mathcal{F})$}.
\end{dfn}
%
Obviously $\mathcal{F}\subset\mathcal{F}_{e}$ and $\mathcal{F}_{e}$ is a linear
subspace of $L^{0}(E,\meas)$. By virtue of \cite[Proposition 2]{Sch:Fatou},
$\mathcal{F}=\mathcal{F}_{e}\cap L^{2}(E,\meas)$, and
for $u,v\in\mathcal{F}_{e}$ with approximating sequences
$\{u_{n}\}_{n\in\mathbb{N}}$ and $\{v_{n}\}_{n\in\mathbb{N}}$, respectively,
the limit $\lim_{n\to\infty}\mathcal{E}(u_{n},v_{n})\in\mathbb{R}$
exists and is independent of particular choices of
$\{u_{n}\}_{n\in\mathbb{N}}$ and $\{v_{n}\}_{n\in\mathbb{N}}$,
as discussed in \cite[before Definition 3]{Sch:extend}.
By setting $\mathcal{E}(u,v):=\lim_{n\to\infty}\mathcal{E}(u_{n},v_{n})$,
$\mathcal{E}$ is extended to a non-negative definite symmetric bilinear form on
$\mathcal{F}_{e}$. Then it is easy to see that
$\lim_{n\to\infty}\mathcal{E}(u-u_{n},u-u_{n})=0$ for
$u\in\mathcal{F}_{e}$ and any approximating sequence
$\{u_{n}\}_{n\in\mathbb{N}}\subset\mathcal{F}$ for $u$.
Moreover, we have the following proposition due to Schmuland \cite{Sch:extend}, which is
easily proved by utilizing a version \cite[Theorem A.4.1-(ii)]{CF} of the Banach --- Saks theorem.
%
\begin{prop}[{\cite[Lemma 2]{Sch:extend}}]\label{prop:Fe-Fatou}
Let $u\in L^{0}(E,\meas)$ and $\{u_{n}\}_{n\in\mathbb{N}}\subset\mathcal{F}$
satisfy $\lim_{n\to\infty}u_{n}=u$ $\meas$-a.e.\ and
$\liminf_{n\to\infty}\mathcal{E}(u_{n},u_{n})<\infty$.
Then $u\in\mathcal{F}_{e}$,
$\mathcal{E}(u,u)\leq\liminf_{n\to\infty}\mathcal{E}(u_{n},u_{n})$, and
$\liminf_{n\to\infty}\mathcal{E}(u_{n},v)\leq\mathcal{E}(u,v)
	\leq\limsup_{n\to\infty}\mathcal{E}(u_{n},v)$
for any $v\in\mathcal{F}_{e}$.
\end{prop}
%
In particular, we easily see from Proposition \ref{prop:Fe-Fatou} that
$u^{+}\in\mathcal{F}_{e}$ and $\mathcal{E}(u^{+},u^{+})\leq\mathcal{E}(u,u)$
for any $u\in\mathcal{F}_{e}$.
%
\begin{remn}
For symmetric Dirichlet forms, the properties of $\mathcal{F}_{e}$ stated above are well-known
and most of them are proved in the textbooks \cite[Section 1.1]{CF} and \cite[Section 4.1]{FT}
and also in \cite[Section 1]{F:Fe}. In fact, we can verify similar results in a quite general
setting; see Schmuland \cite{Sch:extend} for details.
\end{remn}
%
%\begin{ntn}
%We write $\|u\|_{\mathcal{E}}:=\mathcal{E}(u,u)^{1/2}$ for $u\in\mathcal{F}_{e}$.
%\end{ntn}
%
The next proposition (Proposition \ref{prop:Fe-Tt} below) requires the following lemmas.
%
\begin{lem}\label{lem:Fe-norm}
Let $\eta\in L^{1}(E,\meas)\cap L^{2}(E,\meas)$ be such that $\eta>0$ $\meas$-a.e.,
and set
$\|u\|_{\mathcal{F}_{e}}:=\mathcal{E}(u,u)^{1/2}+\int_{E}(|u|\wedge 1)\eta\,d\meas$
for $u\in\mathcal{F}_{e}$. Then we have the following assertions:

\noindent
\textup{(1)} $\|u+v\|_{\mathcal{F}_{e}}\leq\|u\|_{\mathcal{F}_{e}}+\|v\|_{\mathcal{F}_{e}}$
and $\|au\|_{\mathcal{F}_{e}}\leq(|a|\vee 1)\|u\|_{\mathcal{F}_{e}}$
for any $u,v\in\mathcal{F}_{e}$ and any $a\in\mathbb{R}$.

\noindent
\textup{(2)} $\mathcal{F}_{e}$ is a complete metric space under the metric
$d_{\mathcal{F}_{e}}$ given by
$d_{\mathcal{F}_{e}}(u,v):=\|u-v\|_{\mathcal{F}_{e}}$.
\end{lem}
%
\begin{proof}
(1) is immediate and $d_{\mathcal{F}_{e}}$ is clearly a metric on $\mathcal{F}_{e}$.
For the proof of its completeness,
let $\{u_{n}\}_{n\in\mathbb{N}}\subset\mathcal{F}_{e}$
be a Cauchy sequence in $(\mathcal{F}_{e},d_{\mathcal{F}_{e}})$.
Noting that $\mathcal{F}$ is dense in $(\mathcal{F}_{e},d_{\mathcal{F}_{e}})$,
for each $n\in\mathbb{N}$ take $v_{n}\in\mathcal{F}$ such that
$\|v_{n}-u_{n}\|_{\mathcal{F}_{e}}\leq n^{-1}$. Then
$\{v_{n}\}_{n\in\mathbb{N}}$ is also a Cauchy sequence in
$(\mathcal{F}_{e},d_{\mathcal{F}_{e}})$.
A Borel-Cantelli argument easily yields a subsequence
$\{v_{n_{k}}\}_{k\in\mathbb{N}}$ of $\{v_{n}\}_{n\in\mathbb{N}}$
converging $\meas$-a.e.\ to some $u\in L^{0}(E,\meas)$, which means that
$u\in\mathcal{F}_{e}$ with approximating sequence
$\{v_{n_{k}}\}_{k\in\mathbb{N}}$ and hence that
$\lim_{k\to\infty}\|u-v_{n_{k}}\|_{\mathcal{F}_{e}}=0$.
The same argument also implies that every subsequence of $\{v_{n}\}_{n\in\mathbb{N}}$
admits a further subsequence converging to $u$ in
$(\mathcal{F}_{e},d_{\mathcal{F}_{e}})$,
from which $\lim_{n\to\infty}\|u-v_{n}\|_{\mathcal{F}_{e}}=0$ follows.
Thus $\lim_{n\to\infty}\|u-u_{n}\|_{\mathcal{F}_{e}}=0$.
\end{proof}
%
\begin{lem}\label{lem:Fe-Tt}
\textup{(1)} $\mathcal{F}_{e}\subset\bigcap_{t\in(0,\infty)}\mathcal{D}[T_{t}]$ and
$T_{t}(\mathcal{F}_{e})\subset\mathcal{F}_{e}$ for any $t\in(0,\infty)$.

\noindent
\textup{(2)} Let $\eta$ and $\|\cdot\|_{\mathcal{F}_{e}}$ be as in Lemma \textup{\ref{lem:Fe-norm}},
and let $u\in\mathcal{F}_{e}$. Then
$\mathcal{E}(T_{t}u,T_{t}u)\leq\mathcal{E}(u,u)$,
$\|u-T_{t}u\|_{2}^{2}\leq t\mathcal{E}(u,u)$ and
$\|T_{t}u\|_{\mathcal{F}_{e}}\leq(3+\|\eta\|_{2}\sqrt{t})\|u\|_{\mathcal{F}_{e}}$
for any $t\in(0,\infty)$, $T_{s}T_{t}u=T_{s+t}u$ for any $s,t\in(0,\infty)$,
and $\lim_{t\downarrow 0}\|u-T_{t}u\|_{\mathcal{F}_{e}}=0$.
\end{lem}
%
\begin{proof}
Let $\eta$, $\|\cdot\|_{\mathcal{F}_{e}}$ and $d_{\mathcal{F}_{e}}$
be as in Lemma \ref{lem:Fe-norm}.
First we prove (2) for $u\in\mathcal{F}$. The fourth assertion is clear.
$T_{t}u\in\mathcal{F}$ and $\mathcal{E}(T_{t}u,T_{t}u)\leq\mathcal{E}(u,u)$
for $t\in(0,\infty)$ by \cite[Lemma 1.3.3-(i)]{FOT}, and
$\lim_{t\downarrow 0}\|u-T_{t}u\|_{\mathcal{F}_{e}}=0$ by \cite[Lemma 1.3.3-(iii)]{FOT}.
Let $t\in(0,\infty)$. Noting that
$\langle f-T_{t}f,T_{t}f\rangle=\|T_{t/2}f\|_{2}^{2}-\|T_{t}f\|_{2}^{2}\geq 0$
for $f\in L^{2}(E,\meas)$, we have
$\|u-T_{t}u\|_{2}^{2}=\langle u-T_{t}u,u\rangle-\langle u-T_{t}u,T_{t}u\rangle
	\leq\langle u-T_{t}u,u\rangle\leq t\mathcal{E}(u,u)$
by \cite[Lemma 1.3.4-(i)]{FOT}. Applying these estimates to
$\|u-T_{t}u\|_{\mathcal{F}_{e}}
	\leq\mathcal{E}(u,u)^{1/2}+\mathcal{E}(T_{t}u,T_{t}u)^{1/2}+\|\eta\|_{2}\|u-T_{t}u\|_{2}$
easily yields
$\|T_{t}u\|_{\mathcal{F}_{e}}\leq(3+\|\eta\|_{2}\sqrt{t})\|u\|_{\mathcal{F}_{e}}$.

Now since $\mathcal{F}$ is dense in a complete metric space
$(\mathcal{F}_{e},d_{\mathcal{F}_{e}})$, it follows from the previous paragraph
that $T_{t}|_{\mathcal{F}}$ is uniquely extended to a continuous map
$T^{e}_{t}$ from $(\mathcal{F}_{e},d_{\mathcal{F}_{e}})$ to itself, and
then clearly $T^{e}_{t}$ is linear and the assertions of (2) are true
with $T^{e}_{t}$ in place of $T_{t}$.

Let $t\in(0,\infty)$ and $u\in\mathcal{F}_{e}\cap L_{+}(E,\meas)$.
It remains to show $T^{e}_{t}u=T_{t}u$, as $v^{+},v^{-}\in\mathcal{F}_{e}$
for $v\in\mathcal{F}_{e}$. Since
$v^{+}\wedge u\in\mathcal{F}_{e}\cap L^{2}(E,\meas)=\mathcal{F}$ and
$\mathcal{E}(v^{+}\wedge u,v^{+}\wedge u)^{1/2}\leq\mathcal{E}(v,v)^{1/2}+\mathcal{E}(u,u)^{1/2}$
for any $v\in\mathcal{F}$ by the positivity preserving property of $(\mathcal{E},\mathcal{F})$,
an application of the Banach-Saks theorem \cite[Theorem A.4.1-(ii)]{CF} assures
the existence of an approximating sequence $\{w_{n}\}_{n\in\mathbb{N}}$ for $u$
such that $0\leq w_{n}\leq u$ $\meas$-a.e. A Borel --- Cantelli argument yields
a subsequence $\{w_{n_{k}}\}_{k\in\mathbb{N}}$ such that
$\lim_{k\to\infty}T_{t}w_{n_{k}}=T^{e}_{t}u$ $\meas$-a.e.,
and $T^{e}_{t}u=T_{t}u$ follows by letting $k\to\infty$ in
$T_{t}(\inf_{j\geq k}w_{n_{j}})\leq T_{t}w_{n_{k}}\leq T_{t}u$ $\meas$-a.e.
\end{proof}
%

The following proposition (Proposition \ref{prop:Fe-Tt}), which seems new
in spite of its easiness, plays an essential role in the proof of
$\textrm{\bfseries\upshape 1)}\Rightarrow\textrm{\bfseries\upshape 2)}$
of Theorem \ref{thm:Liouville}. Proposition \ref{prop:Fe-Tt}-(2) is an extension
of a result of Chen and Kuwae \cite[Lemma 3.1]{CK:subh} for functions in $\mathcal{F}$
to those in $\mathcal{F}_{e}$, and Proposition \ref{prop:Fe-Tt}-(3) extends
a basic fact for functions in $\mathcal{F}$ to those in $\mathcal{F}_{e}$.
%
\begin{prop}\label{prop:Fe-Tt}
\textup{(1)} Let $u\in\mathcal{F}_{e}$ and $v\in\mathcal{F}$. Then
\begin{equation}\label{eq:Fe-Tt}
\lim_{t\downarrow 0}\frac{1}{t}\langle u-T_{t}u,v\rangle
    =\mathcal{E}(u,v)
    \mspace{24mu}\textrm{and}\mspace{24mu}
    \langle u-T_{t}u,v\rangle=\int_{0}^{t}\mathcal{E}(u,T_{s}v)ds,
    \mspace{12mu}t\in(0,\infty).
\end{equation}

\noindent
\textup{(2)} Let $u\in\mathcal{F}_{e}$. Then $u$ is $\mathcal{E}$-excessive in the wide sense
if and only if $\mathcal{E}(u,v)\geq 0$ for any $v\in\mathcal{F}\cap L_{+}(E,\meas)$,
or equivalently, for any $v\in\mathcal{F}_{e}\cap L_{+}(E,\meas)$.

\noindent
\textup{(3)} Let $u\in\mathcal{F}_{e}$. Then $T_{t}u=u$ for any $t\in(0,\infty)$
if and only if $\mathcal{E}(u,u)=0$.
\end{prop}
%
\begin{proof}
(1) Let $u\in\mathcal{F}_{e}$, $v\in\mathcal{F}$ and
set $\varphi(t):=\langle u-T_{t}u,v\rangle$ for $t\in[0,\infty)$, where $T_{0}u:=u$.
Then $t^{-1}|\varphi(t)|\leq\mathcal{E}(u,u)^{1/2}\mathcal{E}(v,v)^{1/2}$ for
$t\in(0,\infty)$ and $\lim_{t\downarrow 0}t^{-1}\varphi(t)=\mathcal{E}(u,v)$
if $u\in\mathcal{F}$ by \cite[Lemma 1.3.4-(i)]{FOT}, and the same are true
for $u\in\mathcal{F}_{e}$ as well by Lemma \ref{lem:Fe-Tt}.
Using Lemma \ref{lem:Fe-Tt}, we easily see also that
$\varphi'(t)=\mathcal{E}(u,T_{t}v)$ for $t\in[0,\infty)$ and that
$\varphi'$ is continuous on $[0,\infty)$, proving \eqref{eq:Fe-Tt}.

\noindent
(2) The third assertion of Proposition \ref{prop:Fe-Fatou} together with the positivity
preserving property of $(\mathcal{E},\mathcal{F})$ easily implies that
$\mathcal{E}(u,v)\geq 0$ for any $v\in\mathcal{F}\cap L_{+}(E,\meas)$ if and only if
the same is true for any $v\in\mathcal{F}_{e}\cap L_{+}(E,\meas)$.
The rest of the assertion is immediate from \eqref{eq:Fe-Tt}.

\noindent
(3) This is an immediate consequence of (2).
\end{proof}
%
The next proposition (Proposition \ref{prop:excessive-char}),
which characterizes the notion of $\mathcal{E}$-excessive functions
in terms of $\mathcal{F}_{e}$ and $\mathcal{E}$, is of independent interest.
The proof is based on a result \cite[Corollary 2.4]{Ouh:PA96} of Ouhabaz which
provides a characterization of invariance of closed convex sets for semigroups
on Hilbert spaces. A similar argument in a more general framework can be
found in Shigekawa \cite{Shigekawa:convex}.
%
\begin{prop}\label{prop:excessive-char}
Let $u\in L_{+}(E,\meas)$. Then $u$ is $\mathcal{E}$-excessive if and only if
$v\wedge u\in\mathcal{F}_{e}$ and $\mathcal{E}(v\wedge u,v\wedge u)\leq\mathcal{E}(v,v)$
for any $v\in\mathcal{F}_{e}$.
\end{prop}
%
\begin{crl}\label{cor:excessive-char}
The notion of $\mathcal{E}$-excessive functions
is determined solely by the pair $(\mathcal{F}_{e},\mathcal{E})$ of the extended space
$\mathcal{F}_{e}$ and the form
$\mathcal{E}:\mathcal{F}_{e}\times\mathcal{F}_{e}\to\mathbb{R}$.
\end{crl}
%
\begin{crl}\label{cor:excessive}
Let $u\in L_{+}(E,\meas)$ be $\mathcal{E}$-excessive and $v\in\mathcal{F}_{e}$.
Suppose $u\leq v$ $\meas$-a.e. Then $u\in\mathcal{F}_{e}$ and
$\mathcal{E}(u,u)\leq\mathcal{E}(v,v)$.
\end{crl}
%
\begin{remn}\label{rmk:excessive}
Chen and Kuwae \cite[Lemma 3.3]{CK:subh} gave a probabilistic proof of
Corollary \textup{\ref{cor:excessive}} for the Dirichlet forms
associated with symmetric right Markov processes.
\end{remn}
%
\begin{proof}[Proof of Proposition \textup{\ref{prop:excessive-char}}]
Let $K_{u}:=\{f\in L^{2}(E,\meas)\mid f\leq u\ \meas\textrm{-a.e.}\}$, which is clearly
a closed convex subset of $L^{2}(E,\meas)$. We claim that
\begin{equation}\label{eq:excessive-K}
\textrm{$u$ is $\mathcal{E}$-excessive}\quad\textrm{if and only if}\quad
  \textrm{$T_{t}(K_{u})\subset K_{u}$ for any $t\in(0,\infty)$.}
\end{equation}
Indeed, let $t\in(0,\infty)$.
If $T_{t}u\leq u$ $\meas$-a.e.\ then $T_{t}f\leq 